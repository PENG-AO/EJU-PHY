% extra contents of chap3

\section{波动延伸}

\subsection{波的解析式推导法二}

假设$t=0$时刻的波形为
\begin{equation*}
    y_0=A\sin\left(\frac{2\pi}{\lambda}x\right)
\end{equation*}
那么$t$时间后改图像会向右平移$vt$单位长度,即
\begin{equation*}
    y=A\sin\left(\frac{2\pi}{\lambda}\left(x-vt\right)\right)
    =A\sin\left(2\pi\left(\frac{x}{\lambda}-\frac{t}{T}\right)\right)
\end{equation*}

\subsection{差频公式推导}

设两束声波分别为$y_1=\sin((\omega+\theta)t)$和$y_2=\sin((\omega-\theta)t)$,其中$\omega$远大于$\theta$。那么,根据波的叠加原理可知合成波为
\begin{align*}
    y=&\sin((\omega+\theta)t)+\sin((\omega-\theta)t)\\
    =&(\sin\omega t\cos\theta t+\cos\omega t\sin\theta t)+(\sin\omega t\cos\theta t-\cos\omega t\sin\theta t)\\
    =&(2\cos\theta t)\sin\omega t
\end{align*}
此外,声音的强度与振幅的平方相关,所以合成波的波峰与波谷都会被人耳当成差频接收。因此差频的频率为合成波括号中控制振幅部分频率的两倍。
\begin{align*}
    f=&2\times\frac{\theta}{2\pi}\\
    =&\frac{\omega_1-\omega_2}{2\pi}\\
    =&f_1-f_2
\end{align*}

\subsection{光程公式推导}

使波长为$\lambda$的波在折射率为$n$的介质中传播$l$长的距离,那么在这段空间中存在$\frac{l}{\lambda^\prime}=\frac{nl}{\lambda}$个波。倘若把这些波换算到真空中($n=1$)中,则需要
\begin{equation*}
    L=\frac{l}{\lambda^\prime}\times\lambda=nl
\end{equation*}
的距离。称$L$为光程,满足$L=nl$。

\subsection{双缝干涉光程差推导}

\begin{equation*}
    |S_1P-S_2P|=S_2H\approx d\sin\theta\approx d\tan\theta=\frac{dx}{l}
\end{equation*}

\subsection{薄膜干涉光程差推导}

根据三角关系以及折射定律可知
\begin{equation*}
    \begin{cases}
        AH=AB\sin\phi\\
        BD=AB\sin\theta\\
        \sin\theta=n\sin\phi
    \end{cases}
\end{equation*}
即$n\cdot AH=BD$,$AH$与$BD$等光程。也就是说两束光到B点前的光程差为$HC+BC$。在此,将$BC$向下对称翻转得到$BC^\prime$,因而
\begin{equation*}
    HC+BC=HC+BC^\prime=HC^\prime=2d\cos\phi
\end{equation*}
由于这个距离是折射率为$n$的介质中的,所以其真空中等效的光程为$2nd\cos\phi$。

\subsection{牛顿环干涉光程差推导}

结合勾股定理关于$R$、$r$、$d$列方程即可。
\begin{align*}
    R^2=&r^2+(R-d)^2\\
    r^2=&(2R-d)d\\
    \downarrow&\quad R\gg d\\
    r^2=&2Rd
\end{align*}
