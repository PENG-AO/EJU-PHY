% extra contents of chap1

\section{力学延伸}

\subsection{平抛运动抛物线验证}

\begin{equation*}
    \begin{cases}
        x=v_0t\\
        y=\frac12gt^2
    \end{cases}\implies
    y=\frac{g}{2{v_0}^2}x^2
\end{equation*}

\subsection{动能推导}

\begin{equation*}
    E_k=\int\vec{F}\cdot d\vec{s}
        =\int m\vec{a}\cdot d\vec{s}
        =\int m\vec{v}\cdot d\vec{v}
        =\frac12mv^2
\end{equation*}

\subsection{机械能守恒推导}

\begin{align*}
    \int_{t_1}^{t_2}mv\frac{dv}{dt}dt&=\int_{t_1}^{t_2}F\frac{dx}{dt}dt\\
    m\int_{v_1}^{v_2}vdv&=\int_{x_1}^{x_2}Fdx\\
    m\int_{v_1}^{v_2}vdv&=\int_{x_0}^{x_2}Fdx-\int_{x_0}^{x_1}Fdx\\
    \Delta E_k&=-\Delta U\\
    \Delta E_k+\Delta U&=0
\end{align*}

\subsection{重力势能推导}

\begin{equation*}
    E_p=W=\int_{-h}^0mg\cdot dx=mgh
\end{equation*}

\subsection{弹力势能推导}

\begin{equation*}
    E_p=W=\int_{x}^0-k\Delta x\cdot d\Delta x=\frac12kx^2
\end{equation*}

\subsection{动量与冲量关系推导}

\begin{equation*}
    \int_{t_1}^{t_2}Fdt
    =\int_{t_1}^{t_2}m\frac{dv}{dt}dt
    =\int_{v_1}^{v_2}mdv
    =mv_2-mv_1
\end{equation*}

\subsection{圆周运动向心加速度推导}

\begin{gather*}
    v=\left(\frac{dx}{dt},\frac{dy}{dt}\right)
    =(-r\omega\sin\omega t,r\omega\cos\omega t)\\
    a=\left(\frac{dv_x}{dt},\frac{dv_y}{dt}\right)
    =(-r\omega^2\cos\omega t,-r\omega^2\sin\omega t)
    =-\omega^2\vec{r}
\end{gather*}

\subsection{单摆周期推导}

\begin{equation*}
    F\approx -mg\sin\theta=-mg\cdot\frac{x}{l}=-\frac{mg}{l}x
\end{equation*}

\subsection{开普勒定律推导}

将直角坐标系转换为极坐标系处理,
\begin{equation*}
    \begin{pmatrix}
        A_r\\ A_\phi
    \end{pmatrix}=
    \begin{pmatrix}
        \cos\phi & \sin\phi\\
        -\sin\phi & \cos\phi
    \end{pmatrix}
    \begin{pmatrix}
        A_x\\ A_y
    \end{pmatrix}
\end{equation*}
由关系式$x=r\cos\phi,y=r\sin\phi$重复计算可得,$r$方向和$\phi$方向的速度为:
\begin{gather*}
    v_x=\dot{x}=\dot{r}\cos\phi-r\dot{\phi}\sin\phi\\
    v_y=\dot{y}=\dot{r}\sin\phi+r\dot{\phi}\cos\phi\\
    \Downarrow\\
    v_r=v_x\cos\phi+v_y\sin\phi=\dot{r}\\
    v_\phi=-v_x\sin\phi+v_y\cos\phi=r\dot{\phi}
\end{gather*}
$r$方向和$\phi$方向的加速度为:
\begin{gather*}
    a_x=\ddot{x}=(\ddot{r}-r\dot{\phi}^2)\cos\phi-(2\dot{r}\dot{\phi}+r\ddot{\phi})\sin\phi\\
    a_y=\ddot{y}=(\ddot{r}-r\dot{\phi}^2)\sin\phi+(2\dot{r}\dot{\phi}+r\ddot{\phi})\cos\phi\\
    \Downarrow\\
    a_r=a_x\cos\phi+a_y\sin\phi=\ddot{r}-r\dot{\phi}^2\\
    a_\phi=-a_x\sin\phi+a_y\cos\phi=2\dot{r}\dot{\phi}+r\ddot{\phi}=\frac1r\frac{d}{dt}(r^2\dot{\phi})
\end{gather*}
如此一来行星的$r$方向和$\phi$方向的运动方程即为
\begin{gather*}
    m(\ddot{r}-r\dot{\phi}^2)=-G\frac{Mm}{r^2}\\
    m\frac1r\frac{d}{dt}(r^2\dot{\phi})=0
\end{gather*}
由$\phi$方向的运动方程可知$r^2\dot{\phi}$为与时间无关的常数,其形式可以用扇形面积公式的方式解释为面积速度。因此$r^2\dot{\phi}=h$,开普勒第二定律得证。

随后用$\dot{\phi}=\frac{h}{r^2}$改写$r$方向的运动方程可得
\begin{equation*}
    \ddot{r}-\frac{h^2}{r^3}=-\frac{GM}{r^2}
\end{equation*}
其中
\begin{gather*}
    \dot{r}=\frac{dr}{dt}=\frac{dr}{d\phi}\frac{d\phi}{dt}=\frac{h}{r^2}\frac{dr}{d\phi}\\
    \ddot{r}=\frac{d\dot{r}}{d\phi}\frac{d\phi}{dt}=\frac{h}{r^2}\left(\frac{d}{d\phi}\left(\frac{h}{r^2}\frac{dr}{d\phi}\right)\right)
\end{gather*}
代入上式可得
\begin{equation*}
    \frac{d}{d\phi}\left(\frac{1}{r^2}\frac{dr}{d\phi}\right)-\frac1r=\frac{-GM}{h^2}
\end{equation*}
在此将$\frac1r$置换为$u$,即$dr=\frac{-1}{u^2}du$,则有
\begin{align*}
    \frac{d}{d\phi}\left(u^2\frac{-1}{u^2}\frac{du}{d\phi}\right)-u&=\frac{-GM}{h^2}\\
    \frac{d}{d\phi}\frac{du}{d\phi}+u&=\frac{GM}{h^2}\\
    \frac{d^2u}{d\phi^2}+u&=\frac1l
\end{align*}
求解非齐次常微分方程可得
\begin{equation*}
    u=A\cos(\phi+\phi_0)+\frac1l
\end{equation*}
取$\phi_0=0$,设$A=\frac{\varepsilon}{l}$,整理可得
\begin{equation*}
    r=\frac{l}{\varepsilon\cos\phi+1}
\end{equation*}
即行星轨道为圆锥曲线,具体形状取决于其离心率。开普勒第一定律得证。

最后,由椭圆面积公式可求行星公转周期为
\begin{equation*}
    T=\frac{\pi ab}{h/2}
\end{equation*}
结合$l$的数学意义(半正焦弦:$al=b^2$)和物理意义($\frac1l=\frac{GM}{h^2}$)可得
\begin{equation*}
    T^2=\frac{4\pi^2 a^2b^2}{h^2}
    =\frac{4\pi^2la^3}{h^2}
    =\frac{4\pi^2}{GM}a^3
\end{equation*}
至此开普勒第三定律得证。

\subsection{万有引力势能推导}

\begin{equation*}
    E_p=W=\int_r^\infty-G\frac{Mm}{x^2}dx=-G\frac{Mm}{r}
\end{equation*}
