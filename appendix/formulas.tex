% appendix of formulas

\chapter{公式总结}

\section{三角复习}

\subsection{弧度制}

\begin{itemize}
    \item 定义:
    \begin{equation*}
        \textrm{弧度}=\frac{\textrm{弧长}}{\textrm{半径}}
    \end{equation*}
    \item 特殊值:
    \begin{center}
        \renewcommand\arraystretch{1.2}
        \begin{tabular}{c|c|c|c}
            \hline
            角度         & 弧度        & 角度         & 弧度 \\\hline
            $30^\circ$  & $\frac\pi6$ & $120^\circ$  & $\frac{2\pi}3$\\
            $45^\circ$  & $\frac\pi4$ & $135^\circ$  & $\frac{3\pi}4$\\
            $60^\circ$  & $\frac\pi3$ & $150^\circ$  & $\frac{5\pi}6$\\
            $90^\circ$  & $\frac\pi2$ & $180^\circ$  & $\pi$\\
            \hline
        \end{tabular}
    \end{center}
\end{itemize}

\subsection{三角函数}

\begin{itemize}
    \item 正弦:$\frac{对边}{斜边}$
    \item 余弦:$\frac{邻边}{斜边}$
    \item 正切:$\frac{对边}{邻边}$
    \item 特殊值:
    \begin{center}
        \renewcommand\arraystretch{1.2}
        \begin{tabular}{c|ccc}
            \hline
            角度 & 正弦 & 余弦 & 正切 \\\hline
            $0^\circ$   & $0$              & $1$              & $0$              \\
            $30^\circ$  & $\frac12$        & $\frac{\sqrt3}2$ & $\frac1{\sqrt3}$ \\
            $45^\circ$  & $\frac{\sqrt2}2$ & $\frac{\sqrt2}2$ & $1$              \\
            $60^\circ$  & $\frac{\sqrt3}2$ & $\frac12$        & $\sqrt3$         \\
            $90^\circ$  & $1$              & $0$              & $\infty$         \\
            $180^\circ$ & $0$              & $-1$             & $\infty$         \\
            $\frac\pi2-\theta$ & $\cos\theta$ & $\sin\theta$ & $\frac1{\tan\theta}$\\
            \hline
        \end{tabular}
    \end{center}
\end{itemize}

\section{微分公式}

\subsection{基本初等函数的导数}

\begin{itemize}
    \item 常函数:$y^\prime=0$
    \item 幂函数:$y^\prime=a\cdot x^{a-1}$
    \item 指数函数:$y^\prime=a^x\cdot\ln a$
    \item 对数函数:$\frac1{x\ln a}$
    \item 正弦函数:$(\sin x)^\prime=\cos x$
    \item 余弦函数:$(\cos x)^\prime=-\sin x$
    \item 正切函数:$(\tan x)^\prime=\frac1{\cos^2x}$
\end{itemize}

\subsection{组合函数的导数}

\begin{itemize}
    \item $(f(x)\pm g(x))^\prime=f^\prime(x)\pm g^\prime(x)$
    \item $(f(x)\cdot g(x))^\prime=f^\prime(x)\cdot g(x)+f(x)\cdot g^\prime(x)$
    \item $(\frac{f(x)}{g(x)})^\prime=\frac{f^\prime(x)\cdot g(x)-f(x)\cdot g^\prime(x)}{g^2(x)}$
    \item $f(g(x))^\prime=\frac{df(x)}{dg(x)}\cdot\frac{dg(x)}{dx}=f^\prime(g(x))g^\prime(x)$
\end{itemize}

\section{积分公式}

\begin{itemize}
    \item 常函数:$\int kdx=kx+C$
    \item 幂函数($a\neq-1$):$\int x^adx=\frac{x^{a+1}}{a+1}+C$
    \item 幂函数($a=-1$):$\int x^{-1}dx=\ln|x|+C$
    \item 指数函数:$\int a^xdx=\frac{a^x}{\ln a}+C$
    \item 正弦函数:$\int\sin xdx=-\cos x+C$
    \item 余弦函数:$\int\cos xdx=\sin x+C$
\end{itemize}

\section{力学公式}

\subsection{运动与力}

\paragraph{运动学基本公式}
\begin{gather*}
    v=v_0+at\\
    x=v_0t+\frac{1}{2}at^2\\
    v^2-{v_0}^2=2ax
\end{gather*}

\paragraph{运动分析}
\begin{itemize}
    \item 合成分解:$\vec{v}\rightleftharpoons\vec{v_1}+\vec{v_2}$
    \item 相对速度:$\textrm{相对}=\textrm{对象}-\textrm{参考/基准}$
\end{itemize}

\paragraph{胡克定律}
\begin{equation*}
    \vec{F}=-k\vec{\Delta x}\quad(k:\textrm{バネ定数},\Delta x:\textrm{基于原长的形变量})
\end{equation*}

\paragraph{弹簧串并联}
\begin{itemize}
    \item 串联(受力一致)
    \begin{equation*}
        \begin{cases}
            F=k_1x_1=k_2x_2\\
            F=K(x_1+x_2)
        \end{cases}\implies\frac1K=\frac1{k_1}+\frac1{k_2}
    \end{equation*}
    \item 并联(形变一致)
    \begin{equation*}
        \begin{cases}
            F=F_1+F_2=k_1x+k_2x\\F=Kx
        \end{cases}\implies K=k_1+k_2
    \end{equation*}
\end{itemize}

\paragraph{牛顿运动定律}
\begin{itemize}
    \item 惯性法则:物体\underline{不受力}或\underline{合外力为0}时,运动状态不发生改变(惯性)。
    \item 运动法则:物体受力后会产生与力\underline{同方向}的加速度。加速度大小与力的大小成正比,与物体质量成反比。
    \begin{equation*}
        \vec{a}=k\cdot\frac{\vec{F}}{m}\to
        \vec{F}=m\vec{a}
    \end{equation*}
    \item 作用力与反作用力法则:物体A向物体B施力后,会受到来自物体B的\underline{大小相等}、\underline{方向相反}(等大反向)的力。
\end{itemize}

\paragraph{力矩}
\begin{equation*}
    M=F\cdot r\cdot\sin\theta
\end{equation*}

\paragraph{刚体平衡条件}
\begin{itemize}
    \item 受力平衡:$\sum\vec{F}=0$
    \item 力矩平衡:$\sum\vec{M}=0$
\end{itemize}

\subsection{能量与动量}

\paragraph{功}
\begin{equation*}
    W=\vec{F}\cdot\vec{s}=Fs\cos\theta
\end{equation*}
\begin{itemize}
    \item 可正可负,取决于位移的方向
    \item 力与位移方向垂直时不做功,$W=0$
\end{itemize}

\paragraph{动能}
\begin{equation*}
    E_k=\frac12mv^2
\end{equation*}

\paragraph{势能}
\begin{itemize}
    \item 重力势能(重力による位置エネルギー):
    \begin{equation*}
        E_p=mgh
    \end{equation*}
    \item 弹力势能(弾性力による位置エネルギー):
    \begin{equation*}
        E_p=\frac12kx^2
    \end{equation*}    
\end{itemize}

\paragraph{机械能}
\begin{center}
    力学的エネルギー=運動エネルギー+位置エネルギー
\end{center}

\paragraph{动量守恒}
\begin{equation*}
    m_1\vec{v_1}+m_2\vec{v_2}=m_1\vec{{v_1}^\prime}+m_2\vec{{v_2}^\prime}
\end{equation*}

\paragraph{反弹系数}
\begin{equation*}
    e=-\frac{{v_1}^\prime-{v_2}^\prime}{v_1-v_2}\quad(0\le e\le1)
\end{equation*}
\begin{itemize}
    \item $e=0$:完全非弹性碰撞
    \item $0<e<1$:非弹性碰撞
    \item $e=1$:弹性碰撞(此时动能守恒)
\end{itemize}

\paragraph{一般碰撞公式}
\begin{equation*}
    \begin{cases}
        {v_1}^\prime=\frac{1}{m_1+m_2}(m_1v_1+m_2v_2-em_2(v_1-v_2))\\
        {v_2}^\prime=\frac{1}{m_1+m_2}(m_1v_1+m_2v_2+em_1(v_1-v_2))
    \end{cases}
\end{equation*}

\paragraph{固定面碰撞结论}
\begin{equation*}
    \begin{cases}
        v^\prime=ev\\
        h^\prime=e^2h
    \end{cases}
\end{equation*}

\subsection{特殊运动}

\paragraph{角速度与周期}
\begin{equation*}
    \omega = \frac{d\theta}{dt}
\end{equation*}
\begin{itemize}
    \item 单位:$rad/s$
    \item 线速度:$v=\omega r$
    \item 周期:$T=\frac{2\pi}{\omega}$
\end{itemize}

\paragraph{圆周运动加速度}
\begin{itemize}
    \item 大小:
    \begin{equation*}
        a=\omega^2r=\frac{v^2}{r}
    \end{equation*}
    \item 方向:指向圆心
\end{itemize}

\paragraph{简谐振动的运动学信息}
\begin{itemize}
    \item 位移:$x=A\sin(\omega t+\theta_0)$
    \item 速度:$v=A\omega\cos(\omega t+\theta_0)$
    \item 加速度:$a=-A\omega^2\sin(\omega t+\theta_0)=-\omega^2x$
\end{itemize}

\paragraph{回复力}
\begin{equation*}
    F=ma=-m\omega^2x=-kx\quad(k=m\omega^2)
\end{equation*}
\begin{itemize}
    \item $F\propto x$
    \item 方向指向振动中心
\end{itemize}

\paragraph{简谐振动周期}
\begin{equation*}
    T=2\pi\sqrt{\frac{m}{k}}
\end{equation*}

\paragraph{单摆周期公式}
\begin{equation*}
    T=2\pi\sqrt{\frac{l}{g}}
\end{equation*}

\paragraph{开普勒定律}
\begin{itemize}
    \item 轨道法则:行星运动在以恒星为焦点的椭圆轨道上
    \item 面积法则:行星与恒星连线的$\begin{cases}\textrm{在单位时间内扫过的面积}\\\textrm{面积速度}\end{cases}$相等
    \item 周期法则:$T^2=ka^3\quad(k:const)$
\end{itemize}

\paragraph{面积速度}
\begin{equation*}
    \frac{dS}{dt}=\frac{\frac12r(v\cdot dt\sin\theta)}{dt}=\frac12rv\sin\theta
\end{equation*}

\paragraph{万有引力公式}
\begin{equation*}
    F=G\frac{m_1m_2}{r^2}\quad(\textrm{万有引力定数:}G=6.67\times10^{-11}N\cdot m^2\cdot kg^{-2})
\end{equation*}

\paragraph{黄金代换式}
\begin{equation*}
    G\frac{Mm}{R^2}=mg\implies
    GM=gR^2
\end{equation*}

\paragraph{万有引力势能}
\begin{equation*}
    E_p=-G\frac{Mm}{r}
\end{equation*}

\section{热学公式}

\subsection{热与能量}

\paragraph{开氏温度}
\begin{center}
    絶対温度$T=t+273(K)$
\end{center}

\paragraph{热容量与比热}
\begin{itemize}
    \item 热容量:物体升高1K所需的热量
    \item 比热:单位质量的物体升高1K所需的热量,物体单位质量的热容量
\end{itemize}
\begin{gather*}
    C=m\cdot c\\
    \Delta Q=C\cdot\Delta t=c\cdot m\cdot\Delta t
\end{gather*}

\paragraph{热量守恒定律}
\begin{itemize}
    \item 系统中吸热=系统中放热
    \item $Q_\textrm{吸}=Q_\textrm{放}$
\end{itemize}

\subsection{气体分子运动}

\paragraph{波意尔查理定律}
\begin{itemize}
    \item 波意尔定律(ボイルの法則)
    \begin{equation*}
        T=const\implies P\cdot V=const
    \end{equation*}
    \item 查理定律(シャルルの法則)
    \begin{equation*}
        P=const\implies\frac{V}{T}=const
    \end{equation*}
    \item 波意尔查理定律(ボイル・シャルルの法則)
    \begin{equation*}
        \frac{PV}{T}=const
    \end{equation*}
\end{itemize}

\paragraph{理想气体状态方程}
\begin{equation*}
    PV=nRT\quad(R\textrm{:気体定数})
\end{equation*}

\paragraph{气体内能}
\begin{equation*}
    U=\frac32nRT=\frac32PV\quad(U\propto T)
\end{equation*}

\paragraph{气体做功}
\begin{equation*}
    W=F\cdot\Delta x=PS\cdot\Delta x=P\Delta V
\end{equation*}
\begin{itemize}
    \item 压缩:$\Delta U>0$
    \item 膨胀:$\Delta U<0$
\end{itemize}

\paragraph{热力学第一定律}
\begin{equation*}
    \Delta U=Q_\textrm{吸}+W_\textrm{された}
\end{equation*}

\paragraph{热效率}
\begin{equation*}
    \eta=\frac{W_\textrm{实际}}{Q_\textrm{吸}}
    =\frac{W_\textrm{した}-W_\textrm{された}}{Q_\textrm{吸}}
\end{equation*}

\section{波动公式}

\subsection{波的性质}

\paragraph{波的基本公式}
\begin{equation*}
    v=\frac{\lambda}{T}=\lambda f
\end{equation*}

\paragraph{波的解析式}
\begin{align*}
    y(x)=&A\sin\left(\omega\left(t-\frac{x}{v}\right)\right)\\
    =&A\sin\left(2\pi\left(\frac{t}{T}-\frac{x}{\lambda}\right)\right)
\end{align*}

\paragraph{波的干涉条件}
\begin{itemize}
    \item 加强:$|r_1-r_2|=2n\cdot\frac{\lambda}{2}$
    \item 减弱:$|r_1-r_2|=(2n+1)\cdot\frac{\lambda}{2}$
\end{itemize}

\paragraph{波的反射}
\begin{center}
    \renewcommand\arraystretch{1.2}
    \begin{tabular}{c|cc}
        \hline
        反射端&反射波图示&反射波相变\\\hline
        自由端&镜面对称&0相变\\
        固定端&中心对称&$\pi$相变\\\hline
    \end{tabular}
\end{center}

\paragraph{折射定律}
\begin{equation*}
    \frac{\sin\theta_1}{\sin\theta_2}=
    \frac{v_1}{v_2}=
    \frac{\lambda_1}{\lambda_2}=
    \frac{n_2}{n_1}(=n_{12})
\end{equation*}

\subsection{声波}

\paragraph{差频公式}
\begin{equation*}
    f=|f_1-f_2|
\end{equation*}

\paragraph{多普勒效应}
\begin{equation*}
    f^\prime=\frac{V\pm u}{V\pm v}f
\end{equation*}
\begin{itemize}
    \item V:波速
    \item u:观测者的速度
    \item v:波源的速度
\end{itemize}

\paragraph{弦上波速}
\begin{equation*}
    v=\sqrt{\frac{S}{\rho}}
\end{equation*}

\paragraph{弦振动频率}
\begin{itemize}
    \item $\lambda=\frac{2n}{l}$
    \item $f=\frac{nv}{2l}\implies nf_1=f_n$
\end{itemize}

\paragraph{气柱振动}
\begin{itemize}
    \item 开管振动
    \begin{itemize}
        \item $\lambda=\frac{2}{n}l$
        \item $f=\frac{nv}{2l}\implies nf_1=f_n$
    \end{itemize}
    \item 闭管振动
    \begin{itemize}
        \item $\lambda=\frac{4}{2n-1}l$
        \item $f=\frac{(2n-1)v}{4l}\implies (2n-1)f_1=f_n$
    \end{itemize}
\end{itemize}

\subsection{光波}

\paragraph{透镜公式}
\begin{equation*}
    \frac{1}{a}+\frac{1}{b}=\frac{1}{f}
\end{equation*}
\begin{itemize}
    \item 参数:
    \begin{center}
        \renewcommand\arraystretch{1.2}
        \begin{tabular}{c|cc}
            \hline
            参数&正数&负数\\\hline
            焦距f&凸透镜&凹透镜\\
            物距a&实物&虚物\\
            相距b&实相&虚像\\\hline
        \end{tabular}
    \end{center}
    \item 倍率:$m=\left\lvert\frac{b}{a}\right\rvert$
\end{itemize}

\paragraph{双缝干涉条件}
\begin{equation*}
    \frac{dx}{l}=
    \begin{cases}
        2m\cdot\frac{\lambda}{2}&\implies\textrm{明线}\\
        (2m+1)\cdot\frac{\lambda}{2}&\implies\textrm{暗线}
    \end{cases}
\end{equation*}

\paragraph{回折格子干涉条件}
\begin{equation*}
    d\sin\theta=m\lambda
\end{equation*}

\paragraph{光程公式}
\begin{equation*}
    L=nl
\end{equation*}

\paragraph{薄膜干涉条件}
\begin{equation*}
    2nd\cos\phi=
    \begin{cases}
        2m\cdot\frac{\lambda}{2}&\implies
        \textrm{0相变明线/}\pi\textrm{相变暗线}\\
        (2m+1)\cdot\frac{\lambda}{2}&\implies
        \textrm{0相变暗线/}\pi\textrm{相变明线}
    \end{cases}
\end{equation*}

\paragraph{楔形膜干涉条件}
\begin{equation*}
    2d=2x\tan\theta=
    \begin{cases}
        2m\cdot\frac{\lambda}{2}&\implies
        \textrm{0相变明线/}\pi\textrm{相变暗线}\\
        (2m+1)\cdot\frac{\lambda}{2}&\implies
        \textrm{0相变暗线/}\pi\textrm{相变明线}
    \end{cases}
\end{equation*}

\paragraph{牛顿环干涉条件}
\begin{equation*}
    2d=\frac{r^2}{R}=
    \begin{cases}
        2m\cdot\frac{\lambda}{2}&\implies
        \textrm{0相变明线/}\pi\textrm{相变暗线}\\
        (2m+1)\cdot\frac{\lambda}{2}&\implies
        \textrm{0相变暗线/}\pi\textrm{相变明线}
    \end{cases}
\end{equation*}

\section{电磁公式}

\subsection{电场}

\paragraph{库仑定律}
\begin{equation*}
    F=k\frac{Qq}{r^2}
\end{equation*}
\begin{itemize}
    \item $k=9\times10^9N\cdot m^2\cdot C^{-2}$
    \item 同性相斥,异性相吸
\end{itemize}

\paragraph{电场}
\begin{equation*}
    \vec{E}=\frac{\vec{F}}{q}
    \iff
    \vec{F}=q\vec{E}
\end{equation*}
\begin{itemize}
    \item 单位:N/C
    \item 点电荷周围的电场:$E=k\frac{Q}{r^2}$
\end{itemize}

\paragraph{电势能与电势}
\begin{equation*}
    U=qV
\end{equation*}

\paragraph{点电荷电势}
\begin{equation*}
    U=k\frac{kQ}{r}
\end{equation*}

\paragraph{电容器定义式}
\begin{equation*}
    Q=CV
\end{equation*}
\begin{itemize}
    \item V:极板两端的电势差
    \item C:电容量,日文为電気容量,单位为法拉(F)
\end{itemize}

\paragraph{平行板电容器}
\begin{equation*}
    C=\varepsilon_0\frac{S}{d}
\end{equation*}
\begin{itemize}
    \item $\varepsilon_0$:绝对诱电率
    \item 极板间电场:$E=\frac{V}{d}=\frac{Q}{\varepsilon_0S}$
\end{itemize}

\paragraph{电容器串联}
\begin{itemize}
    \item 电荷相等
    \item $\frac{Q}{C_1}+\frac{Q}{C_2}=V$
    \item $\frac{Q}{C}=V$
    \item $\frac{1}{C}=\frac{1}{C_1}+\frac{1}{C_2}$
\end{itemize}

\paragraph{电容器并联}
\begin{itemize}
    \item 电压相等
    \item $VC_1+VC_2=Q$
    \item $VC=Q$
    \item $C=C_1+C_2$
\end{itemize}

\paragraph{电容器储能}
\begin{equation*}
    U=\frac12QV=\frac12CV^2=\frac{Q^2}{2C}
\end{equation*}

\paragraph{极板间引力}
\begin{equation*}
    F=QE_\textrm{上}=\frac12QE
\end{equation*}

\subsection{电流}

\paragraph{欧姆定律}
\begin{equation*}
    V=IR
\end{equation*}
\begin{itemize}
    \item 电阻R:日文为電気抵抗
    \begin{itemize}
        \item 单位:欧姆($\Omega$)
        \item 串联:$R=R_1+R_2$
        \item 并联:$\frac{1}{R}=\frac{1}{R_1}+\frac{1}{R_2}$
    \end{itemize}
    \item 焦耳热
    \begin{equation*}
        Q=VIt=I^2Rt=\frac{V^2}{R}t
    \end{equation*}
    \item (电)功率
    \begin{equation*}
        W=VI=I^2R=\frac{V^2}{R}
    \end{equation*}
\end{itemize}

\paragraph{基尔霍夫定律}
\begin{itemize}
    \item 电流表述:回路中任意一个节点都满足$I_\textrm{流入}=I_\textrm{流出}$
    \item 电压表述:任意一个闭合回路都满足$V_\textrm{下降}=V_\textrm{上升}$
\end{itemize}

\paragraph{惠斯通电桥}
\begin{equation*}
    \frac{R_1}{R_2}=\frac{R_3}{R_4}
\end{equation*}

\subsection{磁场}

\textbackslash\textbackslash TODO

\subsection{电磁感应}

\textbackslash\textbackslash TODO
