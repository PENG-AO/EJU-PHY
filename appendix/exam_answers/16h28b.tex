% brief answers of 16h28b

\section{16年第二回(H28B)}

\paragraph{1. (\hyperref[subsec:1.1.2]{1.1.2})} 略
\paragraph{2. (\hyperref[subsec:1.1.1]{1.1.1})} v-t图像中面积为路程。
\paragraph{3. (\hyperref[subsec:1.1.2]{1.1.2})}

\begin{gather*}
    \frac12m(v_0\sin\theta)^2=mgH\\
    H=\frac{(v_0\sin\theta)^2}{2g}\\
    t=2\times\frac{v_0\sin\theta}{g}\\
    L=v_0\sin\theta\times t=\frac{2v_0^2\sin\theta\cos\theta}{g}
\end{gather*}

\paragraph{4. (\hyperref[subsec:1.2.1]{1.2.1}, \hyperref[subsec:1.2.2]{1.2.2})} 略
\paragraph{5. (\hyperref[subsec:1.2.2]{1.2.2})}

\begin{equation*}
    \begin{cases}
        mv_0=2mv_B\cos\theta+mv_A\cos\theta\\
        2mv_B\sin\theta=mv_A\sin\theta
    \end{cases}
\end{equation*}

\paragraph{6. (\hyperref[subsec:1.3.1]{1.3.1})}

\begin{equation*}
    \begin{cases}
        mgl\cos\theta=\frac12mv^2\\
        T-mg\cos\theta=\frac{mv^2}{l}
    \end{cases}
\end{equation*}

\paragraph{7. (\hyperref[sec:2.1]{2.1})}

\begin{gather*}
    2.1\times20\times20+20\times330+4.2\times20\times T=4.2\times100\times(20-T)\\
    T\approx 1.9
\end{gather*}

\paragraph{8. (\hyperref[subsec:2.2.2]{2.2.2})}

\begin{gather*}
    \begin{cases}
        1\times10^5\times6\times10^{-3}=nR\cdot300\\
        1\times10^5\times V=nR\cdot 400
    \end{cases}\\
    W=p\Delta V=1\times10^5\times(8\times10^{-3}-6\times10^{-3})=200J
\end{gather*}

\paragraph{9. (\hyperref[subsec:2.2.3]{2.2.3})} 略
\paragraph{10. (\hyperref[subsec:3.1.1]{3.1.1})} 略
\paragraph{11. (\hyperref[subsec:3.2.2]{3.2.2})}

\begin{gather*}
    f_A<f_b\implies f_B=f_A+n\\
    \frac{V+u}{V}f=\frac{V-u}{V}(f+n)\\
    \frac{u}{V}=\frac{n}{n+2f}
\end{gather*}

\paragraph{12. (\hyperref[subsec:3.3.1]{3.3.1})} 略
\paragraph{13. (\hyperref[subsec:4.1.1]{4.1.1})} 略
\paragraph{14. (\hyperref[subsec:4.1.2]{4.1.2})}

\begin{align*}
    W=&W_{A\to O}-W_{B\to O}\\
    =&U_A-U_B\\
    =&-2q\left(\frac{kq}{d}-\frac{kq}{5d}\right)\\
    =&-\frac85\frac{kq^2}{d}
\end{align*}

\paragraph{15. (\hyperref[subsec:4.1.3]{4.1.3})}

\begin{gather*}
    Q_P=V_1C\\
    (V-0)C-(V_2-V)C=Q_P\\
    2VC-V_2C=V_1C\\
    V=\frac12(V_1+V_2)
\end{gather*}

\paragraph{16. (\hyperref[subsec:4.2.2]{4.2.2})} 略
\paragraph{17. (\hyperref[subsec:4.2.2]{4.2.2})} 略
\paragraph{18. (\hyperref[subsec:4.3.2]{4.3.2})}

\begin{equation*}
    F_x=\frac{\mu I}{2\pi\sqrt{a^2+x^2}}\times I\times l\times\frac{-x}{\sqrt{a^2+x^2}}=\frac{\mu I^2l}{2\pi}\frac{-x}{a^2+x^2}
\end{equation*}

\paragraph{19. (\hyperref[sec:5.2]{5.2})}

\begin{equation*}
    T=10\times7\times10^8=7\times10^9
\end{equation*}
