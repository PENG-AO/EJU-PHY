% brief answers of 19r01a

\section{19年第一回(R01A)}

\paragraph{1. (\hyperref[subsec:1.1.3]{1.1.3})}

\begin{equation*}
    m_1g\cdot l_1\sin\theta=m_2g\cdot l_2\cos\theta
\end{equation*}

\paragraph{2. (\hyperref[subsec:1.1.2]{1.1.2})}

\begin{equation*}
    \begin{cases}
        \textrm{水平}:\frac{h}{2}=\frac{v_0}{\sqrt{2}}t\\
        \textrm{竖直}:h=\frac{v_0}{\sqrt{2}}t+\frac12gt^2
    \end{cases}
\end{equation*}

\paragraph{3. (\hyperref[subsec:1.1.2]{1.1.2})}

\begin{equation*}
    \begin{cases}
        a_\textrm{下}=g\sin\theta-\mu g\cos\theta\\
        a_\textrm{上}=g\sin\theta+\mu g\cos\theta
    \end{cases}
\end{equation*}

\paragraph{4. (\hyperref[subsec:1.2.1]{1.2.1}, \hyperref[subsec:1.2.2]{1.2.2})}

\subparagraph{step1 (P接触到Q之前的瞬间)}

\begin{equation*}
    Mgl=\frac12Mv^2\implies
    v=\sqrt{2gl}
\end{equation*}

\subparagraph{step2 (P与Q碰撞)}

\begin{equation*}
    Mv=Mv_p+mv_Q
\end{equation*}

\subparagraph{step3 (P上升,Q滑出)}

\begin{equation*}
    \begin{cases}
        \frac12Mv_p^2=Mg\frac{l}{4}\\
        \frac12Mv_Q^2=\mu'mgd
    \end{cases}
\end{equation*}

\paragraph{5. (\hyperref[subsec:1.3.2]{1.3.2})}

\begin{equation*}
    \begin{cases}
        T_\textrm{振}=2\pi\sqrt{\frac{m}{k}}\xlongequal{mg=kd}2\pi\sqrt{\frac{d}{g}}\\
        T_\textrm{摆}=2\pi\sqrt{\frac{l}{g}}
    \end{cases}
\end{equation*}

\paragraph{6. (\hyperref[subsec:1.3.3]{1.3.3})}

\begin{gather*}
    \frac{GMm}{4R^2}=m\left(\frac{2\pi}{T}\right)^22R\\
    T=2\pi\sqrt{\frac{8R^3}{GM}}\xlongequal{GMm/R^2=mg}4\pi\sqrt{\frac{2R}{g}}
\end{gather*}

\paragraph{7. (\hyperref[sec:2.1]{2.1})}

\begin{gather*}
    C_1(t_1-T)=C_2(T-t_2)\\
    T=\frac{C_1t_1+C_2t_2}{C_1+C_2}\\
    Q=C_1(t_1-T)=\frac{C_1C_2(t_1-t_2)}{C_1+C_2}
\end{gather*}

\paragraph{8. (\hyperref[subsec:2.2.1]{2.2.1})}

\begin{equation*}
    \begin{cases}
        p_A\cdot(S_AL_A)=nRT_A\\
        p_B\cdot(S_BL_B)=nRT_B
    \end{cases}\xLongrightarrow{p_AS_A=p_BS_B}
    \frac{T_B}{T_A}=\frac{L_B}{L_A}
\end{equation*}

\paragraph{9. (\hyperref[subsec:2.2.2]{2.2.2})}

\begin{align*}
    \Delta U=&Q_\textrm{吸}+W_\textrm{された}\\
    Q_\textrm{吸}=&\Delta U-W_\textrm{された}\\
    =&\frac32(3p_0V_0-3p_0V_0)-\frac12\cdot2V_0(p_0+3p_0)\\
    =&-4p_0V_0
\end{align*}

\paragraph{10. (\hyperref[subsec:3.1.1]{3.1.1})} 略
\paragraph{11. (\hyperref[subsec:3.2.3]{3.2.3})} 略
\paragraph{12. (\hyperref[subsec:3.1.1]{3.1.1})} 在A处列一般折射情况,在B处列全反射临界条件即可。
\paragraph{13. (\hyperref[subsec:4.1.1]{4.1.1})} 经分析可知只有$x<0$的范围内会出现满足条件的点。

\begin{gather*}
    -\frac{kq}{x^2}+\frac{4kq}{(a-x)^2}=0\\
    x=-a,\frac{a}{3}(\textrm{舍})
\end{gather*}

\paragraph{14. (\hyperref[subsec:4.1.1]{4.1.1})} 略
\paragraph{15. (\hyperref[subsec:4.2.1]{4.2.1})}

\begin{gather*}
    \frac1R+\frac{1}{3k}=\frac{1}{\frac{9}{6m}-0.5k}\implies
    R=1.5k\\
    I_R=6m\times\frac{3k}{1.5k+3k}=4m\\
    P=I_R^2R=24m
\end{gather*}

\paragraph{16. (\hyperref[subsec:4.1.3]{4.1.3})}

\begin{gather*}
    \frac1C=\frac1{C_1}+\frac1{C_2}\implies C=0.5\mu\\
    U_C=\frac12CV^2=0.25\mu\\
    W_V=QV=0.5\mu\\
    W_R=W_V-U_C=0.5\mu-0.25\mu=0.25\mu
\end{gather*}

\paragraph{17. (\hyperref[subsec:4.3.1]{4.3.1})} 略
\paragraph{18. (\hyperref[subsec:4.3.3]{4.3.3})} 略
\paragraph{19.}

\begin{equation*}
    \begin{cases}
        \textrm{质子}:2u+1d=1\\
        \textrm{中子}:1u+2d=0
    \end{cases}
\end{equation*}
