% brief answers of 23r05b

\section{23年第二回(R05B)}

\paragraph{1. (\hyperref[subsec:刚体与力]{刚体与力})}

\begin{equation*}
    20\times1.2=W\times0.3+10\times0.6
\end{equation*}

\paragraph{2. (\hyperref[subsec:运动与力]{运动与力})} v-t图像正面积为位移。

\paragraph{3. (\hyperref[subsec:能量]{能量})} F-x图像正面积为做功,结合动能定理可求。

\paragraph{4. (\hyperref[subsec:能量]{能量}, \hyperref[subsec:动量]{动量})}

\begin{equation*}
    \begin{cases}
        mv=MV&\implies\frac{v}{V}=\frac{M}{m}\\
        E_k=\frac{p^2}{2m}&\implies\frac{e}{E}=\frac{m}{M}
    \end{cases}
\end{equation*}

\paragraph{5. (\hyperref[subsec:简谐振动]{简谐振动})} 弹簧振子的振动中心为受力平衡处。

\paragraph{6. (\hyperref[subsec:天体运动]{天体运动})}

\begin{equation*}
    \frac{\text{周期}^2}{\text{半径}^3}=\frac{27^2}{60^3}=\frac{T^2}{5^3}
    \implies T=\frac{3\sqrt3}{8}\text{(day)}\approx 15.58\text{(hour)}
\end{equation*}

\paragraph{7. (\hyperref[sec:热与能量]{热与能量})}

\begin{equation*}
    120\times(20-13)+4.2\times400\times(20-13)=c\times200\times(90-20)
\end{equation*}

\paragraph{8. (\hyperref[subsec:气体状态变化]{气体状态变化})}

\begin{equation*}
    \begin{cases}
        p_0V_0&=nRT_0\\
        p_1\cdot1.3V_0&=nRT_1
    \end{cases}
\end{equation*}

\paragraph{9. (\hyperref[subsec:热力学第一定律]{热力学第一定律})} 略

\paragraph{10. (\hyperref[subsec:波的传播]{波的传播})} 略

\paragraph{11. (\hyperref[subsec:多普勒效应]{多普勒效应})}

\begin{equation*}
    f=f_A-f_B=\frac{V}{V-v}f_0-\frac{V}{V+2v}f_0
\end{equation*}

\paragraph{12. (\hyperref[subsec:光的干涉]{光的干涉})}

\begin{gather*}
    \theta_A=\theta_B=\pi\\
    2nd=(2m+1)\frac\lambda2\\
    d=\frac{(2m+1)\lambda}{4n}
\end{gather*}

\paragraph{13. (\hyperref[subsec:电场]{电场})}

\begin{equation*}
    k\frac{2Q}{d^2}=k\frac{Q}{(d-a)^2}
\end{equation*}

\paragraph{14. (\hyperref[subsec:电势]{电势})}

\begin{equation*}
    W_\text{外}=U_C-U_A=\frac{kQq}{2R}-\frac{kQq}{4R}
\end{equation*}

\paragraph{15. (\hyperref[subsec:电容器]{电容器})}

\begin{equation*}
    4C(V-V_1)=2C\cdot V_1
\end{equation*}

\paragraph{16. (\hyperref[subsec:欧姆定律]{欧姆定律})} 略

\paragraph{17. (\hyperref[subsec:磁场]{磁场})}

\begin{equation*}
    \frac{I}{2\pi\cdot2r}+\frac{I_1}{2r}+\frac{I}{2\pi\cdot2r}
\end{equation*}

\paragraph{18. (\hyperref[subsec:电磁感应]{电磁感应})}

\begin{equation*}
    V=Bva, I=\frac{V}{R_1}+\frac{V}{R_2}
\end{equation*}

\paragraph{19. (\hyperref[sec:原子模型]{原子模型})}

\begin{center}
    \ce{^3_1H -> ^0_{-1}e + ^3_2He}
\end{center}
