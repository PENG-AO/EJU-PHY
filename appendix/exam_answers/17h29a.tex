% brief answers of 17h29a

\section{17年第一回(H29A)}

\paragraph{1. (\hyperref[subsec:运动与力]{运动与力})} 略
\paragraph{2. (\hyperref[subsec:运动与力]{运动与力})}

\begin{equation*}
    F_0=10+f_{A+B}=10+0.5\times(6+4)\times10=60N
\end{equation*}

\paragraph{3. (\hyperref[subsec:动量]{动量})}

\begin{gather*}
    \begin{cases}
        x=vt\\h=\frac12gt^2
    \end{cases}\implies
    \frac{x_B}{x_A}=\frac{v_B}{v_A}
    \xlongequal{\begin{cases}
        v_A=\frac{mv-emv}{2m}\\
        v_B=\frac{mv+emv}{2m}
    \end{cases}}
    \frac{1+0.6}{1-0.6}=4
\end{gather*}

\paragraph{4. (\hyperref[subsec:圆周运动]{圆周运动})}

\begin{equation*}
    \begin{cases}
        T-G=F_\textrm{向}\\
        mgl+\frac12mv_0^2=\frac12mv^2
    \end{cases}
\end{equation*}

\paragraph{5. (\hyperref[subsec:能量]{能量}, \hyperref[subsec:动量]{动量})}

\begin{gather*}
    E_k=\frac{p^2}{2m}\xLongrightarrow{p_A=p_B}
    \frac{E_{kA}}{E_{kB}}=\frac{m_B}{m_A}\\
    E_p=E_{kA}+E_{kB}\\
    E_{kA}=E_p\times\frac{m_B}{m_A+m_B}
\end{gather*}

\paragraph{6. (\hyperref[subsec:天体运动]{天体运动})}

\begin{gather*}
    \begin{cases}
        K=\frac12mv^2\\
        G\frac{Mm}{r^2}=\frac{mv^2}{r}
    \end{cases}\implies
    K=\frac{GMm}{2r}\\
    \frac{K_B}{K_A}=\frac{\frac{m_B}{r_B}}{\frac{m_A}{r_A}}=\frac14
\end{gather*}

\paragraph{7. (\hyperref[sec:热与能量]{热与能量})}

\begin{gather*}
    4.2\times200\times(20-0)=2.1\times100\times(0-(-10))+330\times(100-m)\\
    330\times(100-m)=2.1\times100\times70\\
    m\approx55.4g
\end{gather*}

\paragraph{8. (\hyperref[subsec:气体法则]{气体法则})}

\begin{align*}
    \Delta U=&Q_\textrm{吸}+W_\textrm{された}\\
    =&2.5\times10^3-(1\times10^{-1}\times10^5)\times10^{-1}\\
    =&1.5\times10^3J
\end{align*}

\paragraph{9. (\hyperref[subsec:热力学第一定律]{热力学第一定律})}

\begin{gather*}
    \Delta U=\frac32\Delta(pV)=\frac32(6p_0V_0-p_0V_0)=\frac{15}{2}p_0V_0\\
    W_\textrm{した}=\frac12\cdot(p_0+2p_0)\cdot2V_0=3p_0V_0\\
    Q_\textrm{吸}=\Delta U+W_\textrm{した}=\frac{21}{2}p_0V_0
\end{gather*}

\paragraph{10. (\hyperref[subsec:波的传播]{波的传播})} 构建密度变化和纵波疏密部之间的关联即可。
\paragraph{11. (\hyperref[subsec:共振现象]{共振现象})}

\begin{gather*}
    l=\frac{n\lambda}{2}
    \xLongrightarrow[\lambda f=v]{v=k\sqrt{mg}}
    m=\frac{4f^2l^2}{gk^2}\times\frac{1}{n^2}\\
    \frac{m'}{m}=\left(\frac{1}{\frac12}\right)^2=4
\end{gather*}

\paragraph{12. (\hyperref[subsec:衍射·反射·折射]{衍射·反射·折射})} 略
\paragraph{13. (\hyperref[subsec:电场]{电场})} 略
\paragraph{14. (\hyperref[subsec:电容器]{电容器})}

\begin{equation*}
    C'=C_\textrm{左}+C_\textrm{右}
    =\varepsilon\frac{\frac23S}{d}+\varepsilon\frac{\frac13S}{\frac12d}
    =\frac43\cdot\varepsilon\frac Sd
    =\frac43C
\end{equation*}

\paragraph{15. (\hyperref[subsec:直流电路]{直流电路})}

\begin{equation*}
    \left(\frac{6}{1+R}\right)^2\times R=5
\end{equation*}

\paragraph{16. (\hyperref[subsec:磁场]{磁场})} 略
\paragraph{17. (\hyperref[subsec:安培力]{安培力})} 偶力为作用线平行、大小相等的一对力,其力矩可由力的大小和两个力的垂直距离的乘积计算。
\paragraph{18. (\hyperref[subsec:电磁感应]{电磁感应})} 略
\paragraph{19. (\hyperref[sec:原子模型]{原子模型})}

\begin{equation*}
    \frac{113-101}{2}=6
\end{equation*}
