% brief answers of 15h27a

\section{15年第一回(H27A)}

\paragraph{1. \hyperref[subsec:1.1.3]{(1.1.3)}}

\begin{equation*}
    m_1g(l\cos\theta-d\sin\theta)=m_2g(l\cos\theta+d\sin\theta)
\end{equation*}

\paragraph{2. \hyperref[subsec:1.1.2]{(1.1.2)}} 弹性碰撞,无能量损失,可考虑对称展开。

\begin{equation*}
    \begin{cases}
        h=\frac12gt^2\\
        2L=vt
    \end{cases}
\end{equation*}

\paragraph{3. \hyperref[subsec:1.2.2]{(1.2.2)}}

\begin{gather*}
    v_A'=\frac{m_Av_A-em_Bv_A}{m_A+m_B}=\frac{m_A-m_B}{m_A+m_B}v\\
    p=m_A(v_A'-v_A)=-\frac{2m_Am_B}{m_A+m_B}v
\end{gather*}

\paragraph{4. \hyperref[subsec:1.2.1]{(1.2.1)}}

\begin{gather*}
    mg=kv\\
    P=Fv=mg\times\frac{mg}{k}=\frac{(mg)^2}{k}
\end{gather*}

\paragraph{5. \hyperref[subsec:1.2.2]{(1.2.2)}}

\begin{equation*}
        W=E_k=\frac{mv^2}{2}=\frac{p^2}{2m}\\
\end{equation*}

\paragraph{6. \hyperref[subsec:1.3.1]{(1.3.1)}}

\begin{gather*}
    F_\textrm{向}=T+G\\
    T=m\frac{v^2}{r}-mg
\end{gather*}

\paragraph{7. \hyperref[sec:2.1]{(2.1)}}

\begin{gather*}
    Q_\textrm{总}=10\times50\times60=3\times10^4J\\
    Q_\textrm{冰升温}=2.1\times100\times20=4.2\times10^3J\\
    Q_\textrm{冰融化}=330\times100=3.3\times10^4J\\
    \therefore\textrm{冰无法全部融化成水}
\end{gather*}

\paragraph{8. \hyperref[subsec:2.2.2]{(2.2.2)}} 从效果考虑对外做功的大小。

\begin{gather*}
    \Delta U=Q_\textrm{吸}+W_\textrm{された}\\
    Q_\textrm{吸}=\Delta U+W_\textrm{した}\\
    \begin{cases}
        \Delta U=\frac32nR\Delta T=nC_v\Delta T\\
        W_\textrm{した}=p_0S\Delta x+\frac12k\Delta x^2
    \end{cases}
\end{gather*}

\paragraph{9. \hyperref[subsec:2.2.3]{(2.2.3)}}

\begin{gather*}
    \begin{array}{c|ccc}
        & \Delta U & Q_\textrm{吸} & W_\textrm{された} \\\hline
        A\to B & - & - & 0 \\
        B\to C & 0 & - & + \\
        C\to A & + & + & - \\
    \end{array}\\
    \eta=\frac{W_\textrm{对外实际做功}}{Q_\textrm{纯吸热}}
    =\frac{\frac12\times\frac14p_0\times\frac14v_0}{\frac32\times p_0\times\frac14v_0+p_0\times\frac14v_0}=\frac{1}{20}
\end{gather*}

\paragraph{10. \hyperref[subsec:3.1.1]{(3.1.1)}} 略
\paragraph{11. \hyperref[subsec:3.1.2]{(3.1.2)}} 需注意此时缝间距并不远小于A到B的距离,因此不应使用双缝干涉公式。

\begin{equation*}
    |r_1-r_2|=1=2m\times\frac\lambda2\quad(m=2)\\
\end{equation*}

\paragraph{12. \hyperref[subsec:3.3.1]{(3.3.1)}}

\begin{equation*}
    \frac1a+\frac1b=\frac1f\implies
    \begin{cases}
        b=\frac{af}{a-f}=\frac{f}{1-f/a}\\
        m=\left|\frac{b}{a}\right|=\left|\frac{f}{a-f}\right|
    \end{cases}
\end{equation*}

\paragraph{13. \hyperref[subsec:4.1.1]{(4.1.1)}} 略
\paragraph{14. \hyperref[subsec:4.1.3]{(4.1.3)}} BC连通后等电势,相当于BC间被抽出,后用平行板电容器电容量定义是计算即可。

\paragraph{15. \hyperref[subsec:4.2.3]{(4.2.3)}} 设左右两部分均为顺时针电流,记作$I_1$和$I_2$,并设AB段为$I_1-I_2$。

\begin{equation*}
    \begin{cases}
        \textrm{左}: 6-10I_1-10(I_1-I_2)=0\\
        \textrm{右}: 6+10(I_1-I_2)-10I_2=0
    \end{cases}\to
    I_1-I_2=0
\end{equation*}

\paragraph{16. \hyperref[subsec:4.3.1]{(4.3.1)}}

\begin{equation*}
    \begin{cases}
        \textrm{图1}: \vec{H}=(0,0,H)\\
        \textrm{图2}: \vec{H}=0\implies \vec{H_{2//x}}=(0,0,-H)\\
        \textrm{图3}: \vec{H}=\vec{H_1}+\vec{H_{2//z}}=(0,0,H)+(H,0,0)
    \end{cases}
\end{equation*}

\paragraph{17. \hyperref[subsec:4.3.2]{(4.3.2)}} 略
\paragraph{18. \hyperref[subsec:4.4.1]{(4.4.1)}} 略
\paragraph{19. \hyperref[sec:5.2]{(5.2)}}

\begin{equation*}
    \frac{\frac{mv^2}{r}}{mvr}=\frac{\frac{ke^2}{r^2}}{\frac{nh}{2\pi}}\implies
    v=\frac{2\pi ke^2}{nh}
\end{equation*}
