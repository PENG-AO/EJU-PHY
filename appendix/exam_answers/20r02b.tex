% brief answers of 20r02b

\section{20年第二回(R02B)}

\paragraph{1. (\hyperref[subsec:1.1.3]{1.1.3})}

\begin{equation*}
    (G-f)\cdot L=G\cdot x
\end{equation*}

\paragraph{2. (\hyperref[subsec:1.1.2]{1.1.2})} 由于车体减速,因此物体合外力向左,弹簧收缩。

\begin{equation*}
    F=ma=kx
\end{equation*}

\paragraph{3. (\hyperref[subsec:1.2.2]{1.2.2})} F-t图像中面积为冲量。
\paragraph{4. (\hyperref[subsec:1.2.1]{1.2.1}), \hyperref[subsec:1.2.2]{1.2.2}} 应注意本题中为B撞击A。

\begin{gather*}
    \frac12mv^2=\mu mgL\implies L=\frac{v^2}{2\mu g}\\
    \begin{cases}
        v_A=\frac{mv+emv}{2m}\\
        v_B=\frac{mv-emv}{2m}
    \end{cases}\\
    \frac{L_A}{L_B}=\frac{v_A^2}{v_B^2}=\left(\frac{1+e}{1-e}\right)^2
\end{gather*}

\paragraph{5. (\hyperref[subsec:1.3.1]{1.3.1})}

\begin{equation*}
    \begin{cases}
        F_\textrm{向}=\frac{mv^2}{R}=mg\\
        mg(L(1-\cos60^\circ)-2R)=\frac12mv^2
    \end{cases}
\end{equation*}

\paragraph{6. (\hyperref[subsec:1.3.3]{1.3.3})}

\begin{gather*}
    5R\cdot2v=R_\textrm{远地点}\cdot v\implies R_\textrm{远地点}=10R\\
    \frac{F_\textrm{远地点}}{F_\textrm{地表}}=
    \frac{G\frac{Mm}{(10R)^2}}{G\frac{Mm}{R^2}}=\frac{1}{100}
\end{gather*}

\paragraph{7. (\hyperref[sec:2.1]{2.1})}

\begin{gather*}
    150\times t=4.2\times100\times10+330\times5\\
    t=39
\end{gather*}

\paragraph{8. (\hyperref[subsec:2.2.1]{2.2.1})}

\begin{gather*}
    \frac32n_1RT_1+\frac32n_2RT_2=\frac32(n_1+n_2)RT_3\\
    T_3=\frac{n_1T_1+n_2T_2}{n_1+n_2}
\end{gather*}

\paragraph{9. (\hyperref[subsec:2.2.3]{2.2.3})} 略
\paragraph{10. (\hyperref[subsec:3.1.1]{3.1.1})} 略
\paragraph{11. (\hyperref[subsec:3.2.3]{3.2.3})}

\begin{gather*}
    L=\frac{\lambda}{2}\xlongequal{\lambda f=v}\frac{v}{2f}=\frac{k\sqrt{mg}}{2f}\\
    \frac{L_2}{L_1}=\sqrt{\frac{m_2}{m_1}}
\end{gather*}

\paragraph{12. (\hyperref[subsec:3.3.1]{3.3.1})}

\begin{equation*}
    \frac43\cdot\sin\theta=1\cdot\sin90^\circ
\end{equation*}

\paragraph{13. (\hyperref[subsec:4.1.1]{4.1.1})}

\begin{gather*}
    \frac{F}{mg}=\frac{\frac{kQq}{4l^2\sin^2\theta}}{mg}=\tan\theta\\
    Q=\frac{4mgl^2\sin^3\theta}{kq\cos\theta}
\end{gather*}

\paragraph{14. (\hyperref[subsec:4.1.2]{4.1.2})}

\begin{gather*}
    V=\frac{kQ}{r}=\frac{2\sqrt{3}kQ}{3a}\\
    V_\textrm{中心}=3V-5V=-\frac{4\sqrt{3}kQ}{3a}
\end{gather*}

\paragraph{15. (\hyperref[subsec:4.1.3]{4.1.3})} 左侧与右侧两个电容器属于并联关系,设Q点电位为$x$,则电源上部电位为$V+x$

\begin{equation*}
    C(0-x)=2C(V+x)
\end{equation*}

\paragraph{16. (\hyperref[subsec:4.2.2]{4.2.2})} 略
\paragraph{17. (\hyperref[subsec:4.3.3]{4.3.3})} 由电荷性质和质量关系可知新粒子会飞向左侧,且不会落在AO段。因此求出落在AD段的条件即可,若不满足则一定会飞去PD段。

\begin{gather*}
    \begin{cases}
        l<2r_1\\
        \sqrt{r_1^2-(l-r_1)^2}<l
    \end{cases}\implies
    \frac{l}{2}<r_1<l\\
    r_2=2r_1\implies
    l<r_2<2l
\end{gather*}

\paragraph{18. (\hyperref[subsec:4.4.1]{4.4.1})}

\begin{equation*}
    F=IBl=\frac{V}{R}Bl=\frac{\Delta B}{\Delta t}BSl
\end{equation*}

\paragraph{19. (\hyperref[sec:5.1]{5.1})}

\begin{equation*}
    K=\frac{p^2}{2m}\xlongequal{\lambda=h/p}\frac{h^2}{2m\lambda^2}
\end{equation*}
