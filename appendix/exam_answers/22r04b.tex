% brief answers of 21r04b

\section{22年第二回(R04B)}

\paragraph{1. (\hyperref[subsec:刚体与力]{刚体与力})}

\begin{equation*}
    \begin{cases}
        \textrm{水平}&T_1\cos60^\circ=T_2\cos30^\circ\\
        \textrm{竖直}&T_1\sin60^\circ+T_2\sin30^\circ=mg
    \end{cases}
\end{equation*}

\paragraph{2. (\hyperref[subsec:运动与力]{运动与力})} 斜面方向的加速度为$a=\frac{g}{2}$,因此A到下降为止的时间为$t=\frac{v_0}{a}=\frac{2v_0}{g}$,则在t时刻下A与B的运动距离之和大于斜面长即可。

\begin{equation*}
    \begin{cases}
        l_A=v_0t-\frac12at^2\\
        l_B=\frac12at^2\\
        l_A+l_B\ge \frac{h}{\sin30^\circ}
    \end{cases}
\end{equation*}

\paragraph{3. (浮力(未列出))}

\subparagraph{漂浮}
\begin{equation*}
    \begin{cases}
        5\times10^{-1}=m_\textrm{水}+\frac{F_{\textrm{浮}(80\%V)}}{g}\\
        F_{\textrm{浮}(80\%V)}=mg
    \end{cases}
\end{equation*}

\subparagraph{浸入}
\begin{equation*}
    5.5\times10^{-1}=m_\textrm{水}+\frac{F_{\textrm{浮}(100\%V)}}{g}
\end{equation*}

\begin{equation*}
    5.5\times10^{-1}-5\times10^{-1}=0.05=\frac{F_{\textrm{浮}(20\%V)}}{g}\implies
    m=\frac{0.05}{20\%}\times80\%=0.2
\end{equation*}

\paragraph{4. (\hyperref[subsec:能量]{能量}, \hyperref[subsec:动量]{动量})}

\subparagraph{step1 (P接触到Q前的瞬间)}
\begin{equation*}
    \frac12kd^2=\frac12mv_P^2
\end{equation*}

\subparagraph{step2 (P与Q相撞)}
\begin{equation*}
    v_Q=\frac{mv_P+mv_P}{m+2m}=\frac23v_P
\end{equation*}

\subparagraph{step3 (Q飞出下落)}
\begin{equation*}
    x=v_Qt\xLongrightarrow{1/2gt^2=h}\frac{2d}{d}\sqrt{\frac{2kh}{mg}}
\end{equation*}

\paragraph{5. (\hyperref[subsec:能量]{能量})}

\begin{equation*}
    \begin{cases}
        \frac12kd^2=\mu\cdot mg\cdot L_0\\
        \frac12k(2d)^2=\mu\cdot3mg\cdot L_1
    \end{cases}
\end{equation*}

\paragraph{6. (\hyperref[subsec:圆周运动]{圆周运动})}

\begin{equation*}
    \begin{cases}
        mgr=\frac12mv^2\\
        N-mg=\frac{mv^2}{r}
    \end{cases}
\end{equation*}

\paragraph{7. (\hyperref[sec:热与能量]{热与能量})}

\begin{equation*}
    60\times(15-10)+4.2\times200\times(15-10)=C\times100\times(65-15)
\end{equation*}

\paragraph{8. (\hyperref[subsec:气体法则]{气体法则})}

\begin{equation*}
    \frac32p_0V_A=\frac32p_1(V_A+V_B)
\end{equation*}

\paragraph{9. (\hyperref[subsec:气体状态变化]{气体状态变化})}

\begin{equation*}
    \begin{array}{c|ccc}
        & \Delta U & Q_\textrm{吸} & W_\textrm{された} \\\hline
        path1 & \multirow{2}{*}{x} & 60 & -30 \\
        path2 & & \textrm{step 3} & 20 \\\hline
        & 0 & 10(\textrm{step 2}) & -10(\textrm{step 1})
    \end{array}
\end{equation*}

\paragraph{10. (\hyperref[subsec:波的传播]{波的传播})} 略

\paragraph{11. (\hyperref[subsec:共振现象]{共振现象})}

\begin{equation*}
    \begin{cases}
        case1&0.6L-0.2L=0.5\lambda_1\\
        case2&0.75L-0.25L=0.5\lambda_2
    \end{cases}\implies
    \begin{cases}
        \lambda_1=0.8L,&f_1=\frac{5V}{4L}\\
        \lambda_2=L,&f_2=\frac{V}{L}
    \end{cases}
\end{equation*}

\paragraph{12. (\hyperref[subsec:光的折射]{光的折射}, \hyperref[subsec:光的干涉]{光的干涉})}

\begin{equation*}
    \begin{cases}
        d\sin\theta=n\lambda\quad(n=1)\\
        1\cdot\sin30^\circ=n\cdot\sin\theta
    \end{cases}
\end{equation*}

\paragraph{13. (\hyperref[subsec:电场]{电场})}

\begin{equation*}
    \begin{cases}
        \vec{F_A}=\frac{kQ^2}{(\sqrt2l)^2}\left(\frac{-1}{\sqrt2},\frac{-1}{\sqrt2}\right)=\frac{kQ^2}{2l^2}\left(\frac{-1}{\sqrt2},\frac{-1}{\sqrt2}\right)\\
        \vec{F_B}=\frac{2kQ^2}{(\sqrt2l)^2}\left(\frac{1}{\sqrt2},\frac{-1}{\sqrt2}\right)=\frac{kQ^2}{2l^2}\left(\frac{2}{\sqrt2},\frac{-2}{\sqrt2}\right)
    \end{cases}\implies
    \vec{F_A}+\vec{F_B}=\frac{kQ^2}{2l^2}\left(\frac{1}{\sqrt2},\frac{-3}{\sqrt2}\right)\\
\end{equation*}

\paragraph{14. (\hyperref[subsec:电容器]{电容器})} 略

\paragraph{15. (\hyperref[subsec:直流电路]{直流电路})}

\subparagraph{case 1:A处顺时针电流$I_1$,B处顺时针电流$I_2$,C处电流向下$I_1-I_2$}
\begin{equation*}
    \begin{cases}
        E-I_1R-(I_1-I_2)R=0\\
        E+(I_1-I_2)R-I_2R=0
    \end{cases}\implies I_1-I_2=0
\end{equation*}

\subparagraph{case 2:A处顺时针电流$I_1$,B处逆时针电流$I_2$,C处电流向下$I_1+I_2$}
\begin{equation*}
    \begin{cases}
        E-I_1R-(I_1+I_2)R=0\\
        E-I_2R-(I_1+I_2)R=0
    \end{cases}\implies I_1+I_2=\frac{2E}{3R}
\end{equation*}

\paragraph{16. (\hyperref[subsec:磁场]{磁场})}

\begin{equation*}
    \frac{I_1}{2\pi(x_0-0)}=\frac{I_2}{2\pi(d-x_0)}
\end{equation*}

\paragraph{17. (\hyperref[subsec:洛伦兹力]{洛伦兹力})}

\begin{equation*}
    r=\frac{mv}{qB}\implies\frac{q}{m}=\frac{v}{Br}
\end{equation*}

\paragraph{18. (\hyperref[subsec:电磁感应]{电磁感应})} 略

\paragraph{19. (\hyperref[sec:原子模型]{原子模型})}

\begin{center}
    \ce{^4_2He + ^9_4Be -> ^1_0n + ^{12}_6C}
\end{center}
