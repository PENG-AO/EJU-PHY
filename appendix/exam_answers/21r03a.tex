% brief answers of 21r03a

\section{21年第一回(R03A)}

\paragraph{1. (\hyperref[subsec:运动与力]{运动与力})}

\begin{equation*}
    a=\frac{F_0}{m_A+m_B}=\frac{F_0-F}{m_A}
\end{equation*}

\paragraph{2. (\hyperref[subsec:动量]{动量})} F-t图像中面积为冲量。

\begin{gather*}
    I=\Delta p\\
    \frac12\left(T+\frac{T}{4}\right)F_0=mv
\end{gather*}

\paragraph{3. (\hyperref[subsec:运动与力]{运动与力})}

\subparagraph{整体法}

\begin{equation*}
    3mg-mg-2mg\cos60^\circ=6ma
\end{equation*}

\subparagraph{隔离法} 设物体左侧拉力为$T_1$,右侧拉力为$T_2$。

\begin{equation*}
    \begin{cases}
        T_1-mg=ma\\
        T_2-T_1-2mg\cos60^\circ=2ma\\
        3mg-T_2=3ma
    \end{cases}
\end{equation*}

\paragraph{4. (\hyperref[subsec:动量]{动量})}

\begin{gather*}
    v_A=\frac{m_Av_A+m_bv_B-em_B(v_A-v_B)}{m_A+m_B}=\frac{3-e}{2}\\
    e\in[0,1]\implies v_A\in\left[1,\frac32\right]
\end{gather*}

\paragraph{5. (\hyperref[subsec:简谐振动]{简谐振动})}

\begin{gather*}
    \frac12kL^2=\frac12kx^2+K(x)\\
    K(x)=\frac12kL^2-\frac12kx^2
\end{gather*}

\paragraph{6. (\hyperref[subsec:圆周运动]{圆周运动})}

\begin{gather*}
    F_\textrm{向}=T\cos\theta\\
    m\omega^2l\cos\theta=T\cos\theta\\
    T=m\omega^2l
\end{gather*}

\paragraph{7. (\hyperref[sec:热与能量]{热与能量})}

\begin{gather*}
    4.2\times120\times(20-0)=2.1\times40\times(0-(-10))+330\times(40-x)\\
    x=12
\end{gather*}

\paragraph{8. (\hyperref[subsec:热力学第一定律]{热力学第一定律})}

\begin{gather*}
    \begin{cases}
        p_0V_0=nRT_0\\
        p_0V=nRT
    \end{cases}\implies V=\frac{T}{T_0}V_0\\
    W_\textrm{された}=-p\Delta V=-p_0(V-V_0)=\frac{p_0V_0(T_0-T)}{T_0}
\end{gather*}

\paragraph{9. (\hyperref[subsec:气体状态变化]{气体状态变化})} 略
\paragraph{10. (\hyperref[subsec:波的传播]{波的传播})} 略
\paragraph{11. (\hyperref[subsec:共振现象]{共振现象})}

\begin{gather*}
    \begin{cases}
        v=\sqrt{\frac{F}{\rho}}\\
        l=\frac{\lambda}{2}
    \end{cases}\xLongrightarrow{\lambda f=v}f=\frac{1}{2l}\sqrt{\frac{F}{\rho}}\\
    \begin{cases}
        \frac{1}{2a}\sqrt{\frac{F_A}{\rho_A}}=\frac{1}{2a}\sqrt{\frac{F_B}{\rho_B}}\\
        \frac{1}{2a}\sqrt{\frac{sF_A}{\rho_A}}=\frac{1}{2b}\sqrt{\frac{F_B}{\rho_B}}
    \end{cases}\implies s=\frac{a^2}{b^2}
\end{gather*}

\paragraph{12. (\hyperref[subsec:光的折射]{光的折射})}

\begin{equation*}
    1\cdot\sin60^\circ=1.5\cdot\sin\theta
\end{equation*}

\paragraph{13. (\hyperref[subsec:电场]{电场})}

\begin{equation*}
    \sqrt{2}\frac{2kq}{a^2}+\frac{kQ}{2a^2}=0
\end{equation*}

\paragraph{14. (\hyperref[subsec:电容器]{电容器})}

\begin{gather*}
    V=\frac{Q}{C+2C}=\frac{Q}{3C}\\
    W_R=\frac{Q^2}{2C}-\frac12(C+2C)V^2=\frac{Q^2}{3C}
\end{gather*}

\paragraph{15. (\hyperref[subsec:欧姆定律]{欧姆定律})}

\begin{gather*}
    \begin{cases}
        R_1=\frac{1}{\frac{1}{R}+\frac{1}{2R}}=\frac{2R}{3}\\
        R_2=R+\frac{R}{2}=\frac{3R}{2}
    \end{cases}\\
    \frac{P_1}{P_2}\xlongequal{P=V^2/R}\frac{R_2}{R_1}=\frac94
\end{gather*}

\paragraph{16. (\hyperref[subsec:磁场]{磁场})}

\begin{equation*}
    \frac{2I}{2\pi(a-d)}=\frac{I}{2\pi(d+a)}
\end{equation*}

\paragraph{17. (\hyperref[subsec:安培力]{安培力})}

\begin{equation*}
    IBl=mg\tan\theta
\end{equation*}

\paragraph{18. (\hyperref[subsec:电磁感应]{电磁感应})} 略
\paragraph{19. (\hyperref[sec:原子模型]{原子模型})}

\begin{equation*}
    \begin{cases}
        \ce{^{235}_{92}U}:235-92=143\\
        \ce{^{140}_{54}Xe}:140-54=86\\
        \ce{^{94}_{38}Sr}:94-38=56
    \end{cases}\implies 143+1-86-56=2
\end{equation*}
