% brief answers of 23r05a

\section{23年第一回(R05A)}

\paragraph{1. (\hyperref[subsec:运动与力]{运动与力})} 水平三者抵消,竖直合力为重力。

\begin{equation*}
    3\times T\cos30^\circ=G
\end{equation*}

\paragraph{2. (\hyperref[subsec:运动与力]{运动与力})}

\begin{equation*}
    \begin{cases}
        T_0=Mg\\
        T_1=ma=m\cdot\frac{Mg}{m+M}
    \end{cases}
\end{equation*}

\paragraph{3. (\hyperref[subsec:运动与力]{运动与力})}

\begin{gather*}
    N=mg-F\sin30^\circ\\
    W=f\cdot l=-\mu l(mg-\frac12F)
\end{gather*}

\paragraph{4. (\hyperref[subsec:能量]{能量}, \hyperref[subsec:动量]{动量})}

\subparagraph{step1 (B下落)}

\begin{equation*}
    \frac12mv_B^2=mgl(1-\cos60^\circ)
\end{equation*}

\subparagraph{step2 (B碰撞A)}

\begin{equation*}
    v_A=\frac{mv_B+mv_B}{m+M}=v_B\quad(\text{速度交换})
\end{equation*}

\subparagraph{step3 (A飞出)}

\begin{equation*}
    \frac12mv^2=\frac12mv_A^2+mg\frac32l
\end{equation*}

\subparagraph{补充: 整个过程中能量无损失,可将A最终的动能来源视作完全由B初始位置处的势能提供}

\begin{equation*}
    \frac12mv^2=mg\left(1-\cos60^\circ+\frac32\right)l
\end{equation*}

\paragraph{5. (\hyperref[subsec:圆周运动]{圆周运动})}

\begin{equation*}
    F_\text{向}=m\omega^2r=G\tan\theta
\end{equation*}

\paragraph{6. (\hyperref[subsec:简谐振动]{简谐振动})} 设向上为正

\begin{gather*}
    h\propto \sin\implies a\propto -\sin\\
    F_\text{合}=N-G\\
    N=ma+G
\end{gather*}

\paragraph{7. (\hyperref[sec:热与能量]{热与能量})}

\begin{equation*}
    160\times(80-T)+4.2\times200\times(80-T)=330\times100+4.2\times100\times T
\end{equation*}

\paragraph{8. (\hyperref[subsec:气体状态变化]{气体状态变化})}

\begin{equation*}
    p_\text{大气压}\cdot V_{15}=(p_\text{大气压}+p_\text{活塞})\cdot V_{10}=(p_\text{大气压}-p_\text{活塞})\cdot V_?
\end{equation*}

\paragraph{9. (\hyperref[subsec:气体状态变化]{气体状态变化})} 略

\paragraph{10. (\hyperref[subsec:波的性质]{波的性质})} 略

\paragraph{11. (\hyperref[subsec:共振现象]{共振现象})}

\begin{gather*}
    l=\frac\lambda2\implies V=\lambda f=2lf\\
    V_1-V_0=2l(f_1-f_0)=13.6
\end{gather*}

\paragraph{12. (\hyperref[sec:光波]{光波})}

\begin{equation*}
    \frac{2D}{c}=\frac{\frac\theta2}{\omega}
\end{equation*}

\paragraph{13. (\hyperref[subsec:电场]{电场})}

\begin{equation*}
    \begin{cases}
        F_x=F_{AC}=\frac{kq(-3q)}{a^2}=-\frac{3kq^2}{a^2}\\
        F_y=F_{BC}=\frac{kq\cdot q}{(-2a)^2}=\frac{kq^2}{4a^2}
    \end{cases}\implies
    |\tan\theta|=\left|\frac{F_y}{F_x}\right|=\frac{1}{12}
\end{equation*}

\paragraph{14. (\hyperref[subsec:电容器]{电容器})}

\begin{gather*}
    \frac{1}{C_\text{下}}=\frac{1}{\frac{C}{2}}+\frac{1}{\frac{C}{3}}\implies C_\text{下}=\frac{C}{5}\\
    C_\text{上}(V-V_A)=C_\text{下}(V_A-0)
\end{gather*}

\paragraph{15. (\hyperref[subsec:欧姆定律]{欧姆定律})}

\begin{gather*}
    R_\text{合}=1k+\frac{4k}{2}+2k=5k\\
    V_A=-I\cdot 2k+10-I\cdot 1k=4
\end{gather*}

\paragraph{16. (\hyperref[subsec:磁场]{磁场})}

\begin{equation*}
    \frac{I_A}{2\pi\cdot\frac{d}{4}}=\frac{I_B}{2\pi\left(d-\frac{d}{4}\right)}
\end{equation*}

\paragraph{17. (\hyperref[subsec:安培力]{安培力})} 略

\paragraph{18. (\hyperref[subsec:交流电]{交流电})} 由图2判断磁场存在于与方向。由图3基于楞次定律判断为“减同”。

\paragraph{19. (\hyperref[sec:原子模型]{原子模型})}

\begin{equation*}
    \begin{cases}
        12&=4\alpha\\
        4&=2\alpha-\beta
    \end{cases}
\end{equation*}
