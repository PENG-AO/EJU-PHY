% brief answers of 21r04a

\section{22年第一回(R04A)}

\paragraph{1. (\hyperref[subsec:运动与力]{运动与力})}

\begin{equation*}
    T_A\cos60^\circ=T_B\cos45^\circ
\end{equation*}

\paragraph{2. (\hyperref[subsec:运动与力]{运动与力})}
设左下的角为$\theta$

\begin{equation*}
    a=\frac{mg\cos\theta-mg\sin\theta}{m+m}
\end{equation*}

\paragraph{3. (\hyperref[subsec:运动与力]{运动与力})}

\begin{equation*}
    a=\frac{mg\sin30^\circ-f}{m}=3
\end{equation*}

\paragraph{4. (\hyperref[subsec:动量]{动量})} 设向右为正方向,且认为A在撞B

\begin{equation*}
    v_A=\frac{mv+\frac32m\cdot\frac{-v}{3}-e\cdot\frac32m\cdot(v-\frac{-v}{3})}{m+\frac32m}=0
\end{equation*}

\paragraph{5. (\hyperref[subsec:圆周运动]{圆周运动})}

\begin{equation*}
    \frac{mg}{\tan\theta}=m\frac{v^2}{r}
\end{equation*}

\paragraph{6. (\hyperref[subsec:天体运动]{天体运动})}

\begin{equation*}
    \frac{mv^2}{3R}=\frac{GMm}{(3R)^2}
    \xLongrightarrow{mg=GMm/R^2}
    v=\sqrt{\frac{gR}{3}}
\end{equation*}

\paragraph{7. (\hyperref[sec:热与能量]{热与能量})}

\begin{equation*}
    t=\frac{10\times3.4\times10^2+4.2\times10\times100+10\times2.3\times10^3}{300}
\end{equation*}

\paragraph{8. (\hyperref[subsec:热力学第一定律]{热力学第一定律})} $Q_\textrm{吸}$始终恒定,到达A处前$W_\textrm{された}<0$,随后便保持为0。因此先前斜率小于其后的。

\paragraph{9. (\hyperref[subsec:热力学第一定律]{热力学第一定律}, \hyperref[subsec:气体状态变化]{气体状态变化})} 略

\paragraph{10. (\hyperref[subsec:波的传播]{波的传播})} 略

\paragraph{11. (\hyperref[subsec:共振现象]{共振现象})} 可基于3倍振动和2倍振动快速求解。

\paragraph{12. (\hyperref[subsec:光的折射]{光的折射})}

\begin{equation*}
    \sin\theta\cdot n=\sin90^\circ\cdot 1
\end{equation*}

\paragraph{13. (\hyperref[subsec:电场]{电场})}

\begin{equation*}
    \begin{cases}
        F_0=\frac{kq^2}{2^2}=\frac{kq^2}{4}\\
        F_1=\sqrt{\left(\frac{kq^2}{1^2}\right)^2+\left(\frac{kq^2}{\sqrt{3}^2}\right)^2}=\frac{\sqrt{10}}{3}kq^2
    \end{cases}
\end{equation*}

\paragraph{14. (\hyperref[subsec:电容器]{电容器})}

\begin{equation*}
    V^\prime=\frac{CV}{C+\frac58C}=\frac{8}{13}V
\end{equation*}

\paragraph{15. (\hyperref[subsec:欧姆定律]{欧姆定律})} 略

\paragraph{16. (\hyperref[subsec:磁场]{磁场})}

\begin{equation*}
    \vec{H_O}+\vec{H_P}=
    \frac{I}{2\pi}(-1,0)+\frac{I}{2\sqrt2\pi}\left(\frac{1}{\sqrt2},\frac{1}{\sqrt2}\right)=
    \left(-\frac{I}{4\pi},\frac{I}{4\pi}\right)
\end{equation*}

\paragraph{17. (\hyperref[subsec:安培力]{安培力}, \hyperref[subsec:电磁感应]{电磁感应})}

\begin{equation*}
    F=IBL=\frac{V}{2R}BL=\frac{BvL}{2R}BL=\frac{B^2vL^2}{2R}
\end{equation*}

\paragraph{18. (\hyperref[subsec:交流电]{交流电})} 略

\paragraph{19. (布拉格反射(未列出))}

\begin{equation*}
    2d\sin\theta=n\lambda\implies d=\frac{1.5\times10^{-10}}{2\times0.25}
\end{equation*}
