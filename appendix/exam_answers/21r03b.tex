% brief answers of 21r03b

\section{21年第二回(R03B)}

\paragraph{1. (\hyperref[subsec:刚体与力]{刚体与力})}

\subparagraph{step1 (受力平衡)}

\begin{equation*}
    \begin{cases}
        T\sin45^\circ=F\\
        T\cos45^\circ=G
    \end{cases}\implies F=G
\end{equation*}

\subparagraph{step2 (力矩平衡,以T的施力点为支点)}

\begin{equation*}
    G\cdot\frac{l}{2}\sin\theta=F\cdot l\cos\theta
\end{equation*}

\paragraph{2. (\hyperref[subsec:速度与加速度]{速度与加速度})} 根据导数几何性质可知:一次导数为负时函数递减,二次导数为负时函数上凸。
\paragraph{3. (\hyperref[subsec:能量]{能量}, \hyperref[subsec:动量]{动量})} F-t图像中面积为冲量。

\begin{gather*}
    I=\frac12\times2\times4=4=\Delta p\\
    K=\frac{p^2}{2m}=16
\end{gather*}

\paragraph{4. (\hyperref[subsec:能量]{能量}, \hyperref[subsec:动量]{动量})}

\subparagraph{step1 (A下滑)}

\begin{equation*}
    \frac12mv^2=mgh\implies v=v_A=v_B=\sqrt{2gh}
\end{equation*}

\subparagraph{step2 (A与B发生碰撞)}

\begin{equation*}
    v_A'=\frac{-mv_A+3mv_B+e\cdot 3m(v_B-(-v_A))}{m+3m}=2v
\end{equation*}

\subparagraph{step3 (A被反弹)}

\begin{equation*}
    \frac12{mv_A'}^2=mgH
\end{equation*}

\paragraph{5. (\hyperref[subsec:简谐振动]{简谐振动})}

\begin{equation*}
    T=2\pi\sqrt{\frac{m+M}{k}}
    \xlongequal{Mg=kL}
    2\pi\sqrt{\frac{(m+M)L}{Mg}}
\end{equation*}

\paragraph{6. (\hyperref[subsec:圆周运动]{圆周运动})}

\begin{equation*}
    \begin{cases}
        T_1-mg=\frac{mv^2}{l}\\
        T_2-mg=\frac{mv^2}{r}
    \end{cases}\xLongrightarrow{mgl=mv^2/2}
    \begin{cases}
        T_1=3mg\\
        T_2=\frac{2l+r}{r}mg
    \end{cases}\xLongrightarrow{T_2=2T_1}
    \frac{2l+r}{r}=2\times3
\end{equation*}

\paragraph{7. (\hyperref[sec:热与能量]{热与能量})}

\begin{gather*}
    0.8\times1.5\times10^3\times(400-100)=
    4.2\times1\times10^3\times(100-50)+2.3\times10^3\times m\\
    m\approx65
\end{gather*}

\paragraph{8. (\hyperref[subsec:气体法则]{气体法则})}

\begin{gather*}
    \frac32p_AV_A+\frac32p_BV_B=\frac32p(V_A+V_B)\\
    p=\frac{p_AV_A+p_BV_B}{V_A+V_B}
\end{gather*}

\paragraph{9. (\hyperref[subsec:热力学第一定律]{热力学第一定律})}

\begin{gather*}
    \Delta T=0\implies\Delta U=0\\
    \Delta U=Q_\textrm{吸}+W_\textrm{された}\\
    Q_\textrm{吸}=W_\textrm{した}<0
\end{gather*}

\paragraph{10. (\hyperref[subsec:波的传播]{波的传播})} 略
\paragraph{11. (\hyperref[subsec:共振现象]{共振现象})}

\begin{gather*}
    100.5-32.5=\frac{\lambda}{2}\implies\lambda=136\\
    f=\frac{v}{\lambda}=250
\end{gather*}

\paragraph{12. (\hyperref[subsec:光的干涉]{光的干涉})} 需明确光程中折射率的含义。

\begin{gather*}
    2nd=(2m+1)\frac{\lambda}{2}\quad(m=0)\\
    d=\frac{\lambda}{4n}\approx1.07\times10^{-7}
\end{gather*}

\paragraph{13. (\hyperref[subsec:电场]{电场})}

\begin{equation*}
    kx_0=\frac{kq^2}{(r+2x_0)^2}=\frac{kQq}{(r-2x_0)^2}
\end{equation*}

\paragraph{14. (\hyperref[subsec:电容器]{电容器})}

\begin{gather*}
    C=\frac{\varepsilon S}{d},Q=CE\\
    C'=\frac{\varepsilon S}{\frac23d}=\frac32C,
    V=\frac{Q}{C'}=\frac23E
\end{gather*}

\paragraph{15. (\hyperref[subsec:直流电路]{直流电路})} 略
\paragraph{16. (\hyperref[subsec:磁场]{磁场})}

\begin{gather*}
    \begin{cases}
        \vec{H_A}=\left(0,\frac{-I}{4\pi a}\right)\\
        \vec{H_B}=\left(0,\frac{2I}{2\pi a}\right)\\
        \vec{H_C}=\left(\frac{I'}{2\pi a},0\right)
    \end{cases}\implies
    \vec{H}=\left(\frac{I'}{2\pi a},\frac{3I}{4\pi a}\right)\\
    \frac{H_y}{H_x}=\tan30^\circ\implies
    \frac{I'}{I}=\frac{3\sqrt3}{2}
\end{gather*}

\paragraph{17. (\hyperref[subsec:安培力]{安培力})}

\begin{equation*}
    \begin{cases}
        F_1=F_{DC}-F_{AB}
        =I\frac{\mu I_0}{2\pi a}a-I\frac{\mu I_0}{4\pi a}a
        =\frac{\mu II_0}{4\pi}\\
        F_2=F_{DC}-F_{AB}
        =I\frac{\mu I_0}{4\pi a}a-I\frac{\mu I_0}{6\pi a}a
        =\frac{\mu II_0}{12\pi}\\
    \end{cases}
\end{equation*}

\paragraph{18. (\hyperref[subsec:电磁感应]{电磁感应})}

\begin{equation*}
    I_2\propto V=\frac{\Delta\Phi}{\Delta t}
    =\frac{\mu S}{2\pi r}\frac{\Delta I_1}{\Delta t}
\end{equation*}

\paragraph{19. (\hyperref[sec:波粒二象性]{波粒二象性})}

\begin{gather*}
    Ve=\frac{p^2}{2m}=\frac{h^2}{2m\lambda^2}\\
    \lambda=\sqrt{\frac{h^2}{2mVe}}
\end{gather*}
