% brief answers of 17h29b

\section{17年第一回(H29A)}

\paragraph{1. (\hyperref[subsec:1.1.2]{1.1.2})} 略
\paragraph{2. (\hyperref[subsec:1.1.2]{1.1.2})}

\begin{equation*}
    F_0+\mu mg\cos\theta=mg\sin\theta
\end{equation*}

\paragraph{3. (\hyperref[subsec:1.2.1]{1.2.1})}

\begin{equation*}
    v^2=2\cdot\frac{g}{3}\cdot h
\end{equation*}

\subparagraph{整体法}

\begin{equation*}
    a=a_A=a_B=\frac{2mg-mg}{2m+m}=\frac{g}{3}
\end{equation*}

\subparagraph{隔离法}

\begin{equation*}
    \begin{cases}
        A: T-mg=mg\\
        B: 2mg-T=2ma
    \end{cases}\implies
    a=\frac{g}{3}
\end{equation*}

\paragraph{4. (\hyperref[subsec:1.2.1]{1.2.1}, \hyperref[subsec:1.2.2]{1.2.2})}

\begin{gather*}
    p_A=p_B\implies
    \begin{cases}
        \frac{v_A}{v_B}=\frac{m_B}{m_A}\\
        \frac{K_A}{K_B}\xlongequal{K=\frac{p^2}{2m}}\frac{m_B}{m_A}
    \end{cases}
\end{gather*}

\paragraph{5. (\hyperref[subsec:1.2.1]{1.2.1})} 一般能量解法优于运动解法。

\subparagraph{能量解法}

\begin{equation*}
    \begin{cases}
        \frac12kd^2=\frac12mv_0^2\\
        \frac12kd^2-\frac12kx^2=\frac12m\left(\frac{v_0}{2}\right)^2
    \end{cases}\implies
    \frac{d^2}{d^2-x^2}=4
\end{equation*}

\subparagraph{运动解法}

\begin{gather*}
    \begin{cases}
        x=d\cos\left(\sqrt{\frac{k}{m}}t\right)\\
        v=-d\sqrt{\frac{k}{m}}\sin\left(\sqrt{\frac{k}{m}}t\right)
    \end{cases}\implies
    x(v)=\sqrt{d^2-\frac{mv^2}{k}}\\
    x(v_0)=0\implies \frac{mv_0^2}{k}=d^2\\
    x\left(\frac{v_0}{2}\right)=\frac{\sqrt{3}}{2}d
\end{gather*}

\paragraph{6. (\hyperref[subsec:1.3.3]{1.3.3})}

\begin{equation*}
    \frac12rv_0=\frac12\cdot5r\cdot\sin\theta\quad\left(\sin\theta=\frac35\right)
\end{equation*}

\paragraph{7. (\hyperref[sec:2.1]{2.1})}

\begin{gather*}
    4.2\times1000\times(30-t)=330\times200+4.2\times200\times t\\
    t=\frac{250}{21}\approx11.9
\end{gather*}

\paragraph{8. (\hyperref[subsec:2.2.1]{2.2.1})}

\begin{gather*}
    \begin{cases}
        p_1\cdot s\cdot l_A=0.2\cdot RT\\
        p_1\cdot s\cdot l_B=0.6\cdot RT
    \end{cases}\implies l_B=3l_A\\
    \begin{cases}
        p_2\cdot s\cdot 1.5l_A=0.2\cdot RT_A\\
        p_2\cdot s\cdot (l_B-0.5l_A)=0.6\cdot RT_B
    \end{cases}\\
    \frac{T_A}{T_B}=3\cdot\frac{1.5l_A}{l_B-0.5l_A}=\frac95
\end{gather*}

\paragraph{9. (\hyperref[subsec:2.2.3]{2.2.3})} 略
\paragraph{10. (\hyperref[subsec:3.1.1]{3.1.1})} 根据y-t图像和v-t图像寻找对应特征即可。
\paragraph{11. (\hyperref[subsec:3.2.2]{3.2.2})}

\subparagraph{step1 (发射)}

\begin{equation*}
    f'=\frac{V+v}{V}f_0
\end{equation*}

\subparagraph{step2 (接收)}

\begin{equation*}
    f=\frac{V}{V-v}f'=\frac{V+v}{V-v}f_0
\end{equation*}

\begin{equation*}
    v=\frac{f-f_0}{f+f_0}V
\end{equation*}

\paragraph{12. (\hyperref[subsec:3.3.2]{3.3.2})}

\begin{gather*}
    2x\tan\theta=m\lambda\\
    \Delta x=\frac{\lambda}{2\tan\theta}=\frac{\lambda L}{2D}
\end{gather*}

\paragraph{13. (\hyperref[subsec:4.1.1]{4.1.1})}

\begin{gather*}
    mg=\frac{F_1}{\tan\theta}=\frac{F_2}{\tan\theta}\\
    F_1=F_2\\
    \frac{kq^2}{a^2}=\frac{kQ^2}{(5a)^2}\\
    \frac{Q}{q}=5
\end{gather*}

\paragraph{14. (\hyperref[subsec:4.1.2]{4.1.2})} 略
\paragraph{15. (\hyperref[subsec:4.1.3]{4.1.3})}

\begin{gather*}
    \begin{cases}
        U_1=\frac{(CV)^2}{2\cdot\frac{C}{2}}=CV^2\\
        U_2=\frac12\cdot\frac{C}{2}\cdot V^2=\frac14CV^2
    \end{cases}
\end{gather*}

\paragraph{16. (\hyperref[subsec:4.2.1]{4.2.1})} 略
\paragraph{17. (\hyperref[subsec:4.3.1]{4.3.1})}

\begin{gather*}
    \vec{H_0}=\left(0,-\frac{I}{2\pi a}\right)\implies
    \begin{cases}
        \vec{H_A}=\left(0,-\frac{I}{2\pi a}\right)\\
        \vec{H_B}=\left(\frac{I}{2\pi a},0\right)\\
        \vec{H_C}=\left(0,\frac{I}{4\pi a}\right)
    \end{cases}\\
    \vec{H_1}=\vec{H_A}+\vec{H_B}+\vec{H_C}=\left(\frac{I}{2\pi a},-\frac{I}{4\pi a}\right)\\
    |\vec{H_1}|=\frac{\sqrt{5}}{2}\frac{I}{2\pi a}=\frac{\sqrt{5}}{2}|\vec{H_0}|
\end{gather*}

\paragraph{18. (\hyperref[subsec:4.3.2]{4.3.2})} 使用整体法基于三根导线总体合外力为0判断即可。
\paragraph{19. (\hyperref[sec:5.2]{5.2})}

\begin{equation*}
    \begin{cases}
        4\alpha=230-206\\
        2\alpha-\beta=90-82
    \end{cases}
\end{equation*}
