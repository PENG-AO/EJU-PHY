% brief answers of 18h30a

\section{18年第一回(H30A)}

\paragraph{1. (\hyperref[subsec:运动与力]{运动与力})}

\begin{equation*}
    T\cos\theta=f=\mu(mg-T\sin\theta)
\end{equation*}

\paragraph{2. (\hyperref[subsec:运动与力]{运动与力})}

\subparagraph{整体法判断加速度}

\begin{equation*}
    a=\frac{m_2g}{m_1+m_2}
\end{equation*}

\subparagraph{隔离法判断拉力}

\begin{equation*}
    a=\frac{T}{m_1}=\frac{G-T}{m_2}\implies
    T=\frac{m_1m_2}{m_1+m_2}g
\end{equation*}

\paragraph{3. (\hyperref[subsec:动量]{动量})}

\begin{gather*}
    v_B=\frac{mv_A+mv_A}{m+\frac{m}{2}}=\frac43v_A\\
    v_C=\frac{\frac{m}{2}v_B+\frac{m}{2}v_B}{\frac{m}{2}+\frac{m}{4}}=\frac43v_B=\frac{16}{9}v_A
\end{gather*}

\paragraph{4. (\hyperref[subsec:动量]{动量})}

\begin{equation*}
    |I|=|\Delta p|=|-mv\sin\theta-mv\sin\theta|=2mv\sin\theta
\end{equation*}

\paragraph{5. (\hyperref[subsec:能量]{能量})}

\subparagraph{step1 (m接触到M之前的瞬间)}

\begin{equation*}
    mgl=\frac12mv_m^2\implies
    v_m=\sqrt{2gl}
\end{equation*}

\subparagraph{step2 (m与M合体)}

\begin{gather*}
    mv_m=(m+M)v_{m+M}\\
    v_{m+M}=\frac{m}{m+M}v_m
\end{gather*}

\subparagraph{step3 (m与M一同上升)}

\begin{equation*}
    \frac12(m+M)v_{m+M}^2=(m+M)gh
\end{equation*}

\paragraph{6. (\hyperref[subsec:简谐振动]{简谐振动})}

\begin{gather*}
    kA=mg\implies A=\frac{mg}{k}\\
    T=2\pi\sqrt{\frac{m}{k}}
\end{gather*}

\paragraph{7. (\hyperref[sec:热与能量]{热与能量})}

\begin{gather*}
    c\times500\times(55-25)=4.2\times500\times(25-20)+300\times(25-20)\\
    c=0.8
\end{gather*}

\paragraph{8. (\hyperref[subsec:气体法则]{气体法则})}

\begin{gather*}
    \begin{cases}
        A: p_1(V_0-\Delta V)=nRT_0\\
        B: p_1(V_0+\Delta V)=nRT_1
    \end{cases}\\
    2p_1V_0=nR(T_0+T_1)\\
    \frac{T_1}{T_0}=\frac{2p_1}{p_0}-1
\end{gather*}

\paragraph{9. (\hyperref[subsec:热力学第一定律]{热力学第一定律})}

\begin{gather*}
    U=N\times\frac12m\bar{v}^2=\frac32nRT\\
    n\times\frac12M\bar{v}^2=\frac32nRT\\
    \sqrt{\bar{v}^2}=\sqrt{\frac{3RT}{M}}
\end{gather*}

\paragraph{10. (\hyperref[subsec:波的传播]{波的传播})} 略
\paragraph{11. (\hyperref[subsec:多普勒效应]{多普勒效应})}

\begin{equation*}
    \frac{V-v}{V}\times342=\frac{V+v}{V}\times338
\end{equation*}

\paragraph{12. (\hyperref[subsec:光的折射]{光的折射})}

\begin{equation*}
    1\cdot\sin45^\circ=n\cdot\sin30^\circ
\end{equation*}

\paragraph{13. (\hyperref[subsec:电场]{电场})}

\begin{equation*}
    \begin{cases}
        \vec{F_B}=\vec{F_C}=\frac{kQ^2}{4d^2}\\
        \vec{F_D}=\frac{kQq}{3d^2}\\
        \vec{F_B}+\vec{F_C}+\vec{F_D}=0
    \end{cases}
\end{equation*}

\paragraph{14. (\hyperref[subsec:电容器]{电容器})}

\begin{equation*}
    E=\frac{V}{3d-d}=\frac{V}{2d}\implies
    \begin{cases}
        V_1=E\cdot\frac{d}{2}=\frac{V}{4}\\
        V_2=E\cdot\frac{3d}{2}=\frac{3V}{4}
    \end{cases}
\end{equation*}

\paragraph{15. (\hyperref[subsec:直流电路]{直流电路})} 设右上和左下的闭合回路均为逆时针电流,记作$I_1$和$I_2$,并设重合部分为$I_1-I_2$。

\begin{gather*}
    \begin{cases}
        3-12(I_1-I_2)-6I_1=0\\
        6+12(I_1-I_2)=0
    \end{cases}\implies
    \begin{cases}
        I_1=1.5\\
        I_2=2
    \end{cases}\\
    P_{6V}=I_2\cdot6=12W
\end{gather*}

\paragraph{16. (\hyperref[subsec:磁场]{磁场})} 略
\paragraph{17. (\hyperref[subsec:安培力]{安培力})} 略
\paragraph{18. (\hyperref[subsec:电磁感应]{电磁感应})} 法拉第电磁感应定律计算大小,楞次定律判断方向即可。

\begin{equation*}
    |V|=\left|\frac{\Delta\Phi}{\Delta t}\right|
    =nS\cdot\left|\frac{\Delta B}{\Delta t}\right|
    =400\cdot3\times10^{-4}\cdot0.2=2.4\times10^{-2}V
\end{equation*}

\paragraph{19. (\hyperref[sec:波粒二象性]{波粒二象性})} 略
