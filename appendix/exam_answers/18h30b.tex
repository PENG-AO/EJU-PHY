% brief answers of 18h30b

\section{18年第二回(H30B)}

\paragraph{1. (\hyperref[subsec:运动与力]{运动与力})} 设$BB'$与水平方向夹角为$\theta$。

\subparagraph{step1 (受力平衡)}

\begin{equation*}
    \begin{cases}
        F_A+F_B\sin\theta=W\\
        F_C=F_B\cos\theta
    \end{cases}
\end{equation*}

\subparagraph{step2 (力矩平衡,以A为支点)}

\begin{equation*}
    2l\times W=4l\sin\theta\times F_B
\end{equation*}

\paragraph{2. (\hyperref[subsec:速度与加速度]{速度与加速度})} x-t图像中图形面积为路程。
\paragraph{3. (\hyperref[subsec:能量]{能量})}

\subparagraph{step1 (从A到B)}

\begin{gather*}
    mg(H-h)=\frac12mv^2\\
    v=\sqrt{2g(H-h)}
\end{gather*}

\subparagraph{step1 (从B到C)}

\begin{gather*}
    \frac12gt^2=h\implies t=\sqrt{\frac{2h}{g}}\\
    2h=vt=\sqrt{2g(H-h)}\cdot\sqrt{\frac{2h}{g}}
\end{gather*}

\paragraph{4. (\hyperref[subsec:动量]{动量})} 日语题。主语为A,被动对象为B,即A受到的来自于B的冲量,应选1。
\paragraph{5. (\hyperref[subsec:圆周运动]{圆周运动})}

\begin{equation*}
    2mg\cos\theta=F_\textrm{向}=m\omega^2l\cos\theta
\end{equation*}

\paragraph{6. (\hyperref[subsec:简谐振动]{简谐振动})} 本题运动学解法为优。

\begin{equation*}
    a=-\omega^2x
\end{equation*}

\paragraph{7. (\hyperref[sec:热与能量]{热与能量})} 设金属容器热容量为C,水的比热为c。

\begin{gather*}
    \begin{cases}
        (50-20)C=c\times500\times(60-50)\\
        (t-20)C=c\times100\times(60-t)
    \end{cases}\implies
    \frac{30}{t-20}=\frac{50}{60-t}
\end{gather*}

\paragraph{8. (\hyperref[subsec:气体法则]{气体法则})}

\subparagraph{初始状态}

\begin{equation*}
    p_0V=nRT_0
\end{equation*}

\subparagraph{终止状态}

\begin{equation*}
    \begin{cases}
        p_1\cdot\frac65V=nRT_B\\
        p_1\cdot\frac45V=nRT_B
    \end{cases}
\end{equation*}

\paragraph{9. (\hyperref[subsec:热力学第一定律]{热力学第一定律})}

\begin{align*}
    \Delta U=&Q_\textrm{吸}+W_\textrm{された}\\
    Q_\textrm{吸}=&\Delta U+W_\textrm{した}\\
    =&\frac32\left(\frac32p_0V_0-p_0V_0\right)+\frac74p_0V_0
\end{align*}

\paragraph{10. (\hyperref[subsec:波的传播]{波的传播})} 基于y-t图像和v-t图像寻找满足条件点的特征即可。
\paragraph{11. (\hyperref[subsec:多普勒效应]{多普勒效应})} 略
\paragraph{12. (\hyperref[subsec:光的干涉]{光的干涉})} 应注意双缝干涉光程差公式中d为缝间距,对应本题即为2d。
\paragraph{13. (\hyperref[subsec:电场]{电场}, \hyperref[subsec:电势]{电势})}

\subparagraph{$x=d$处}

\begin{gather*}
    \begin{cases}
        V=\frac{kQ_1}{d}+\frac{kQ_2}{2d-d}=\frac{k}{d}(Q_1+Q_2)=0\\
        E=\frac{kQ_1}{d^2}-\frac{kQ_2}{(d-2d)^2}=\frac{k}{d^2}(Q_1-Q_2)>0
    \end{cases}\\\implies
    Q_1>Q_2,Q_1=-Q_2
\end{gather*}

\subparagraph{$x=3d$处}

\begin{equation*}
    \begin{cases}
        V=\frac{kQ_1}{3d}+\frac{kQ_2}{3d-2d}=\frac{k}{d}(\frac{Q_1}{3}+Q_2)=-\frac{2kQ_1}{3d}<0\\
        E=\frac{kQ_1}{9d^2}+\frac{kQ_2}{(3d-2d)^2}=\frac{k}{d^2}(\frac{Q_1}{9}+Q_2)=-\frac{8kQ_1}{9d^2}<0
    \end{cases}
\end{equation*}

\paragraph{14. (\hyperref[subsec:电容器]{电容器})}

\begin{gather*}
    \begin{cases}
        Q_1=2\mu\times5=10\mu\\
        Q_2=3\mu\times5=15\mu
    \end{cases}\implies Q=-Q_1+Q_2=5\mu\\
    Q_1'=C_1\times\frac{Q}{C_1+C_2}=2\mu
\end{gather*}

\paragraph{15. (\hyperref[subsec:直流电路]{直流电路})} 设流过$E_1$和$E_2$的电流分别为$I_1$和$I_2$,且在B点汇聚为$I_1+I_2$流向$R_3$。

\begin{gather*}
    \begin{cases}
        E_1\to R_1\to R_3\to E_1:6-3kI_2-6k(I_1+I_2)=0\\
        E_2\to R_2\to R_3\to E_2:3-3kI_1-6k(I_1+I_2)=0
    \end{cases}\\\implies I_1=-0.2m,I_2=0.8m
\end{gather*}

\paragraph{16. (\hyperref[subsec:安培力]{安培力})} 略
\paragraph{17. (\hyperref[subsec:洛伦兹力]{洛伦兹力})}

\begin{gather*}
    a=\frac{F}{m}=\frac{IBl}{m}=\frac{\frac{E-Bvl}{R}Bl}{m}\\
    a=\frac{dv}{dt}=\frac{Bl}{mR}(E-Bl\cdot v)\\
    \int\frac{dv}{E-Bl\cdot v}=\int\frac{Bl}{mR}dt\\
    \frac{-1}{Bl}\cdot\ln(E-Bl\cdot v)=\frac{Bl}{mR}\cdot t\\
    v=\frac{E}{Bl}-\frac{1}{Bl}\exp\left(-\frac{B^2l^2}{mR}t\right)\\
    a=\frac{dv}{dt}=\frac{Bl}{mR}\exp\left(-\frac{B^2l^2}{mR}t\right)
\end{gather*}

\paragraph{18. (\hyperref[subsec:电磁感应]{电磁感应})} 略
\paragraph{19. (\hyperref[sec:波粒二象性]{波粒二象性})}

\begin{gather*}
    E_\textrm{光子}=h\nu\xlongequal{\lambda\nu=c}\frac{hc}{\lambda}\\
    E_\textrm{电子}=\frac{p^2}{2m}\xlongequal{\lambda=h/p}\frac{h^2}{2m\lambda^2}
\end{gather*}
