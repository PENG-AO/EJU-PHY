% brief answers of 15h27b

\section{15年第二回(H27B)}

\paragraph{1. (\hyperref[subsec:1.1.3]{1.1.3})}

\begin{equation*}
    2G\times L\cos\theta=G\times\frac{L}{2}\sin\theta
\end{equation*}

\paragraph{2. (\hyperref[subsec:1.1.1]{1.1.1})}

\begin{equation*}
    \begin{cases}
        v^2=2ax\\
        v=at
    \end{cases}\implies
    v=\frac{2x}{t}
\end{equation*}

\paragraph{3. (\hyperref[subsec:1.1.2]{1.1.2})} 旋转视角至BC水平,则该物体同时具有向下和向右的加速度,且大小均为$g'=\frac{g}{\sqrt{2}}$。

\begin{equation*}
    \begin{cases}
        t=\frac{2v}{g'}\\
        \sqrt{2}h=\frac12g't^2
    \end{cases}
\end{equation*}

\paragraph{4. (\hyperref[subsec:1.2.1]{1.2.1})} 一般能量解法优于运动解法。

\subparagraph{能量解法} 以出发点为重力势能基准。

\begin{equation*}
    \begin{cases}
        mg=kd\\
        \frac12k(3d)^2=\frac12mv^2+3mgd
    \end{cases}\implies
    v=\sqrt{3gd}
\end{equation*}

\subparagraph{运动解法}

\begin{gather*}
    x=A\cos\omega t=2d\cos\sqrt{\frac{k}{m}}t\\
    v=\frac{dx}{dt}=-2d\sqrt{\frac{k}{m}}\sin\sqrt{\frac{k}{m}}t\\
    x=-d\implies t=\sqrt{\frac{k}{m}}\times\frac23\pi\ or\ \sqrt{\frac{k}{m}}\times\frac43\pi\\
    |v|=\left|-2d\sqrt{\frac{k}{m}}\sin\left(\frac23\pi\ or\ \frac43\pi\right)\right|=\sqrt{\frac{3k}{m}}d\\
    mg=kd\implies|v|=\sqrt{3gd}
\end{gather*}

\paragraph{5. (\hyperref[subsec:1.2.2]{1.2.2})} 基于A与B无相对运动和动量守恒可解。
\paragraph{6. (\hyperref[subsec:1.3.1]{1.3.1})}

\begin{gather*}
    \begin{cases}
        F_\textrm{向}=mg\sin30^\circ\\
        \frac12mv_0^2=\frac12mv^2+mgr(1+\sin30^\circ)
    \end{cases}
\end{gather*}

\paragraph{7. (\hyperref[sec:2.1]{2.1})}

\begin{gather*}
    \begin{cases}
        c_A\times0.1\times15=3000\\
        c_B\times0.1\times10=4000
    \end{cases}\implies
    \begin{cases}
        c_A=2000\\
        c_B=4000
    \end{cases}\\
    2000\times0.4\times t+4000\times0.3\times t=10000
\end{gather*}

\paragraph{8. (\hyperref[subsec:2.2.2]{2.2.2})}

\begin{gather*}
    \begin{cases}
        pV_A=n_ART\\
        pV_B=n_BRT
    \end{cases}\implies
    \frac{V_A}{V_B}=\frac{n_A}{n_B}=\frac{N_A}{N_B}
\end{gather*}
\begin{equation*}
    \frac12m\overline{v^2}\times N=\frac32nRT\implies
    \overline{v^2}\propto\frac1m
\end{equation*}

\paragraph{9. (\hyperref[subsec:2.2.2]{2.2.2})}

\begin{gather*}
    \begin{array}{c|ccc}
        & \Delta U & Q_\textrm{吸} & W_\textrm{された} \\\hline
        A\to B & - & - & 0 \\
        B\to C & + & + & -
    \end{array}\\
    Q_\textrm{实际吸热}=n(C_p-C_v)T_0=nRT_0
\end{gather*}

\paragraph{10. (\hyperref[subsec:3.1.3]{3.1.3})} 需区分波面和波的传播方向。
\paragraph{11. (\hyperref[subsec:3.1.2]{3.1.2})}

\begin{align*}
    |r_1-r_2|=&(2m-1)\frac\lambda2\\
    1=&\lambda\left(m-\frac12\right)\\
    f=&v\left(m-\frac12\right),\quad f\ge500Hz\\
    f=&510Hz,\quad m=2
\end{align*}

\paragraph{12. (\hyperref[subsec:3.3.2]{3.3.2})}

\begin{gather*}
    a=\Delta x=\frac{\lambda l}{d}\implies
    \frac{a'}{a}=\frac{d}{d'}=\frac23
\end{gather*}

\paragraph{13. (\hyperref[subsec:4.1.2]{4.1.2})}

\begin{gather*}
    \begin{cases}
        V_D=\frac{kq}{a}-\frac{8kq}{2a}=-\frac{3kq}{a}\\
        V_C=\frac{kq}{2a}-\frac{8kq}{a}=-\frac{15kq}{2a}
    \end{cases}\implies
    V_D-V_C=\frac{9kq}{2a}
\end{gather*}

\paragraph{14. (\hyperref[subsec:4.1.3]{4.1.3})} 串联电荷相等。

\begin{equation*}
    E=\frac{V}{d}=\frac{Q}{Cd}\xLongrightarrow[d_A=d_B]{Q_A=Q_B}
    \frac{E_B}{E_A}=\frac{C_A}{C_B}=\frac{C}{\epsilon C}=\frac1\epsilon
\end{equation*}

\paragraph{15. (\hyperref[subsec:4.2.1]{4.2.1})} 略
\paragraph{16. (\hyperref[subsec:4.3.1]{4.3.1})} 略
\paragraph{17. (\hyperref[subsec:4.3.2]{4.3.2})} 左手定则判断并做力的合成即可。
\paragraph{18. (\hyperref[subsec:4.4.1]{4.4.1})} 楞次定律判断并将导体棒视作电源即可。
\paragraph{19. (\hyperref[sec:5.2]{5.2})}

\begin{gather*}
    E_4-E_2=\frac{hc}{\lambda}\\
    \lambda=\frac{hc}{E_4-E_2}
    =\frac{6.6\times10^{-34}\times3\times10^8}{\frac{-2.2\times10^{-18}}{16}-\frac{-2.2\times10^{-18}}{4}}
    =4.8\times10^{-7}
\end{gather*}
