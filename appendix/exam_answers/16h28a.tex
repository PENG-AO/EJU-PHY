% brief answers of 16h28a

\section{16年第一回(H28A)}

\paragraph{1. (\hyperref[subsec:运动与力]{运动与力})} 受力分析,观察二次导数值即可。

\begin{gather*}
    F-mg=ma\\
    F=mg+\frac{d^2}{dt^2}h
\end{gather*}

\paragraph{2. (\hyperref[subsec:运动与力]{运动与力})}

\begin{equation*}
    \frac{F}{m+M}=\frac{\mu mg}{M}
\end{equation*}

\paragraph{3. (\hyperref[subsec:能量]{能量}, \hyperref[subsec:动量]{动量})}

\begin{equation*}
    \frac12mv^2=mgH\implies
    H=\frac{v^2}{2g}
\end{equation*}

\begin{gather*}
    mv=(m+M)v'\\
    v'=\frac{m}{m+M}v\\
    \frac12mv^2=mgh+\frac12(m+M)\left(\frac{m}{m+M}v\right)^2\\
    h=\frac{M}{m+M}\frac{v^2}{2g}
\end{gather*}

\paragraph{4. (\hyperref[subsec:动量]{动量})} F-t图像中面积为冲量。
\paragraph{5. (\hyperref[subsec:运动与力]{运动与力}, \hyperref[subsec:简谐振动]{简谐振动})} 略
\paragraph{6. (\hyperref[subsec:圆周运动]{圆周运动})}

\begin{equation*}
    \begin{cases}
        F_\textrm{向}=mg\cos\theta\\
        \frac12mv^2=mgr(1-\cos\theta)+\frac12mv_0^2
    \end{cases}
\end{equation*}

\paragraph{7. (\hyperref[sec:热与能量]{热与能量})}

\begin{gather*}
    2\times10^3\times1\times(10\Delta t)+1\times10^3\times(10\Delta t)=P\times\Delta t\\
    P=3\times10^4\\
    3.3\times10^5\times1=P\times t\\
    t=11
\end{gather*}

\paragraph{8. (\hyperref[subsec:气体法则]{气体法则})}

\begin{gather*}
    \begin{cases}
        A: 8p\cdot V=n_ART\\
        B: p\cdot 4V=n_BRT
    \end{cases}\\
    n_A'=\frac{V}{V+4V}(n_A+n_B)
\end{gather*}

\paragraph{9. (\hyperref[subsec:气体状态变化]{气体状态变化})} 略
\paragraph{10. (\hyperref[subsec:衍射·反射·折射]{衍射·反射·折射})} 略
\paragraph{11. (\hyperref[subsec:多普勒效应]{多普勒效应})} 平面多普勒效应,从圆上优劣弧着手,分析最值变化即可。
\paragraph{12. (\hyperref[subsec:光的干涉]{光的干涉})}

\begin{gather*}
    2x\tan\theta=m\lambda\\
    \Delta x=\frac{\lambda}{2\tan\theta}=\frac{\lambda L}{2D}
\end{gather*}

\paragraph{13. (\hyperref[subsec:电场]{电场})} 略
\paragraph{14. (\hyperref[subsec:电势]{电势})} 略
\paragraph{15. (\hyperref[subsec:欧姆定律]{欧姆定律})} 根据电阻定义式可知$R_Y=2R_X$,后根据电功率公式$P=\frac{V^2}{R}$计算即可。
\paragraph{16. (\hyperref[subsec:电容器]{电容器})} 可按照电容器并联处理电荷分配。

\begin{gather*}
    V'=\frac{Q_A}{C_A+C_B}\\
    U_A=U_A'+U_B'+W_R\\
    W_R=\frac12C_AV^2-\frac12(C_A+C_B)V'^2
\end{gather*}

\paragraph{17. (\hyperref[subsec:磁场]{磁场})} 略
\paragraph{18. (\hyperref[subsec:洛伦兹力]{洛伦兹力})}

\begin{gather*}
    qV=\frac12mv^2\implies v=\sqrt{\frac{2qV}{m}}\\
    r=\frac{mv}{qB}=\frac1B\sqrt{\frac{2mV}{q}}\\
    \frac{m'}{m}=\left(\frac{r'}{r}\right)^2=16
\end{gather*}

\paragraph{19. (\hyperref[sec:波粒二象性]{波粒二象性})} 略
