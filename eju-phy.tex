\documentclass[oneside, b5paper]{book}
%% 总设置
\usepackage{xeCJK} % 中英日支持
\setCJKmainfont{LXGW Bright}
\setmainfont{LXGW Bright}
\usepackage{hyperref} % 超链接引用
\hypersetup{
    colorlinks=true,
    linkcolor=blue,
    pdftitle={EJU PHY},
    pdfauthor={PENG AO}
}
%% 图片与插图设置
\usepackage{graphicx} % 图片插入
\graphicspath{{./img}}
\usepackage{amsmath} % 一般公式
\usepackage{amssymb} % 特殊符号
\usepackage{tikz} % 2d绘图
\usepackage{tikz-3dplot} % 3d绘图
\usepackage{circuitikz} % 电路绘图
\usetikzlibrary{
    decorations.pathmorphing,
    decorations.markings,
    angles, quotes, calc
}
\tikzset{
    spring/.style={decoration={
        coil,
        segment length=1mm,
        amplitude=2mm,
        aspect=0.5,
        pre length=1.5mm,
        post length=1.5mm
    }, decorate}, % preset style for a spring
    midarrow/.style={postaction={decoration={
        markings,
        mark= at position 0.5 with {\arrow{latex}}
    }, decorate}}  % preset style for arrow inline
}
\newcommand{\drawangle}[4][$\theta$]{
    \coordinate (a) at #2;
    \coordinate (b) at #3;
    \coordinate (c) at #4;
    \draw pic [draw, "#1", angle eccentricity=1.5] {angle=a--b--c};
} % command for drawing angle
%% 文章展示设置
% 0-chapter:章节
% 1-section:小节
% 2-subsection:次小节
% 3-subsubsection:次次小节。用于实际知识点
% 4-paragraph:段落。用于次次小节内并列的小知识点,或者次小节内简短补充的知识点,一般出现在开头
% 5-subparagraph:次段落。用于段落内的并列内容
\usepackage{ascmac} % 跨页显示
\usepackage{fancybox} % 带框文本
\usepackage{multicol} % 文字分栏
\def\changemargin#1#2{\list{}{\rightmargin#2\leftmargin#1}\item[]}
\let\endchangemargin=\endlist % 临时更改页宽
\usepackage{tocloft} % 设置lof
\usepackage{indentfirst} % 段首自动缩进
\setcounter{tocdepth}{2} % 设置目录层级
\setcounter{secnumdepth}{3} % 设置小节编号层级
\renewcommand\thesubsubsection{\S} % 设置次次小节编号模式
%% 版本控制设置
\usepackage{gitver}
\def\version{3.13.\gitVer{}} % 版本号
\def\updatedate{\today} % 更新日期

\begin{document}
    % title + acknowledge + preface + TOC
    \frontmatter
    % title page

\begin{titlepage}
    \begin{center}
        \vspace{4cm}
        
        \rule{\textwidth}{1.2pt}
        
        \vspace{0.3cm}

        {\Huge \textbf{EJU PHY}}

        \vspace{0.3cm}

        {\LARGE A BRIEF SUMMARY}

        \vspace{0.3cm}

        {\Large \textit{version \version}}

        \rule{\textwidth}{1.2pt}

        \vspace{2cm}

        {\LARGE \textbf{PENG AO}}

        \vfill

        \includegraphics[width=0.4\textwidth]{avatar}

        {\Large the latest update date is \updatedate\\
        \copyright2019-22 PENG AO, All rights reserved.}
    \end{center}
\end{titlepage}

% acknowledge

\clearpage
\begin{center}
    本文档仅作个人使用,禁止任何未经许可的其他用途。
\end{center}

% preface

\clearpage
\chapter*{Preface}
\addcontentsline{toc}{chapter}{前言}

本文档脱胎于2019至2021年间个人授课时所整理的大纲,主要梳理了留学生统一考试\footnote{https://www.jasso.go.jp/ryugaku/eju/index.html}理科物理的知识点,方便教学使用。

如今2022年初,正值本人即将迈入大学\footnote{东京大学,the university of tokyo}四年级之时。出于系统练习书写\LaTeX 文档,为今后学术报告、论文撰写做准备的目的,将此前的大纲进行了重新编辑。在书写过程中尽可能采取了清晰的书写结构,力争完全使用tikz这个绘图语言包来完成文档中的插图。

文章内容参考了「わかりやすい高校物理の部屋」\footnote{https://wakariyasui.sakura.ne.jp/}等网站和河合出版的《物理教室(四訂版)》\footnote{ISBN13:978-4-7772-1375-7}等书籍,基于个人中学\footnote{东北育才外国语学校}时备考留学生统一考试时的所学所感加以补充。

最后,对在教学过程中帮助我逐步做出优化调整的学生们以及编辑过程中辅助我校对的朋友们表示感谢。

\vfill
版本历史
\begin{itemize}
    \item \textit{version 1}:2019年手写版
    \item \textit{version 2}:2020年markdown电子版
    \item \textit{version 3}:2022年本文档
\end{itemize}

\vfill
\begin{flushright}
    PENG AO\\
    2022-03-30 in Tokyo
\end{flushright}

% toc

\tableofcontents

    % main part
    \mainmatter
    % chapter 1

\chapter{力学}

% chapter 1 section 1

\section{运动与力}

\subsection{速度与加速度}

\subsubsection{直线运动}

\begin{figure}[ht!]
    \centering
    \begin{tikzpicture}
        \draw[->] (-0.5, 0) -- (5, 0) node[right] {$x$};
        \fill[fill=gray] (0, 0.1) circle (0.1);
        \node[below] at (0, 0) {$(x, t)$};
        \fill[fill=gray] (3, 0.1) circle (0.1);
        \node[below] at (3, 0) {$(x+\Delta x, t+\Delta t)$};
        \draw[->] (0.5, 0.5) -- node[above] {$v$} (2.5, 0.5);
    \end{tikzpicture}
    \caption{速度与时间}
\end{figure}

\paragraph{速度}对于在x轴上运动的物体,其在这段时间内的平均速度可如下给出:
\begin{equation*}
    \bar{v} = \frac{\Delta x}{\Delta t}
\end{equation*}
其中$\Delta x$的部分叫做変位,其方向为初始位置指向终止位置。当这个时间间隔无限趋于0时,将这个极限值称为(瞬时)速度,即:
\begin{equation*}
    v = \lim_{\Delta t\to0}\frac{\Delta x}{\Delta t}=\frac{dx}{dt}
\end{equation*}
速度的方向由变位的方向决定,单位常用$\left(m/s\right)$。其大小可以用:速度の大きさ、速さ等词汇描述。

\paragraph{加速度}与速度类似,当我们着眼于某个时间间隔内速度的变化时,便可以得到平均加速度的定义。若再将其时间间隔无限缩小至0就有了(瞬时)加速度:
\begin{equation*}
    a = \lim_{\Delta t\to0}\frac{\Delta v}{\Delta t}=\frac{dv}{dt}
\end{equation*}
同样的,由于速度是矢量所以加速度也是具有方向的,其方向取决于速度变化的方向,单位常用$\left(m/s^2\right)$。

\paragraph{图像表示}在$v-t$图像中,根据速度定义可知:
\begin{equation*}
    \textrm{变位的(微小)变化}=\textrm{速度}\times\textrm{时间的(微小)变化}
\end{equation*}
所以,数形结合可得$v-t$图像与时间轴围成的面积即为该时间间隔内物体的距离。但有些时候围成的图形会跨过时间轴,此时根据速度值的正负(方向)可分析得到:横轴以上的面积表示正方向上的距离,横轴以下的部分表示负方向上的距离。
\begin{figure}[ht!]
    \centering
    \begin{tikzpicture}
        \draw[->] (-0.2, 0) -- (3, 0) node[right] {$t$};
        \draw[->] (0, -1) -- (0, 1) node[above] {$v$};
        \filldraw[color=black, fill=gray, fill opacity=0.3] (0, 0) -- 
            (0, 0.5) .. controls (1, 2) and (1.5, -1.5) .. (2.5, -0.8) -- (2.5, 0);
    \end{tikzpicture}
    \caption{$v-t$图像}
\end{figure}
此外,根据加速度的定义,我们可以根据$v-t$图像的斜率来确定其值。
\begin{itemize}
    \item 图形面积$\implies$移动距离
    \item 切线斜率$\implies$加速度
\end{itemize}

\paragraph{等加速度直线运动}加速度一定的直线运动,属于最基本的运动类型。
\begin{figure}[ht!]
    \centering
    \begin{tikzpicture}
        \draw[->] (-0.2, 0) -- (3, 0) node[right] {$t$};
        \draw[->] (0, -0.2) -- (0, 3) node[above] {$v$};
        \draw[domain=0:2.5] plot (\x, {0.5*\x + 1});
        \coordinate[label=left:$v$] (v) at (0, 2);
        \coordinate[label=left:$v_0$] (v0) at (0, 1);
        \coordinate[label=below:$t$] (t) at (2, 0);
        \draw[dashed] (v) -- (2, 2);
        \draw[dashed] (t) -- (2, 2);
        \fill[fill=gray, opacity=0.3] (0, 0) -- (v0) -- (2, 2) -- (t) --cycle;
    \end{tikzpicture}
    \caption{等加速度直线运动图像}
\end{figure}

\begin{itembox}[l]{运动学基本公式}
    \begin{gather*}
        v=v_0+at\\
        x=v_0t+\frac{1}{2}at^2\\
        v^2-{v_0}^2=2ax
    \end{gather*}
\end{itembox}

\subsubsection{平面运动分析}

根据速度/加速度的矢量性,运动可以轻松地被扩展到平面上。而且倘若借助向量、坐标等数学手段,我们就可以驾驭更加复杂的运动形式。
\begin{figure}[ht!]
    \centering
    \begin{tikzpicture}
        \begin{scope}
            \draw[-latex] (0, 0) -- node[below] {$\vec{a}$} (2, 0.5);
            \draw[-latex] (2, 0.5) -- node[right] {$\vec{b}$} (1, 1.5);
            \draw[-latex] (0, 0) -- node[left] {$\vec{c}=\vec{a}+\vec{b}$} (1, 1.5);
        \end{scope}
        \begin{scope}[xshift=100pt]
            \draw[-latex] (0, 0) -- node[below] {$\vec{a}$} (2, 0.5);
            \draw[-latex] (0, 0) -- node[left] {$\vec{b}$} (1, 1.5);
            \draw[-latex] (0, 0) -- (3, 2) node[right] {$\vec{c}=\vec{a}+\vec{b}$};
            \draw[dashed] (2, 0.5) -- (3, 2);
            \draw[dashed] (1, 1.5) -- (3, 2);
        \end{scope}
    \end{tikzpicture}
    \caption{向量合成与分解}
\end{figure}
如果将上述的合成与分解放到平面直角坐标系内操作,就是最常见也最常用的形式。
\begin{figure}[ht!]
    \centering
    \begin{tikzpicture}
        \draw[->] (-0.2, 0) -- (2, 0) node[right] {$x$};
        \draw[->] (0, -0.2) -- (0, 2) node[above] {$y$};
        \draw[thick, -latex] (0, 0) -- (1.5, 1.5) node[right] {$\vec{v}$};
        \draw[thick, -latex] (0, 0) -- (1.5, 0) node[below] {$\vec{v_x}=\vec{v}\cos\theta$};
        \draw[thick, -latex] (0, 0) -- (0, 1.5) node[left] {$\vec{v_y}=\vec{v}\sin\theta$};
        \draw[dashed] (1.5, 0) -- (1.5, 1.5);
        \draw[dashed] (0, 1.5) -- (1.5, 1.5);
        \drawangle{(1, 0)}{(0, 0)}{(1, 1)};
    \end{tikzpicture}
    \caption{速度的正交分解}
\end{figure}
\begin{itembox}[l]{运动分析}
    \begin{itemize}
        \item 合成分解:$\vec{v}\rightleftharpoons\vec{v_1}+\vec{v_2}$
        \item 相对速度:$\textrm{相对}=\textrm{对象}-\textrm{参考/基准}$
    \end{itemize}
\end{itembox}

\subsection{运动与力}

\subsubsection{力}

力的三要素(力的大小、作用线、作用点)、单位(N)等内容比较基础,在此略过。

\paragraph{力的平衡}由于力也是矢量,所以我们也可以对其进行合成/分解的操作。一般称合成得来的力为合力,称分解得来的力为分力。在考虑受力平衡时便借助此思想,将某物体视为质点\footnote{忽略极小或质量分布均匀的物体的大小,将其视作一个点。},其合力为0的状态定义为平衡状态。个人常用$\sum\vec F=0$的方式来简记。

\paragraph{重力}吸引地表所有物体,竖直向下的力。其成因是万有引力。数学形式为:$G=m\cdot g$。

\paragraph{张力/拉力}一般指绳子上的张力或拉力。对于轻质绳子,\underline{同一条绳子}上的拉力处处相等

\paragraph{弹力}日文为弾性力,指的是弹簧为了恢复到\underline{自然长/原长}而产生的力。弹力遵循胡克定律(フックの法則),同时应注意形变量具有方向。
\begin{itembox}[l]{胡克定律}
    \begin{equation*}
        \vec{F}=-k\vec{\Delta x}\quad(k:\textrm{バネ定数},\Delta x:\textrm{基于原长的形变量})
    \end{equation*}
\end{itembox}
\begin{figure}[ht!]
    \centering
    \begin{tikzpicture}
        \fill[fill=gray, opacity=0.3] (0, 0.2) rectangle (6, 0);
        \draw (0, 0) -- (6, 0); 
        \draw[spring] (1, 0) -- ++ (0, -1.6);
        \fill[fill=black] (1, -1.6) circle (3pt);
        \draw[spring] (3, 0) -- ++ (0, -1.6);
        \draw[spring] (3, -1.6) -- ++ (0, -1.6);
        \fill[fill=black] (3, -3.2) circle (3pt);
        \draw[spring] (4.8, 0) -- ++ (0, -1.3);
        \draw[spring] (5.3, 0) -- ++ (0, -1.3);
        \draw (4.8, -1.3) -- (5.3, -1.3);
        \draw (5.05, -1.3) -- (5.05, -1.5);
        \fill[fill=black] (5.05, -1.5) circle (3pt);
    \end{tikzpicture}
    \caption{弹簧串并联}
\end{figure}
\begin{itembox}[l]{弹簧串并联}
    \begin{itemize}
        \item 串联(受力一致)
        \begin{equation*}
            \begin{cases}
                F=k_1x_1=k_2x_2\\
                F=K(x_1+x_2)
            \end{cases}\implies\frac1K=\frac1{k_1}+\frac1{k_2}
        \end{equation*}
        \item 并联(形变一致)
        \begin{equation*}
            \begin{cases}
                F=F_1+F_2=k_1x+k_2x\\F=Kx
            \end{cases}\implies K=k_1+k_2
        \end{equation*}
    \end{itemize}
\end{itembox}

\paragraph{摩擦力}摩擦力来自于物体和接触面之间的支持力,日文为垂直抗力。应明确接触为要因,有接触则会产生支持力,进而才涉及摩擦力问题。其数学形式为:$f=N\cdot\mu$,其中$\mu$为摩擦系数。
\begin{figure}[ht!]
    \centering
    \begin{tikzpicture}
        \draw (0, 0) -- (3, 0);
        \fill[fill=gray, opacity=0.3] (0, 0) rectangle (3, -0.3);
        \filldraw[color=black, fill=gray, fill opacity=0.6, rounded corners=5pt] (1, 0) rectangle (2, 0.75);
        \draw[-latex, thick] (1.5, 0) -- (1.5, 1.1) node[above] {$N$};
        \draw[-latex, thick] (1.5, 0) -- (0.5, 0) node[above] {$f$};
    \end{tikzpicture}
    \caption{摩擦力}
\end{figure}
如果给一个物体施加一个线性变大的力,作用在其身上的摩擦力变化如图。
\begin{figure}[ht!]
    \centering
    \begin{tikzpicture}
        \draw[->] (-0.2, 0) -- (3, 0) node[right] {$t$};
        \draw[->] (0, -0.2) -- (0, 3) node[above] {$f$};
        \draw[thick] (0, 0) -- (1, 2) node[above, fill=white]{$f_0=N\cdot\mu\quad(\mu:\textrm{静止摩擦系数})$};
        \draw[thick] (1, 2) -- (1, 1.5);
        \draw[thick] (1, 1.5) -- (3, 1.5) node[right] {$f=N\cdot\mu^\prime\quad(\mu^\prime:\textrm{滑动摩擦系数})$};
    \end{tikzpicture}
    \caption{摩擦力变化}
\end{figure}
由此可知,静止摩擦力是物体在发生运动之前用来平衡外力的力,其最大值是最大静摩擦力。

\subsubsection{运动法则}

\begin{itembox}[l]{牛顿运动定律}
    \begin{itemize}
        \item 惯性法则:物体\underline{不受力}或\underline{合外力为0}时,运动状态不发生改变(惯性)。
        \item 运动法则:物体受力后会产生与力\underline{同方向}的加速度。加速度大小与力的大小成正比,与物体质量成反比。
        \begin{equation*}
            \vec{a}=k\cdot\frac{\vec{F}}{m}\to
            \vec{F}=m\vec{a}
        \end{equation*}
        \item 作用力与反作用力法则:物体A向物体B施力后,会受到来自物体B的\underline{大小相等}、\underline{方向相反}(等大反向)的力。
    \end{itemize}
\end{itembox}
其中惯性法则揭示的本质问题是:力是改变物体运动状态的原因。同时应注意,作用力与反作用力的分析对象为两个物体,而受力平衡则只针对单个物体论。

\subsubsection{重力相关的运动}

\paragraph{落体运动}根据物体的初速度有以下三种情况。
\begin{itemize}
    \item 自由落体:初速度为0,只受重力。
    \item 上抛运动:初速度向上为$v_0$,只受重力。
    \item 下抛运动:初速度向下为$v_0$,只受重力。
\end{itemize}
将上述条件代入运动学基本公式就可得该特殊情形下的运动方程。
\begin{itembox}[l]{自由落体}
    \begin{equation*}
        \begin{cases}
            v=gt\\
            h=\frac{1}{2}gt^2\\
            v^2=2gh
        \end{cases}
    \end{equation*}
\end{itembox}

\paragraph{抛体运动}将落体运动扩展到平面内得到的便是抛体运动。
\begin{figure}[ht!]
    \centering
    \begin{tikzpicture}
        \draw[->] (-0.2, 0) -- (3, 0) node[right] {$x$};
        \draw[->] (0, 0.2) -- (0, -3) node[below] {$y$};
        \draw[color=gray, domain=0:2.5] plot (\x, {-0.5*\x^2});
        \draw[thick, -latex] (1.4, -0.98) -- (2.4, -2.38) node[right] {$\vec{v}$};
        \draw[-latex] (1.4, -0.98) -- node[above] {$\vec{v_x}$} (2.4, -0.98);
        \draw[-latex] (1.4, -0.98) -- node[left] {$\vec{v_y}$} (1.4, -2.38);
    \end{tikzpicture}
    \caption{平抛运动}
\end{figure}
\subparagraph{平抛运动}
\begin{itemize}
    \item 条件:水平初速度为$v_0$,只受重力。
    \item 分析:
    \begin{equation*}
        \begin{cases}
            \textrm{水平}:a_x=0\implies\textrm{匀速直线运动}\\
            \textrm{竖直}:a_y=g\implies\textrm{自由落体}
        \end{cases}
    \end{equation*}
\end{itemize}

\begin{figure}[ht!]
    \centering
    \begin{tikzpicture}
        \draw[->] (-0.2, 0) -- (5, 0) node[right] {$x$};
        \draw[->] (0, -0.2) -- (0, 2.5) node[above] {$y$};
        \draw[color=gray, domain=0:4] plot (\x, {-0.5*(\x-2)^2+2});
        \draw[thick, -latex] (0, 0) -- (0.5, 1) node[above] {$v_0$};
        \drawangle{(1, 0)}{(0, 0)}{(1, 2)};
        \draw[dashed] (2, 2) -- (2, 0) node[below] {$x_{mid}$};
        \draw[dashed] (2, 2) -- (0, 2) node[left] {$y_{max}$};
        \node[below] at (4, 0) {$x_{max}$};
    \end{tikzpicture}
    \caption{斜抛运动}
\end{figure}
\subparagraph{斜抛运动}
\begin{itemize}
    \item 条件:初速度为$v_0$,倾角$\theta$,只受重力。
    \item 分析:
    \begin{equation*}
        \begin{cases}
            \textrm{水平:初速度}v_0\cos\theta,\textrm{不受力}\implies\textrm{匀速直线运动}\\
            \textrm{竖直:初速度}v_0\sin\theta,\textrm{受重力}\implies\textrm{上抛运动}
        \end{cases}
    \end{equation*}
    \item 结论1:纵向最大高度
    \begin{gather*}
        \begin{cases}
            v=v_0\sin\theta-gt\\
            y=v_0\sin\theta t-\frac{1}{2}gt^2
        \end{cases}\\
        v=0,\textrm{即}t=\frac{v_0\sin{\theta}}{g}\textrm{时}
        y=\frac{{v_0}^2\sin^2{\theta}}{2g}\textrm{最高}
    \end{gather*}
    \item 结论2:滞空时间
    \begin{equation*}
        y=0\implies t=0,\frac{2v_0\sin{\theta}}{g}\textrm{(对称)}
    \end{equation*}
    \item 结论3:最远距离
    \begin{gather*}
        x=v_0\cos{\theta}\frac{2v_0\sin\theta}{g}=\frac{{v_0}^2\sin{2\theta}}{g}\\
        \therefore\theta=45^\circ\textrm{时},x=\frac{{v_0}^2}{g}\textrm{最大}
        \end{gather*}
\end{itemize}

\subsection{刚体与力}

\subsubsection{刚体}

与质点类似,刚体也是为分析方便而引入的一种理想模型。对于一般物体来说当受力时不可避免地会产生些许形变,然而处理大多数问题的过程中,这些形变无关紧要。那么,不考虑受力形变的刚体便应运而生。

\subsubsection{力矩}

日文为力のモーメント,是一种描述物体绕轴旋转情况的物理量。标准定义为力臂(腕の長さ)和力的向量积\footnote{力矩:$\vec{M}=\vec{r}\times\vec{F}$}。因此力矩是一个矢量,由旋转方向和旋转的强度构成,但解题时常常将这两个信息分开考虑。
\begin{figure}[ht!]
    \centering
    \begin{tikzpicture}[scale=0.8]
        \draw[ultra thick] (0, 0) -- node[above] {$r$} (3, 0);
        \draw[-latex] (3, 0) -- node[above] {$F$} (4, 1);
        \draw[dashed] (3, 0) -- (4, 0);
        \drawangle{(4, 0)}{(3, 0)}{(4, 1)};
        \draw[dashed] (3, 0) -- (1, -2);
        \draw[<->] (0, 0) -- node[fill=white] {$r\sin\theta$} (1.5, -1.5);
    \end{tikzpicture}
    \caption{力矩}
\end{figure}
\begin{itembox}[l]{力矩}
    \begin{equation*}
        M=F\cdot r\cdot\sin\theta
    \end{equation*}
\end{itembox}

\subsubsection{非共点力刚体平衡}

在考虑刚体平衡时,除了需要满足牛顿第一定律的条件以外,还需要保证物体自身不能旋转,即力矩平衡。
\begin{itembox}[l]{刚体平衡条件}
    \begin{itemize}
        \item 受力平衡:$\sum\vec{F}=0$
        \item 力矩平衡:$\sum\vec{M}=0$
    \end{itemize}
\end{itembox}
因此,在刚体上做力的合成时需要确保其总力矩不发生改变。

% chapter 1 section 2

\section{能量与动量}

\subsection{能量}

\subsubsection{功与功率}

倘若物体在力F的作用下运动了s距离,则说该力对物体做了功,日文为仕事,其单位是$J=N\cdot m$。即功是\underline{力在运动方向上随距离的累积}。其严格定义是力与位移的标量积。
\begin{figure}[ht!]
    \centering
    \begin{tikzpicture}
        \draw[->] (-0.2, 0) -- (3, 0) node[right] {$s$};
        \draw[->] (0, -0.2) -- (0, 2) node[above] {$F$};
        \filldraw[color=black, fill=gray, fill opacity=0.3] (0, 0) -- 
            (0, 1) .. controls (1, 0.5) and (1.5, 2) .. (2.5, 1.5) -- (2.5, 0);
        \node at (1.25, 0.5) {$W$};
    \end{tikzpicture}
    \caption{$F-s$图像与功}
\end{figure}
如果物体能对外做功,便说该物体具有能量。
\begin{itembox}[l]{功}
    \begin{equation*}
        W=\vec{F}\cdot\vec{s}=Fs\cos\theta
    \end{equation*}
    \begin{itemize}
        \item 可正可负,取决于位移的方向
        \item 力与位移方向垂直时不做功,$W=0$
    \end{itemize}
\end{itembox}
此外,类比于距离与速度的关系,单位时间内所做的功称为功率,日文为仕事率,其单位是$W=J/s$。处理问题时除了定义式,也常用$P=\frac{W}{t}=\frac{\vec{F}\cdot\vec{s}}{t}=\vec{F}\cdot\vec{v}$。

\subsubsection{动能}

日文为運動エネルギー,专指\underline{运动中}的物体具有的能量。
\begin{itembox}[l]{动能}
    \begin{equation*}
        E_k=\frac12mv^2
    \end{equation*}
\end{itembox}
如图所示,在光滑水平面上,质量为m的物体受恒力F作用运动了s距离。
\begin{figure}[ht!]
    \centering
    \begin{tikzpicture}
        \draw (0, 0) -- (6, 0);
        \fill[fill=gray, opacity=0.3] (0, 0) rectangle (6, -0.5);
        \filldraw[color=black, fill=gray, fill opacity=0.5, rounded corners=5pt] (0.5, 0) rectangle node {$v_0$} (2, 1);
        \draw[thick, -latex] (2, 0.5) -- node[above] {$F$} (2.6, 0.5);
        \filldraw[color=black, fill=gray, fill opacity=0.8, rounded corners=5pt] (3.7, 0) rectangle node {$v$} (5.2, 1);
        \draw[thick, -latex] (5.2, 0.5) -- node[above] {$F$} (5.8, 0.5);
        \draw[<->] (2, -0.3) -- node[fill=gray!30] {$s$} (5.2, -0.3);
    \end{tikzpicture}
    \caption{动能定理}
\end{figure}
在此期间速度从$v_0$变化为了$v$。由牛二定律和运动学基本公式可得
\begin{equation*}
    W_{\textrm{外}}=Fs=mas=\frac12m(v^2-{v_0}^2)=\Delta E_k
\end{equation*}
即在这个过程中外力做功转化为了物体动能的增量。

\subsubsection{势能}

potential energy,日文为位置エネルギー。从英日双语着眼则可对“势”一字简做解析:势能这种能量是潜在的,并且是由物体\underline{位置状态}的改变激发。其具体定义为:物体从当前位置回到基准位置时保存力\footnote{做功与路径无关的力:重力、弹簧弹力、万有引力、电场力等}所做的功。

\begin{itembox}[l]{势能}
    \begin{itemize}
        \item 重力势能(重力による位置エネルギー):
        \begin{equation*}
            E_p=mgh
        \end{equation*}
        \item 弹力势能(弾性力による位置エネルギー):
        \begin{equation*}
            E_p=\frac12kx^2
        \end{equation*}    
    \end{itemize}
\end{itembox}

\subsubsection{机械能}

日文为力学的エネルギー,是动能与势能的统称,
\begin{itembox}[l]{机械能}
    \centering
    力学的エネルギー=運動エネルギー+位置エネルギー
\end{itembox}
在只有保存力做功或者非保存力不做功的情况下守恒。更一般的,因为系统内能量的总和始终一定,只是做了形式上的转化,所以对于非保存力参与的一般情况,可以采取如下两种思路处理问题。

\begin{itemize}
    \item 仅保存力参与:初始时总能量=终止时总能量
    \item 非保存力参与:终始总能量差额=消耗的能量
\end{itemize}

\subsection{动量}

\subsubsection{动量与冲量}

\begin{figure}[ht!]
    \centering
    \begin{tikzpicture}
        \draw[dashed, domain=0:2] plot (\x, {0.5*(\x-1)^2-0.5});
        \draw[thick, -latex] (0, 0) -- node[left] {$v$} (1, -1);
        \draw[dashed, fill=white] (0, 0) circle (3pt);
        \draw[thick, -latex] (2, 0) -- node[right] {$v^\prime$} (3.5, 1.5);
        \draw[dashed, fill=white] (2, 0) circle (3pt);
        \draw[thick, -latex] (1, -0.5) -- (1.2, 0.5) node[above] {$\vec{F}$};
        \draw[fill=white] (1, -0.5) circle (3pt);
    \end{tikzpicture}
    \caption{动量与冲量}
\end{figure}
如图,对于质量为m的物体,在极短的时间t内施加一个力F,使其速度从$v$变为了$v^\prime$,观察其运动方程可知
\begin{gather*}
    \vec{F}=m\vec{a}=m\frac{\vec{v^\prime}-\vec{v}}{\Delta t}\\
    \vec{F}\Delta t=m\vec{v^\prime}-m\vec{v}
\end{gather*}
称其中$\vec{F}\Delta t$的部分为冲量,日文为力積,$m\vec{v}$的部分为动量,日文为運動量。由此可见,动量变化等于冲量\footnote{类比于功,可以说冲量是力在运动方向上随时间的累积}。

\subsubsection{动量守恒}

\begin{figure}[ht!]
    \centering
    \begin{tikzpicture}
        \begin{scope}[xshift=-120]
            \draw[thick, -latex] (0, 0) -- node[above] {$m_1v_1$} ++ (1, 0);
            \fill[color=gray] (0, 0) circle (5pt);
            \draw[thick, -latex] (1.5, 0) -- node[above] {$m_2v_2$} ++ (0.8, 0);
            \fill[color=gray] (1.5, 0) circle (3pt);
        \end{scope}
        \begin{scope}
            \draw[thick, -latex] (0, 0) -- node[above] {$-F\Delta t$} ++ (-1, 0);
            \fill[color=gray] (0, 0) circle (5pt);
            \draw[thick, -latex] (0.2, 0) -- node[above] {$F\Delta t$} ++ (1, 0);
            \fill[color=gray] (0.2, 0) circle (3pt);
        \end{scope}
        \begin{scope}[xshift=80]
            \draw[thick, -latex] (0, 0) -- node[above] {$m_1{v_1}^\prime$} ++ (0.5, 0);
            \fill[color=gray] (0, 0) circle (5pt);
            \draw[thick, -latex] (1.5, 0) -- node[above] {$m_2{v_2}^\prime$} ++ (1.2, 0);
            \fill[color=gray] (1.5, 0) circle (3pt);
        \end{scope}
    \end{tikzpicture}
    \caption{动量守恒}
\end{figure}
对于上图的情形,根据牛三定律可得如下联立式:
\begin{equation*}
    \begin{cases}
        \vec{F}\cdot\Delta t&=m_2\vec{{v_2}^\prime}-m_2\vec{v_2}\\
        -\vec{F}\cdot\Delta t&=m_1\vec{{v_1}^\prime}-m_1\vec{v_1}
    \end{cases}
\end{equation*}
两侧相加整理后可得如下结论。
\begin{itembox}[l]{动量守恒}
    \begin{equation*}
        m_1\vec{v_1}+m_2\vec{v_2}=m_1\vec{{v_1}^\prime}+m_2\vec{{v_2}^\prime}
    \end{equation*}
\end{itembox}
其中左侧为碰撞前两物体的总动量,右侧为碰撞后两物体的总动量,因此在上述情形下动量守恒。

一般的,在由任意个物体组成的系统内,只要不受外力就都会满足动量守恒。而且根据动量的矢量性,其守恒也不局限于直线上,也可扩展到空间内。此时,即便有某一方向不满足条件,我们仍然可以列满足那个方向上的动量守恒。常见模型除了碰撞以外还有单个物体的分裂等。

\subsubsection{反弹系数}

日文为反発係数或者はねかえり係数,是一个描述两物体碰撞效果的数值。
\begin{itembox}[l]{反弹系数}
    \begin{equation*}
        e=-\frac{{v_1}^\prime-{v_2}^\prime}{v_1-v_2}\quad(0\le e\le1)
    \end{equation*}
    \begin{itemize}
        \item $e=0$:完全非弹性碰撞
        \item $0<e<1$:非弹性碰撞
        \item $e=1$:弹性碰撞(此时动能守恒)
    \end{itemize}
\end{itembox}
结合动量守恒公式可得两物体碰撞后速度的一般公式。
\begin{itembox}[l]{一般碰撞公式}
    \begin{equation*}
        \begin{cases}
            {v_1}^\prime=\frac{1}{m_1+m_2}(m_1v_1+m_2v_2-em_2(v_1-v_2))\\
            {v_2}^\prime=\frac{1}{m_1+m_2}(m_1v_1+m_2v_2+em_1(v_1-v_2))
        \end{cases}
    \end{equation*}
\end{itembox}

\subsubsection{特殊实例}

\paragraph{一动碰一静}即$v_2=0$的一般碰撞。此时一般碰撞公式会有所简化,而且两物体碰撞问题以此题型居多。
\begin{itembox}[l]{简化碰撞公式}
    \begin{equation*}
        \begin{cases}
            v_1=\frac{1}{m_1+m_2}(m_1-em_2)v\\
            v_2=\frac{1}{m_1+m_2}(m_1+em_1)v
		\end{cases}
    \end{equation*}
\end{itembox}

\paragraph{固定面碰撞}物体与墙面、地面等固定面碰撞的问题,其速度变化、高度变化十分具有代表性。
\begin{figure}[ht!]
    \centering
    \begin{tikzpicture}
        \draw (0, 0) -- (2, 0);
        \fill[fill=gray, opacity=0.3] (0, 0) rectangle (2, -0.3);
        \draw[-latex, rounded corners=3pt] (1.1, 1.3) -- node[right] {$h$} (1.1, 0) -- (0.9, 0) -- node[left] {$h^\prime$} (0.9, 0.8);
        \draw[fill=white] (1.1, 1.3) circle (3pt);
    \end{tikzpicture}
    \caption{固定面碰撞}
\end{figure}
\begin{itembox}[l]{固定面碰撞结论}
    \begin{equation*}
        \begin{cases}
            v^\prime=ev\\
            h^\prime=e^2h
        \end{cases}
    \end{equation*}
\end{itembox}

\paragraph{斜向碰撞} 倘若物体从斜方向而来与平面发生碰撞,则需要利用运动的矢量性,将其分解处理。
\begin{figure}[ht!]
    \centering
    \begin{tikzpicture}[scale=0.8]
        \draw (0, 0) -- (4, 0);
        \fill[fill=gray, opacity=0.3] (0, 0) rectangle (4, -0.3);
        \fill (0.2, 1.3) circle (3pt);
        \draw[-latex] (0.2, 1.3) -- (3, -1.5) node[right] {$V$};
        \draw[dashed, -latex] (1.5, 0) -- node[left] {$v$} (1.5, -1.5);
        \draw[dashed, -latex] (1.5, -0.1) -- (3, -0.1) node[right, below] {$u$};
        \draw[dashed, -latex] (1.5, 0) -- (1.5, 1) node[above] {$v^\prime=ev$};
        \draw[dashed, -latex] (1.5, 0.1) -- (3, 0.1) node[right, above] {$u^\prime=u$};
        \draw[thick, -latex] (1.5, 0) -- (3, 1) node[right] {$V^\prime$};
    \end{tikzpicture}
    \caption{斜向碰撞}
\end{figure}

% chapter 1 section 3

\section{特殊运动}

\paragraph{惯性力}为非惯性系\footnote{系统参照物做非等速直线运动的系}下维持平衡的力,在部分情形下会使问题分析变得简单。
\begin{figure}[ht!]
    \centering
    \begin{tikzpicture}[scale=0.75]
        \draw (-1, 0) -- (3, 0);
        \fill[fill=gray, opacity=0.3] (-1, 0) rectangle (3, 0.3);
        \draw (2, 0) -- (1, {-sqrt(3)});
        \fill (1, {-sqrt(3)}) circle (3pt);
        \draw[thick, -latex] (1, {-sqrt(3)}) -- ++ (-0.5, 0) node[left] {$-m\vec{a}$};
        \draw[thick, -latex] (1, {-sqrt(3)}) -- ++ (0.5, {0.5*sqrt(3)}) node[right] {$\vec{T}$};
        \draw[thick, -latex] (1, {-sqrt(3)}) -- ++ (0, {-0.5*sqrt(3)}) node[right] {$m\vec{g}$};
    \end{tikzpicture}
    \caption{非惯性系}
\end{figure}

\subsection{圆周运动}
\label{subsec:圆周运动}

\subsubsection{等速圆周运动}

\begin{figure}[ht!]
    \centering
    \begin{minipage}[t]{0.48\textwidth}
        \centering
        \begin{tikzpicture}[scale=0.65]
            \draw[dashed] (0, -1.5) -- (0, 1.5);
            \draw[thick, -latex] (0, 0) -- (0, 1.2) node[right] {$\vec{\omega}$};
            \draw[dashed] (0,0) circle (1.5 and 0.5);
            \draw[thick, -latex] (0, 0) -- node[above] {$\vec{r}$} (-45:1.5 and 0.5);
            \draw[thick, -latex] (-45:1.5 and 0.5) -- ++ (45:1.5 and 0.5) node[right] {$\vec{v}$};
        \end{tikzpicture}
        \caption{角速度}
    \end{minipage}
    \begin{minipage}[t]{0.48\textwidth}
        \centering
        \begin{tikzpicture}[scale=0.65]
            \draw (0, 0) circle (1.5) node[below] {$O$};
            \draw (1.5, 0) -- (0, 0) -- node[above] {$r$} (45:1.5) node[right] {$(r\cos\omega t, r\sin\omega t)$};
            \drawangle[$\omega t$]{(1, 0)}{(0, 0)}{(45:1)};
            \draw[thick, -latex] (45:1.5) -- node[right] {$v$} ++ ($0.6*(135:1.5)$);
        \end{tikzpicture}
        \caption{圆周运动加速度}
    \end{minipage}
\end{figure}
\begin{itembox}[l]{角速度与周期}
    \begin{equation*}
        \omega = \frac{d\theta}{dt}
    \end{equation*}
    \begin{itemize}
        \item 单位:$rad/s$
        \item 线速度:$v=\omega r$
        \item 周期:$T=\frac{2\pi}{\omega}$
    \end{itemize}
\end{itembox}
作为对圆周运动方式的本质描述,角速度是同时具有方向与大小的矢量\footnote{角速度:$\vec{\omega}=\frac1{r^2}\vec{r}\times\vec{v}$}。
\begin{itembox}[l]{圆周运动加速度}
    \begin{itemize}
        \item 大小:
        \begin{equation*}
            a=\omega^2r=\frac{v^2}{r}
        \end{equation*}
        \item 方向:指向圆心
    \end{itemize}
\end{itembox}

\paragraph{向心力}物体做圆周运动的过程中时时刻刻都在改变着运动状态,那么实现这个改变的力便是\underline{向心力}。同时在生活中也常常使用离心力一词,在物理中也同样存在这个说法,其与向心力的区别在于观察角度的不同。
\begin{itemize}
    \item 向心力:是惯性系下观察时维持圆周运动的合外力
    \item 离心力:日文为遠心力,是非惯性系下观察时是物体保持静止的力
\end{itemize}
事实上,解决圆周问题的核心就在于发现是什么力构成了向心力。

\paragraph{圆锥摆}日文为円すい振り子,是一种简单常见的平面圆周运动模型。
\begin{figure}[ht!]
    \centering
    \begin{tikzpicture}
        \begin{scope}[yscale=0.25]
            \draw[dashed] (0, 0) circle (2);
        \end{scope}
        \draw[dashed] (0, 0) -- node[fill=white] {$r$} (-2, 0) -- node[fill=white] {$l$} (0, 2.5) -- (0, 0) node[below] {$O$} -- (2, 0);
        \draw (2, 0) -- (0, 2.5);
        \drawangle{(0, 0)}{(0, 2.5)}{(2, 0)};
        \draw[thick, -latex] (2, 0) -- ++ (-0.8, 1) node[right] {$T$};
        \draw[thick, -latex] (2, 0) -- ++ (0, -1) node[right] {$mg$};
        \fill[fill=gray] (2, 0) circle (3pt);
    \end{tikzpicture}
    \caption{圆锥摆}
\end{figure}
受力分析后可知,水平方向上$T\sin\theta=F_\textrm{向}$,竖直方向上$T\cos\theta=mg$。联立可解角速度大小。
\begin{equation*}
    \omega=\sqrt{\frac{g}{l\cos\theta}}
\end{equation*}

\subsubsection{非等速圆周运动}

\paragraph{绳线模型}接下来考虑如下模型:竖直平面内一个由线牵引的小球的运动。
\begin{figure}[ht!]
    \centering
    \begin{tikzpicture}
        \draw (0, -4) -- node[fill=white] {$r$} (0, 0) -- (-45:4);
        \draw[dashed] (-90:4) arc (-90:-20:4);
        \drawangle{(0, -4)}{(0, 0)}{(-45:4)};
        \draw[dashed] (0, -4) -- (-2, -4);
        \draw[dashed] (-45:4) -- (-2, {-2*sqrt(2)});
        \draw[<->] (-1.8, -4) -- node[fill=white] {$r(1-\cos\theta)$} (-1.8, {-2*sqrt(2)});
        \draw[thick, -latex] (0, -4) -- node[below] {$v_0$} ++ (1, 0);
        \fill[gray] (0, -4) circle (3pt);
        \draw[thick, -latex] (-45:4) -- node[right] {$v$} ++ ($0.5*(45:2)$);
        \draw[thick, -latex] (-45:4) -- node[above] {$T$} ++ ($0.8*(135:2)$);
        \draw[thick, -latex] (-45:4) -- ++ (0, -1) node[below] {$mg$};
        \fill[gray] (-45:4) circle (3pt);
    \end{tikzpicture}
    \caption{竖直平面圆周运动}
\end{figure}
将合力/加速度在平行和垂直运动方向上分解后可发现,垂直于运动方向的部分只负责维持各个瞬间的圆周运动,而平行于运动方向的部分只负责调整速度。此外,运动的物体只受重力(保存力)和绳子的拉力,而拉力时时刻刻又与运动方向垂直(不做功),所以整个系统机械能守恒。

\subparagraph{拉力大小}
\begin{align*}
    T=&G\cos\theta+m\frac{v^2}{r}\\
    \downarrow&\quad\frac12m{v_0}^2=\frac12mv^2+mgr(1-\cos\theta)\\
    =&mg\cos\theta+\frac{m{v_0}^2}{r}-2mg(1-\cos\theta)\\
    =&\frac{m{v_0}^2}{r}+mg(3\cos\theta-2)
\end{align*}

\subparagraph{最小初速度}
\begin{equation*}
    \begin{cases}
        \frac12m{v_0}^2=\frac12mv^2+2mgr\\
        mg=m\frac {v^2}r
    \end{cases}
    \implies v_{min}=v_0=\sqrt{5gr}
\end{equation*}

\paragraph{其他变形}在实际题目中除了用绳线做牵引,还有一些其他相似模型。
\begin{figure}[ht!]
    \centering
    \renewcommand\arraystretch{1.2}
    \begin{tabular}{c|cc}
        \hline
        牵引物&半周前&半周后\\\hline
        线/圆筒面&拉力&无\\
        棒/管内&拉力&支持力\\\hline
    \end{tabular}
    \caption{非等速圆周运动模型}
\end{figure}

\subsection{简谐振动}
\label{subsec:简谐振动}

\subsubsection{基本概念}

首先考虑如下模型:物体在平面上做等速圆周运动,其一侧有光源,另一侧有一光屏,观察光屏上点的运动模式。
\begin{figure}[ht!]
    \centering
    \begin{tikzpicture}
        \begin{scope}[xshift=-80]
            \draw[->] (-1.5, 0) -- (1.5, 0);
            \draw[->] (0, -1.5) -- (0, 1.5);
            \draw (0, 0) circle (1);
            \fill[fill=gray, opacity=0.3] (0, 0) -- (1, 0) arc (0 : 120 : 1) --cycle;
            \coordinate (P) at (120 : 1);
            \fill (P) circle (2pt);
            \draw (0, 0) -- (P);
            \draw[-latex] (150:0.5) arc (150:330:0.5) node[left] {$\omega$};
        \end{scope}
        \begin{scope}
            \draw[->] (-0.2, 0) -- ({rad(240)}, 0) node[right] {$t$};
            \draw[->] (0, -1.5) -- (0, 1.5) node[above] {$y$};
            \draw[domain=0:rad(200)] plot[samples=50] (\x, {sin(2*\x r)});
            \coordinate (Q) at ({rad(60)}, {sin(60)});
            \fill (Q) circle (2pt);
        \end{scope}
        \draw[dashed] (P) -- (Q);
    \end{tikzpicture}
    \caption{简谐振动}
\end{figure}
可见该点的位置(y坐标)与时间呈三角函数关系。这种物理量与时间呈三角函数关系的运动即为\underline{简谐振动},日文为単振動。其运动学信息可由等速圆周运动轻松求得。
\begin{itembox}[l]{简谐振动的运动学信息}
    \begin{itemize}
        \item 位移:$x=A\sin(\omega t+\theta_0)$
        \item 速度:$v=A\omega\cos(\omega t+\theta_0)$
        \item 加速度:$a=-A\omega^2\sin(\omega t+\theta_0)=-\omega^2x$
    \end{itemize}
\end{itembox}

\subsubsection{回复力}

结合牛二定律可见,做简谐振动的物体始终都会受到一个与位移方向相反的力,名为\underline{回复力},日文为復元力。因为位移的方向是以振动中心为基准向外的,所以回复力的方向即是始终指向其振动中心。
\begin{itembox}[l]{回复力}
    \begin{equation*}
        F=ma=-m\omega^2x=-kx\quad(k=m\omega^2)
    \end{equation*}
    \begin{itemize}
        \item $F\propto x$
        \item 方向指向振动中心
    \end{itemize}
\end{itembox}
此外,回复力的数学形式与弹簧弹力相像,所以不难推断回复力做功也与路径无关,属于保存力。因此简谐振动也满足机械能守恒。

虽然简谐振动中角速度不存在实际意义,但利用$T=\frac{2\pi}{\omega}$的关系,我们就可以得到简谐振动的周期。
\begin{itembox}[l]{简谐振动周期}
    \begin{equation*}
        T=2\pi\sqrt{\frac{m}{k}}
    \end{equation*}
\end{itembox}

\subsubsection{弹簧振子}

日文为バネ振り子,是最常见的简谐振动模型。
\begin{figure}[ht!]
    \centering
    \begin{tikzpicture}
        \begin{scope}
            \draw (0, 0) -- (5, 0);
            \fill[fill=gray, opacity=0.3] (0, 0) rectangle (5, -0.3);
            \draw (0.2, 0) -- (0.2, 0.5);
            \draw[spring] (0.2, 0.25) -- ++ (3.2, 0);
            \draw[fill=white] (3.5, 0.25) circle (6pt);
            \draw[-latex, rounded corners=2pt] (3.5, 1) -- (5, 1) -- (5, 0.9) -- (2, 0.9) -- (2, 0.8) -- (3.5, 0.8);
        \end{scope}
        \begin{scope}[xshift=200, yshift=50]
            \draw (0, 0) -- (3, 0);
            \fill[fill=gray, opacity=0.3] (0, 0) rectangle (3, 0.3);
            \draw[spring] (1, 0) -- ++ (0, -2);
            \draw[spring] (2, 0) -- ++ (0, -2.3);
            \draw[fill=white] (2, -2.5) circle (6pt);
            \draw[-latex, rounded corners=2pt] (2.3, -2.5) -- (2.3, -4) -- (2.4, -4) -- (2.4, -1) -- (2.5, -1) -- (2.5, -2.5);
        \end{scope}
    \end{tikzpicture}
    \caption{弹簧振子}
\end{figure}
对于水平放置的弹簧,其回复力就是弹簧弹力,振动中心为原长处。对于竖直放置的情况受力分析后,可列如下等式。
\begin{align*}
    F=&mg-kx\\
    \downarrow&\quad mg=kx_0\\
    =&-k(x-x_0)
\end{align*}
即回复力中的k值仍旧是弹簧的弹性系数,振动中心下移至了平衡位置。更一般的,对于任意形式放置的弹簧上述结论皆成立。

\subsubsection{单摆}

日文为単振り子,属于摆动幅度极小的竖直平面圆周运动的模型。
\begin{figure}[ht!]
    \centering
    \begin{tikzpicture}
        \draw (-0.5, 0) -- (0.5, 0);
        \fill[fill=gray, opacity=0.3] (-0.5, 0) rectangle (0.5, 0.3);
        \draw[->] (-1.5, -1.5) -- (1.5, -1.5);
        \draw[dashed] (-125:1.5) arc (-125:-55:1.5);
        \draw (-60:1.5) -- (0, 0);
        \draw[dashed] (-90:1.5) -- (0, 0) -- node[fill=white] {$l$} (-120:1.5);
        \drawangle{(-90:1.5)}{(0, 0)}{(-60:1.5)};
        \fill[fill=gray] (-60:1.5) circle (3pt);
    \end{tikzpicture}
    \caption{单摆}
\end{figure}
鉴于其周期推导略复杂,这里直接给出结论。
\begin{itembox}[l]{单摆周期公式}
    \begin{equation*}
        T=2\pi\sqrt{\frac{l}{g}}
    \end{equation*}
\end{itembox}

\subsection{天体运动}
\label{subsec:天体运动}

\subsubsection{开普勒定律}

开普勒定律是开普勒基于其老师第谷的实验数据总结而得的。而且此定律不仅适用于恒星与其行星之间,行星与其卫星也同样适用。
\begin{itembox}[l]{开普勒定律}
    \begin{itemize}
        \item 轨道法则:行星运动在以恒星为焦点的椭圆轨道上
        \item 面积法则:行星与恒星连线的$\begin{cases}\textrm{在单位时间内扫过的面积}\\\textrm{面积速度}\end{cases}$相等
        \item 周期法则:$T^2=ka^3\quad(k:const)$
    \end{itemize}
\end{itembox}
\begin{figure}[ht!]
    \centering
    \begin{tikzpicture}
        \fill ({-sqrt(5)}, 0) circle (1.5pt);
        \draw (0, 0) circle (3 and 2);
        \draw ({-sqrt(5)}, 0) -- node[fill=white] {$r$} ++ ($({sqrt(5)}, 0)+(10:3 and 2)$);
        \draw ({-sqrt(5)}, 0) -- ++ ($({sqrt(5)}, 0)+(30:3 and 2)$);
        \draw[thick, -latex] (10:3 and 2) -- ++ (-0.3, 1) node[right] {$v$};
        \drawangle{($(10:3 and 2)+(-0.3, 1)$)}{(10:3 and 2)}{(0, 0)};
    \end{tikzpicture}
    \caption{面积速度}
\end{figure}
\begin{itembox}[l]{面积速度}
    \begin{equation*}
        \frac{dS}{dt}=\frac{\frac12r(v\cdot dt\sin\theta)}{dt}=\frac12rv\sin\theta
    \end{equation*}
\end{itembox}

\subsubsection{万有引力定律}

\paragraph{万有引力}是牛顿根据开普勒第三定律的思想发展出来的,描述物体间引力的理论。
\begin{itembox}[l]{万有引力公式}
    \begin{equation*}
        F=G\frac{m_1m_2}{r^2}\quad(\textrm{万有引力定数:}G=6.67\times10^{-11}N\cdot m^2\cdot kg^{-2})
    \end{equation*}
\end{itembox}
在实际运算过程中,一般将地表附近的万有引力视为重力。
\begin{itembox}[l]{黄金代换式}
    \begin{equation*}
        G\frac{Mm}{R^2}=mg\implies
        GM=gR^2
    \end{equation*}
\end{itembox}
但严格意义上讲,由于地球自转,地表上的物体还会受到离心力(非惯性系)。因此,实际的g值会比黄金代换式求得的值小一些。
\begin{figure}[ht!]
    \centering
    \begin{tikzpicture}[scale=0.8]
        \fill (0, 0) circle (1.5pt);
        \draw (0, 0) circle (2);
        \draw[thick, -latex] (45:2) -- (45:1) node[left] {万有引力};
        \draw[thick, -latex] (45:2) -- ($(45:2)+(0.5,0)$) node[right] {离心力};
        \draw[thick, -latex] (45:2) -- ($(45:1)+(0.5,0)$) node[below] {重力};
    \end{tikzpicture}
    \caption{万有引力与重力}
\end{figure}

\paragraph{万有引力势能}与重力类似,万有引力也有其对应的势能,日文为万有引力による位置エネルギー。为方便运算一般取无限远处为势能基准点。
\begin{figure}[ht!]
    \centering
    \begin{tikzpicture}[scale=0.8]
        \filldraw[color=black, fill=gray, fill opacity=0.3] (-30:1.5) arc (-30:210:1.5);
        \draw (-4, 0) -- (2, 0);
        \draw[->] (0, 0) -- (0, 4) node[right] {正};
        \draw[dashed] (0, 4) -- (-3, 4);
        \draw[<->] (-3, 0) -- node[fill=white] {$\infty$} (-3, 4);
        \draw[dashed] (0, 3) -- (-2, 3);
        \draw[<->] (-2, 0) -- node[fill=white] {$r$} (-2, 3);
        \draw[thick, -latex] (0, 3) -- node[right] {万有引力} (0, 2);
        \fill[fill=gray, opacity=0.5] (0, 3) circle (5pt);
    \end{tikzpicture}
    \caption{万有引力势能}
\end{figure}
\begin{itembox}[l]{万有引力势能}
    \begin{equation*}
        E_p=-G\frac{Mm}{r}
    \end{equation*}
\end{itembox}

\paragraph{应用}基于万有引力和万有引力势能即可试求两个宇宙速度。
\begin{figure}[ht!]
    \centering
    \begin{tikzpicture}
        \fill[fill=gray, opacity=0.3] (0, 0) circle (1.2);
        \draw[thick, -latex] (90:1.4) arc (90:-260:1.4) node[left, above] {第一宇宙速度};
        \draw[thick, -latex] (90:1.4) arc (90:0:2 and 3) node[right] {第二宇宙速度};
    \end{tikzpicture}
    \caption{宇宙速度}
\end{figure}

\subparagraph{第一宇宙速度}能够围绕地球旋转的最小速度。
\begin{equation*}
    m\frac{v^2}{r}=G\frac{Mm}{r^2}\implies
    v=\sqrt{gr}
\end{equation*}

\subparagraph{第二宇宙速度}能够挣脱地球引力的最小速度。
\begin{equation*}
    \frac12mv^2-G\frac{Mm}{r}=0(E_k)+0(E_p)\implies
    v=\sqrt{2gr}
\end{equation*}


    % chapter 2

\chapter{热学}

% chapter 2 section 1

\section{热与能量}
\label{sec:2.1}

\paragraph{热}即是宏观上物体处于静止状态,但其中构成该物体的分子仍然在做着随机、杂乱无章的运动,即\underline{热运动}。在这个过程中分子彼此不断碰撞,交换着热运动时所具有的能量。这些转移在分子间的热运动的能量就是所谓的\underline{热}或是\underline{热能},日文为熱/熱エネルギー。

\paragraph{温度}由于一般的观察只能停留于宏观层面,很难真正地去计算分子间实际交换的能量,所以我们用温度这个宏观物理量去描述粒子的微观信息。
\begin{itembox}[l]{开氏温度}
    \centering
    絶対温度$T=t+273(K)$
\end{itembox}
\begin{itembox}[l]{热容量与比热}
    \begin{itemize}
        \item 热容量:物体升高1K所需的热量
        \item 比热:单位质量的物体升高1K所需的热量,物体单位质量的热容量
    \end{itemize}
    \begin{gather*}
        C=m\cdot c\\
        \Delta Q=C\cdot\Delta t=c\cdot m\cdot\Delta t
    \end{gather*}
\end{itembox}

\paragraph{断热容器}是不与周边物体做热交换的容器。在这样的环境下,容器内的物体只会与彼此传递热量,从而达到了整体热量不增不减的效果,即\underline{热量守恒},日文为熱量保存。
\begin{itembox}[l]{热量守恒定律}
    \begin{itemize}
        \item 系统中吸热=系统中放热
        \item $Q_\textrm{吸}=Q_\textrm{放}$
    \end{itemize}
\end{itembox}

\paragraph{潜热}在物体发生三态变化时,会有一段持续吸热/放热但温度不变的过程。因为物体的温度仅取决于内部分子的动能,然而物体改变状态时只有其分子间力引起的势能发生变化,所以虽然吸热/放热但温度不增减。我们将这个期间内吸收/放出的热量称为\underline{潜热}。常有融解热、蒸发热。
\begin{figure}[ht!]
    \centering
    \begin{tikzpicture}
        \node[draw, circle] (A) at (0, 0) {固态};
        \node[draw, circle] (B) at (60:3.6) {气态};
        \node[draw, circle] (C) at (3.6, 0) {液态};
        \draw[thick, <->] (A) -- node[fill=white] {昇華 昇華} (B);
        \draw[thick, <->] (A) -- node[fill=white] {凝固 融解} (C);
        \draw[thick, <->] (B) -- node[fill=white] {凝縮 蒸発} (C);
    \end{tikzpicture}
    \caption{三态变化}
\end{figure}

% chapter 2 section 2

\section{气体分子运动}

\paragraph{气体压强}日文为気体の圧力,由于气体分子冲撞(假想的)接触面而成,单位为帕斯卡(Pa)。
\begin{equation*}
    P=\frac{F}{S}
\end{equation*}

\subsection{气体法则}

\begin{itembox}[l]{波意尔查理定律}
    \begin{itemize}
        \item 波意尔定律(ボイルの法則)
        \begin{equation*}
            T=const\implies P\cdot V=const
        \end{equation*}
        \item 查理定律(シャルルの法則)
        \begin{equation*}
            P=const\implies\frac{V}{T}=const
        \end{equation*}
        \item 波意尔查理定律(ボイル・シャルルの法則)
        \begin{equation*}
            \frac{PV}{T}=const
        \end{equation*}
    \end{itemize}
\end{itembox}
\begin{figure}[ht!]
    \centering
    \begin{minipage}[t]{0.48\textwidth}
        \centering
        \begin{tikzpicture}[scale=0.8]
            \draw[->] (0, 0) -- (0, 3) node[above] {$P$};
            \draw[->] (0, 0) -- (3, 0) node[right] {$V$};
            \draw[thick, domain=0.18:2.7] plot (\x, {0.5/\x}) node[above] {低温};
            \draw[thick, domain=0.74:2.7] plot (\x, {2/\x}) node[above] {高温};
        \end{tikzpicture}
        \caption{波意尔定律}
    \end{minipage}
    \begin{minipage}[t]{0.48\textwidth}
        \centering
        \begin{tikzpicture}[scale=0.8]
            \draw[->] (0, 0) -- (0, 3) node[above] {$V$};
            \draw[->] (0, 0) -- (3, 0) node[right] {$T$};
            \draw[thick, dashed] (0, 0) -- (0.5, 0.5);
            \draw[thick] (0.5, 0.5) -- (2.7, 2.7);
        \end{tikzpicture}
        \caption{查理定律}
    \end{minipage}
\end{figure}

\paragraph{理想气体状态方程}\underline{理想气体}即忽略分子间力,严格满足波意尔查理定律的气体。因此,当一般气体处于\underline{高温+低压}的环境下时会表现得像理想气体。拓展波意尔查理定律后可得普适于理想气体的方程。
\begin{itembox}[l]{理想气体状态方程}
    \begin{equation*}
        PV=nRT\quad(R\textrm{:気体定数})
    \end{equation*}
\end{itembox}

\subsection{热力学第一定律}

\subsubsection{气体内能}

分子热运动的动能和由分子间力产生的势能共同构成一般的气体内能。由于理想气体忽略了其分子间力,所以理想气体的内能就等于其分子动能之和。
\begin{itembox}[l]{气体内能}
    \begin{equation*}
        U=\frac32nRT=\frac32PV\quad(U\propto T)
    \end{equation*}
\end{itembox}

\subsubsection{热力学第一定律}

一般有两种方式可以改变物体的内能:
\begin{itemize}
    \item 吸热/放热
    \item 做功/被做功
\end{itemize}
操作热的方式十分直观:通过宏观温度直接改变内能。另一方面,让气体做功的方式是通过其体积的改变来实现内能变化的。
\begin{itembox}[l]{气体做功}
    \begin{equation*}
        W=F\cdot\Delta x=PS\cdot\Delta x=P\Delta V
    \end{equation*}
    \begin{itemize}
        \item 压缩:$\Delta U>0$
        \item 膨胀:$\Delta U<0$
    \end{itemize}
\end{itembox}
于是将这两种改变内能的方式结合在一起便有了热力学第一定律的内容。
\begin{figure}[ht!]
    \centering
    \begin{minipage}[t]{0.48\textwidth}
        \centering
        \begin{tikzpicture}
            \draw[->] (-0.2, 0) -- (3, 0) node[right] {$V$};
            \draw[->] (0, -0.2) -- (0, 2) node[above] {$P$};
            \filldraw[color=black, fill=gray, fill opacity=0.3] (0, 0) -- 
                (0, 1) .. controls (1, 0.5) and (1.5, 2) .. (2.5, 1.5) -- (2.5, 0);
            \node at (1.25, 0.5) {$W$};
        \end{tikzpicture}
        \caption{气体做功}
    \end{minipage}
    \begin{minipage}[t]{0.48\textwidth}
        \centering
        \begin{tikzpicture}
            \draw[thick] (0, 2) -- (0.3, 2) {[rounded corners=5pt] -- (0.3, 0.3) -- (1.5, 0.3)} -- (1.5, 2) -- (1.8, 2) {[rounded corners=5pt] -- (1.8, 0) -- (0, 0)} -- cycle {};
            \filldraw[color=black, fill=gray, fill opacity=0.3] (0.3, 1.7) -- (1.5, 1.7) -- (1.5, 1.5) -- (0.3, 1.5) -- cycle;
            \draw[ultra thick, -latex] (0.9, -0.2) -- (0.9, 0.8);
            \draw[ultra thick, -latex] (0.9, 2) -- (0.9, 1);
            \node at (0.9, -0.5) {热量Q};
            \node at (0.9, 2.3) {功W};
        \end{tikzpicture}
        \caption{热力学第一定律}
    \end{minipage}
\end{figure}
\begin{itembox}[l]{热力学第一定律}
    \begin{equation*}
        \Delta U=Q_\textrm{吸}+W_\textrm{された}
    \end{equation*}
\end{itembox}

\subsection{气体状态变化}

分别固定气体的体积、压强、温度等物理量,我们能够得到以下几种典型状态变化。

\paragraph{等积变化}保持气体体积不变的变化。
\begin{itemize}
    \item $W=P\Delta V=0$
    \item $Q_\textrm{吸}=\Delta U=\frac32nR\Delta T=nC_v\Delta T$
    \item $C_v=\frac32R$:定積モル比熱
\end{itemize}

\paragraph{等压变化}保持气体压强不变的变化。
\begin{itemize}
    \item $W=P\Delta V=nR\Delta T$
    \item $Q_\textrm{吸}=\Delta U+W_\textrm{した}=\frac52nR\Delta T=nC_p\Delta T$
    \item $C_v=\frac52R$:定圧モル比熱
\end{itemize}

\paragraph{等温变化}保持气体温度不变的变化。
\begin{itemize}
    \item $\Delta U=0$
    \item $Q_\textrm{吸}=W_\textrm{した}$
\end{itemize}

\paragraph{断热变化}保持气体吸热为0的变化。
\begin{itemize}
    \item $Q_\textrm{吸}=0$
    \item $\Delta U=W_\textrm{された}$
    \begin{itemize}
        \item 断热膨胀:T减少,U减少
        \item 断热压缩:T增大,U增大
    \end{itemize}
\end{itemize}

\begin{figure}[ht!]
    \centering
    \begin{minipage}[t]{0.48\textwidth}
        \centering
        \begin{tikzpicture}
            \draw[->] (0, 0) -- (0, 3) node[above] {$P$};
            \draw[->] (0, 0) -- (3, 0) node[right] {$V$};
            \coordinate (p1v1) at (0.5, 2.8);
            \coordinate (p2v1) at (0.5, 0.5);
            \coordinate (p2v2) at (2.8, 0.5);
            \draw[thick, midarrow] (p2v1) -- (p1v1);
            \draw[thick, midarrow] (p2v2) -- (p2v1);
            \draw[thick, midarrow] (p1v1) .. controls (1.2, 1.2) .. (p2v2);
            \draw[dashed] (p1v1) -- (0, 2.8) node[left] {$P_1$};
            \draw[dashed] (p2v1) -- (0, 0.5) node[left] {$P_2$};
            \draw[dashed] (p2v1) -- (0.5, 0) node[below] {$V_1$};
            \draw[dashed] (p2v2) -- (2.8, 0) node[below] {$V_2$};
        \end{tikzpicture}
        \caption{气体状态变化}
    \end{minipage}
    \begin{minipage}[t]{0.48\textwidth}
        \centering
        \begin{tikzpicture}
            \draw[->] (0, 0) -- (0, 3) node[above] {$P$};
            \draw[->] (0, 0) -- (3, 0) node[right] {$V$};
            \draw[dashed, domain=0.37:2.7] plot (\x, {1/\x});
            \draw[dashed, domain=0.74:2.7] plot (\x, {2/\x});
            \draw[thick, midarrow] (0.74, 2.7) .. controls (1.2, 1.2) .. (2.5, 0.4);
        \end{tikzpicture}
        \caption{断热变化}
    \end{minipage}
\end{figure}

\subparagraph{断热自由膨胀}

一断热容器由挡板分隔为左右两部分,其中左侧空腔注有理想气体,右侧为真空。打开隔板,左侧气体自由扩散至右侧,最终充满整个容器。整个过程中气体扩散无有阻碍,因此不存在做功,加之容器断热,根据热力学第一定律可知整个系统能量不变。
\begin{itemize}
    \item $Q=W=0\implies\Delta U=0\implies\Delta T=0$
    \item $P_1V_1=P_2V_2$
\end{itemize}
\begin{figure}[ht!]
    \centering
    \begin{tikzpicture}
        \begin{scope}
            \fill [fill=gray, opacity=0.5] (1.5, 0) -- ++ (0, 2) {[rounded corners=5pt] -- ++(-1.5, 0) -- ++ (0, -2)} -- cycle {};
            \draw[ultra thick, rounded corners=5pt] (0, 0) rectangle (3, 2);
            \draw[thick] (1.5, 0) -- (1.5, 2);
            \node at (0.75, 1) {gas};
            \node at (2.25, 1) {vacuum};
            \draw[thick, -latex] (3.2, 1) -- (3.8, 1);
            \filldraw[ultra thick, rounded corners=5pt, color=black, fill=gray, fill opacity=0.3] (4, 0) rectangle (7, 2);
            \draw[thick] (5.5, 0) -- (5.5, 0.5) -- ++ (0.2, 0.2);
            \draw[thick] (5.5, 2) -- (5.5, 1.5) -- ++ (0.2, -0.2);
        \end{scope}
    \end{tikzpicture}
    \caption{断热自由膨胀}
\end{figure}

\subparagraph{断热气体混合}

与断热自由膨胀类似,只不过将断热容器的右侧空腔也注入理想气体,打开活塞后两种气体混合。从整体出发,硬质容器不发生体积变化且断热,因此根据热力学第一定律可得内部气体总内能不发生改变。
\begin{gather*}
    U_A+U_B=U\\
    \frac32n_ART_A+\frac32n_BRT_B=\frac32(n_A+n_B)RT\\
    T=\frac{n_AT_A+n_BT_B}{n_A+n_B}
\end{gather*}
\begin{figure}[ht!]
    \centering
    \begin{tikzpicture}
        \filldraw[ultra thick, color=black, fill=gray, fill opacity=0.8] (5:1) arc (5:355:1);
        \filldraw[ultra thick, color=black, fill=gray, fill opacity=0.3] ($(3, 0)+(-176.15:1.3)$) arc (-176.15:176.15:1.3);
        \draw[ultra thick] (5:1) -- ($(3, 0)+(176.15:1.3)$);
        \draw[ultra thick] (-5:1) -- ($(3, 0)+(-176.15:1.3)$);
        \draw[thick, |-] (1.35, 0.3) -- (1.35, -0.3);
        \node at (0, 0) {gas A};
        \node at (3, 0) {gas B};
    \end{tikzpicture}
    \caption{断热气体混合}
\end{figure}

\paragraph{热效率}吸热做功的机械为热机,其热到功的转化率为热效率。
\begin{itembox}[l]{热效率}
    \begin{equation*}
        \eta=\frac{W_\textrm{实际}}{Q_\textrm{吸}}
        =\frac{W_\textrm{した}-W_\textrm{された}}{Q_\textrm{吸}}
    \end{equation*}
\end{itembox}


    % chapter 3

\chapter{波动}

% chapter 3 section 1

\section{波的性质}

\subsection{波的传播}
\label{subsec:波的传播}

\subsubsection{波}

波即某种物理信息在空间中传播的现象。例如我们拿住一根绳子的一端,把其另一端固定在墙面上,抖动握持的一端,这便形成了一种简单的波。
\begin{figure}[ht!]
    \centering
    \begin{tikzpicture}
        \draw[thick] (5, 0.5) -- (5, -0.5);
        \fill (5, 0) circle (2pt);
        \draw[rounded corners=3pt, thick, -latex] (-0.2, 0.3) -- (-0.2, 0.7) -- (0, 0.7) -- (0, 0.3);
        \draw[dashed, domain=0:5] plot[smooth] (\x, {exp(-4*\x^2)});
        \draw[domain=0:5] plot[smooth] (\x, {exp(-4*(\x-3)^2)});
        \draw[thick, dashed, -latex] (1, 0.5) -- (2, 0.5);
    \end{tikzpicture}
    \caption{波的形成}
\end{figure}
将其中传递波的事物称为\underline{介质},日文为媒質。在上述例子中介质就是绳子上的每一段绳结。以下是一些描述波的时候常用的物理量。
\begin{itemize}
    \item 波形:描述某时刻波上各点状态的曲线
    \item 波速$v$:日文为波の速さ,即波的传播速度
    \item 波峰/波谷:日文为山/谷,即波形上的最高处/最低处
    \item 波长$\lambda$:相邻的波峰间的距离
    \item 振幅$A$:介质振动的位移
    \item 周期$T$:等同于介质的振动周期
    \item 频率$f$:日文为振動数,周期的倒数
\end{itemize}
基于上述信息便可得到波的基本公式。
\begin{itembox}[l]{波的基本公式}
    \begin{equation*}
        v=\frac{\lambda}{T}=\lambda f
    \end{equation*}
\end{itembox}

\subsubsection{横波与纵波}

\paragraph{定义}根据介质的振动方向和波的传播方向的关系,可以如下对波进行分类。
\begin{itemize}
    \item 横波:振动方向与传播方向垂直的波
    \item 纵波:振动方向与传播方向平行的波
\end{itemize}
其中纵波会呈现明显的疏密变化,所以也叫做疏密波。

\paragraph{描绘}对于横波,我们可以通过将各个介质的位置描绘在坐标平面上的方式来十分轻松地记录其波形等信息。然而若是使用相同的方式(逐个描点)去记录纵波则会发现所有点都聚集在x轴上,很难对其进行清晰地梳理。为此,我们仍然使用横波的绘图方式,只是对每个点的含义做重新诠释,使其能够更好地展现纵波的特征。
\begin{figure}[ht!]
    \centering
    \begin{tikzpicture}
        \draw[->] (0, -1) -- (0, 2) node[above] {$y$};
        \draw[->] (0, 0) -- (4, 0) node[right] {$x$};
        \draw[thick, domain=-0.5:{pi+0.5}] plot (\x, {sin(2*(\x r))});
        \draw[dashed] ({pi/8}, 0) -- ({pi/8}, {sin(pi/4 r)});
        \draw[thick, -latex] ({pi/8}, 0) -- ({pi/8+sin(pi/4 r)}, 0);
        \draw[thick, midarrow] ({pi/8+sin(pi/4 r)}, 0) arc (0:90:{sin(pi/4 r)});
        \filldraw[color=black, fill=white] ({pi/8}, 0) circle (2pt);
        \fill ({pi/8+sin(pi/4 r)}, 0) circle (2pt);
        \fill ({pi/8}, {sin(pi/4 r)}) circle (2pt);
        \node[below=5pt, fill=white] at ({pi/8+sin(pi/4 r)}, 0) {$(x+y,0)$};
        \node[above=5pt, fill=white] at ({pi/8}, {sin(pi/4 r)}) {$(x,y)$};
    \end{tikzpicture}
    \caption{纵波的图示}
\end{figure}
对于纵波上原本在$(x,0)$点的介质,倘若在某一时刻它关于自身的基准位置发生了$y$大小的偏移,那么根据纵波的定义,该点当前的坐标将会是$(x+y,0)$。既然如此,我们则可以将该点的基准位置信息和振动信息分离,让x轴和y轴分别来呈现,也就是将这个介质描绘在$(x,y)$处。得益于这种表现方式,我们就可以更容易地判断纵波的疏部与密部的位置了。比如\tikz{\draw (-0.2,-0.2) cos (0,0) -- (0,0) sin (0.2,0.2);}的部分,其中左侧部分介质负向偏移、右侧部分正向偏移,中点两侧的介质均往两侧远离,因此这里是疏部。另一种情况也是如此的分析方式。

\subsubsection{波的解析式}

至此不难发现波上每一个介质的振动情况是依存于当前的时间和其在空间上的位置的,因此为了准确描述任意时刻波上任意位置的介质的位置就需要用形如$y=f(x,t)$的函数才能完成。
\begin{figure}[ht!]
    \centering
    \begin{tikzpicture}
        \draw[->] (0,0,-0.5) -- (0,0,5) node[below] {$t$};
        \draw[->] (-0.5,0,0) -- (5,0,0) node[right] {$x$};
        \draw[->] (0,-1,0) -- (0,1,0) node[above] {$y$};
        \draw[thick, -latex] (1.5,1,0) -- node[above] {x-y} (2.5,1,0);
        \draw[thick, -latex] (0,1,2) -- node[above left] {y-t} (0,1,4);
        \draw[thick, draw=gray!50] plot[variable=\x,domain=0:3, samples=100,smooth] (\x, {sin((pi*(0-\x)) r)/2}, 0) node[below right] {$t=0$};
        \draw[dashed] (0,0,2) -- (5,0,2);
        \draw[thick, draw=gray!75] plot[variable=\x,domain=0:4, samples=100,smooth] (\x, {sin((pi*(1-\x)) r)/2}, 2) node[below right] {$t=1$};
        \draw[dashed] (0,0,4) -- (5,0,4);
        \draw[thick, draw=gray!100] plot[variable=\x,domain=0:5, samples=100,smooth] (\x, {sin((pi*(2-\x)) r)/2}, 4) node[below right] {$t=2$};
        \draw[thick] plot[variable=\t,domain=0:4, samples=100,smooth] (0, {sin((pi*(\t-0)) r)/2}, \t);
        \begin{scope}[xyplane=0]
            \fill[fill=gray, fill opacity=0.2] (0,-0.7) rectangle (4.5,0.7);
        \end{scope}
        \begin{scope}[yzplane=0]
            \fill[fill=gray, fill opacity=0.2] (-0.7,0) rectangle (0.7,4);
        \end{scope}
    \end{tikzpicture}
    \caption{波的解析式}
\end{figure}
于是,基于波的定义,我们可以从波源着手来推导正弦波的解析式。正弦波的波源做着简谐振动,假设其振动为$y=A\sin(\omega t)$。那么,波源的振动将会在$\frac{x}{v}$时间后传播到波上处于$x$位置上的任意一点,因此该点的$y-t$振动图像就是波源振动图像向右平移$\frac{x}{v}$个单位后的样子。
\begin{itembox}[l]{波的解析式}
    \begin{align*}
        y(x,t)=&A\sin\left(\omega\left(t-\frac{x}{v}\right)\right)\\
        =&A\sin\left(2\pi\left(\frac{t}{T}-\frac{x}{\lambda}\right)\right)
    \end{align*}
\end{itembox}
其中$\sin$内部的内容叫做位相、波的传播方向可由$v$做调整。同样的,通过平移某一时刻$y-x$图像的方式也可以完成上述推导。

\subsection{波的干涉}
\label{subsec:波的干涉}

\paragraph{脉冲波}日文为パルス波,指的是由一组振动形成的波。
\begin{figure}[ht!]
    \centering
    \begin{tikzpicture}
        \draw[dashed] (0,0) -- (4,0);
        \draw[domain=0:4] plot[smooth] (\x, {exp(-4*(\x-2)^2)});
    \end{tikzpicture}
    \caption{脉冲波}
\end{figure}

\subsubsection{波的叠加原理}

波在传播的过程中遵循\underline{叠加原理}和\underline{独立性原理}。即在与其他波相遇时会发生叠加,且满足$y=y_1+y_2$,在彼此分开后仍然保持叠加前的振动方式。
\begin{figure}[ht!]
    \centering
    \begin{tikzpicture}
        \begin{scope}
            \draw[dashed] (0,0) -- (6,0);
            \draw[domain=0:3] plot[smooth] (\x, {exp(-4*(\x-1)^2)});
            \draw[domain=3:6] plot[smooth] (\x, {0.5*exp(-4*(\x-5)^2)});
            \draw[thick, -latex] (2,0.5) -- (2.5, 0.5);
            \draw[thick, -latex] (4,0.5) -- (3.5, 0.5);
        \end{scope}
        \begin{scope}[yshift=-50]
            \draw[dashed] (0,0) -- (6,0);
            \draw[dashed, domain=2.5:3.5] plot[smooth] (\x, {exp(-4*(\x-2.5)^2)});
            \draw[dashed, domain=2.5:3.5] plot[smooth] (\x, {0.5*exp(-4*(\x-3.5)^2)});
            \draw[domain=0:6] plot[smooth] (\x, {exp(-4*(\x-2.5)^2)+0.5*exp(-4*(\x-3.5)^2)});
        \end{scope}
        \begin{scope}[yshift=-100]
            \draw[dashed] (0,0) -- (6,0);
            \draw[domain=3:6] plot[smooth] (\x, {exp(-4*(\x-5)^2)});
            \draw[domain=0:3] plot[smooth] (\x, {0.5*exp(-4*(\x-1)^2)});
            \draw[thick, -latex] (2.5,0.5) -- (2, 0.5);
            \draw[thick, -latex] (3.5,0.5) -- (4, 0.5);
        \end{scope}
    \end{tikzpicture}
    \caption{波的叠加原理}
\end{figure}

\subsubsection{平面波的干涉}

设想平面上有两个同样的波源同时产生相同的波(周期、最大振幅、位相均相同)。倘若俯视观察平面,则会发现以波源为圆心形如同心圆样式的波纹。其中虚线代表波谷的波面,实线代表波峰的波面,即两个虚线之间的长度为一个波长。根据波的叠加原理分析可知这些波面会发生相互加强/减弱的现象,我们将其称为\underline{波的干涉}。
\begin{figure}[ht!]
    \centering
    \begin{tikzpicture}[scale=0.8]
        \coordinate (A) at (-2.2,0);
        \coordinate (B) at (2.2,0);
        \coordinate (P) at (1.36,1.82);
        \clip (-5,-4) rectangle (5,4);
        \fill (A) circle (2pt) node[left] {A};
        \fill (B) circle (2pt) node[right] {B};
        \foreach \r in {1, 3, 5, 7} {
            \draw[dashed] (A) circle (\r);
            \draw[dashed] (B) circle (\r);
        }
        \foreach \r in {2, 4, 6, 8} {
            \draw (A) circle (\r);
            \draw (B) circle (\r);
        }
        \fill (P) circle (2pt) node[above] {P};
        \draw[thick] (A) -- node[fill=white] {$r_1$} (P);
        \draw[thick] (B) -- node[fill=white] {$r_2$} (P);
    \end{tikzpicture}
    \caption{平面波的干涉}
\end{figure}
总结平面上加强点、减弱点的规律可得波的干涉条件。
\begin{itembox}[l]{波的干涉条件}
    \begin{itemize}
        \item 加强:$|r_1-r_2|=2n\cdot\frac{\lambda}{2}$
        \item 减弱:$|r_1-r_2|=(2n+1)\cdot\frac{\lambda}{2}$
    \end{itemize}
\end{itembox}
其中的加强点,日文为腹,是两个波峰或者两个波谷叠加而得,均为整个波中振幅最大、振动最剧烈的位置,因此叠加后的结果也将会是振动最剧烈的位置。相反,减弱点,日文为節,是由波峰和波谷叠加而得,振幅相互抵消,因此叠加后将不会产生明显的振动。此外,观察干涉条件的数学形式:干涉点到两波源的差值一定,根据圆锥曲线的定义可知加强点/减弱点的连线为双曲线。

\subsubsection{驻波}

\begin{figure}[ht!]
    \centering
    \begin{tikzpicture}
        \foreach \t/\shift/\xshift/\yshift in {
            $t=0$           /0       /0  /0,
            $t=\frac{T}{8}$ /{1*pi/4}/0  /-180,
            $t=\frac{T}{4}$ /{2*pi/4}/0  /-360,
            $t=\frac{3T}{8}$/{3*pi/4}/0  /-540,
            $t=\frac{T}{2}$ /{4*pi/4}/600/0,
            $t=\frac{5T}{8}$/{5*pi/4}/600/-180,
            $t=\frac{3T}{4}$/{6*pi/4}/600/-360,
            $t=\frac{7T}{8}$/{7*pi/4}/600/-540
        } {
            \begin{scope}[scale=0.3, xshift=\xshift, yshift=\yshift]
                \draw[->] (0,-2.5) -- (0,2.5);
                \draw[->] (-6.5,0) -- (6.5,0);
                \node at (-9,0) {\t};
                \draw[dashed, domain=-6:6] plot[smooth] (\x, {sin((\x-\shift) r)});
                \draw[solid, domain=-6:6] plot[smooth] (\x, {-sin((\x+\shift) r)});
                \draw[very thick, domain=-6:6] plot[smooth] (\x, {sin((\x-\shift) r)-sin((\x+\shift) r)});
                \foreach \x in {
                    {-1*pi/2}, {-2*pi/2}, {-3*pi/2},
                    {1*pi/2}, {2*pi/2}, {3*pi/2}
                } {\draw[dotted] (\x,2.5) -- (\x,-2.5);}
            \end{scope}
        }
    \end{tikzpicture}
    \caption{驻波}
\end{figure}
观察上述干涉实验中波源连线上的振动情况,可发现直线上加强点、减弱点交替出现的现象,好似合成波没在移动一般。我们将这样的波称为\underline{驻波},日文为定常波。一般题目中常出现数干涉点个数的问题,为此我们可以根据相邻加强点和减弱点的间距为$\frac{\lambda}{4}$的条件,通过逐点加和的方式处理。

\subsection{衍射·反射·折射}
\label{subsec:衍射·反射·折射}

\subsubsection{衍射}

波绕过障碍物的现象称为\underline{衍射},日文为回折,其本质可由惠更斯原理\footnote{波上各点都可视作为一个新的球面波源}解释。对于波长较长的波,其易衍射,传播范围广,生活中调频广播属于此类。相反,波长较短的波不易衍射,会表现出较强的“直进性”,5G通讯信号属于此类。
\begin{figure}[ht!]
    \centering
    \begin{tikzpicture}
        \draw (-90:0.3) arc (-90:90:0.3);
        \draw (-90:1) arc (-90:90:1);
        \filldraw[color=black, fill=white] (-0.1, 0.1) rectangle (0.1, 1.5);
        \filldraw[color=black, fill=white] (-0.1, -0.1) rectangle (0.1, -1.5);
        \foreach \x in {-1.5, -1, -0.5} {\draw (\x, 1) -- (\x, -1);}
    \end{tikzpicture}
    \caption{波的衍射}
\end{figure}

\subsubsection{反射}

\begin{figure}[ht!]
    \centering
    \begin{tikzpicture}
        \fill[fill=gray, opacity=0.3] (0,0) rectangle (3,-0.3);
        \draw (0,0) -- (3,0);
        \draw[thick, midarrow] (0,2) -- (1.5,0);
        \draw[thick, midarrow] (1.5,0) -- (3,2);
        \draw[dashed, rounded corners=2pt, -latex] (1.45,1) -- (1.45,0.2) -- (1.55,0.2) -- (1.55,1);
    \end{tikzpicture}
    \caption{波的反射}
\end{figure}
直观看来波在发生反射时遵循\underline{反射定律}\footnote{入射角=反射角},且整个过程中波的\underline{速度}、\underline{波长}、\underline{频率}不发生改变。但在介质层面则需要考虑反射端点类型的问题,一般分为以下两种:
\begin{itemize}
    \item 自由端:反射位置的介质可以自由移动的情况
    \item 固定端:反射位置的介质无法自由移动的情况
\end{itemize}
并通过假想一个不存在的波的方式处理各个时刻的反射波。
\begin{figure}[ht!]
    \centering
    \renewcommand\arraystretch{1.2}
    \begin{tabular}{c|cc}
        \hline
        反射端&反射波图示&反射波相变\\\hline
        自由端&镜面对称&0相变\\
        固定端&中心对称&$\pi$相变\\\hline
    \end{tabular}
    \caption{端点反射}
\end{figure}
\begin{figure}[ht!]
    \begin{minipage}[t]{0.48\textwidth}
        \centering
        \begin{tikzpicture}
            \fill[fill=gray, opacity=0.3] (0,2) rectangle (0.3,-0.5);
            \draw (0,-0.5) -- (0,2);
            \draw (-2.5,0) -- (2.5,0);
            \draw[thick, domain=-2:2] plot[smooth] (\x, {exp(-4*(\x+0.3)^2)});
            \draw[dashed, domain=-2:2] plot[smooth] (\x, {exp(-4*(\x-0.3)^2)});
            \draw[very thick, domain=-2:0] plot[smooth] (\x, {exp(-4*(\x+0.3)^2)+exp(-4*(\x-0.3)^2)});
            \draw[thick, -latex] (-2,1) -- (-1,1);
            \draw[thick, dashed, -latex] (2,1) -- (1,1);
        \end{tikzpicture}
        \caption{自由端反射}
    \end{minipage}
    \begin{minipage}[t]{0.48\textwidth}
        \centering
        \begin{tikzpicture}
            \fill[fill=gray, opacity=0.3] (0,1.2) rectangle (0.3,-1.2);
            \draw (0,-1.2) -- (0,1.2);
            \draw (-2.5,0) -- (2.5,0);
            \draw[thick, domain=-2:2] plot[smooth] (\x, {exp(-4*(\x+0.3)^2)});
            \draw[dashed, domain=-2:2] plot[smooth] (\x, {-exp(-4*(\x-0.3)^2)});
            \draw[very thick, domain=-2:0] plot[smooth] (\x, {exp(-4*(\x+0.3)^2)-exp(-4*(\x-0.3)^2)});
            \draw[thick, -latex] (-2,1) -- (-1,1);
            \draw[thick, dashed, -latex] (2,-1) -- (1,-1);
        \end{tikzpicture}
        \caption{固定端反射}
    \end{minipage}
\end{figure}

\subsubsection{折射}

日文为屈折,内容较为基础,在此给出折射定律。
\begin{itembox}[l]{折射定律}
    \begin{equation*}
        \frac{\sin\theta_1}{\sin\theta_2}=
        \frac{v_1}{v_2}=
        \frac{\lambda_1}{\lambda_2}=
        \frac{n_2}{n_1}(=n_{12})
    \end{equation*}
\end{itembox}
需要注意公式中$n_{12}$的定义。此外,将折射角大于$90^\circ$的情况称为\underline{全反射},题目中一般考察临界入射角。

% chapter 3 section 2

\section{声波}

\subsection{声波}

\subsubsection{基本信息}

声波日文为音波,其基本内容与第一节相仿,在此仅逐条列举一些基本信息。
\begin{itemize}
    \item 介质:任何物体
    \item 本质:纵波
    \item 频率:20$\sim$20000Hz(人耳感知范围)
    \item 声速:空气中声速随温度改变:$v\approx331.5+0.6t$
    \item 三要素
    \begin{itemize}
        \item 音の高さ:取决于频率$f$
        \item 音色:取决于传递声音的介质
        \item 音の強さ:取决于$f^2A^2$,即日常使用的分贝(db, decibel)
    \end{itemize}
\end{itemize}

\subsubsection{差频}

日文为うなり,是两束频率相近的声波叠加时所发生的相互加强减弱的现象。其叠加而成的声波的频率满足以下公式。
\begin{itembox}[l]{差频公式}
    \begin{equation*}
        f=|f_1-f_2|
    \end{equation*}
\end{itembox}
\begin{figure}[ht!]
    \centering
    \begin{tikzpicture}
        \begin{scope}[yscale=0.5]
            \draw[->] (-4.5,0) -- (4.5,0) node[right] {$t$};
            \draw[domain=-4:4] plot[samples=100, smooth] (\x, {sin(10*\x r)});
        \end{scope}
        \begin{scope}[yscale=0.5, yshift=-100]
            \draw[->] (-4.5,0) -- (4.5,0) node[right] {$t$};
            \draw[domain=-4:4] plot[samples=100, smooth] (\x, {sin(12*\x r)});
        \end{scope}
        \begin{scope}[yscale=0.5, yshift=-200]
            \draw[->] (-4.5,0) -- (4.5,0) node[right] {$t$};
            \draw[domain=-4:4] plot[samples=100, smooth] (\x, {sin(10*\x r)+sin(12*\x r)});
            \draw[dashed, domain=-4:4] plot[samples=100, smooth] (\x, {2*cos(\x r)});
            \draw[dashed, domain=-4:4] plot[samples=100, smooth] (\x, {-2*cos(\x r)});
        \end{scope}
    \end{tikzpicture}
    \caption{差频}
\end{figure}
实际题目中常与下一小节的多普勒效应结合。

\subsection{多普勒效应}

日文为ドップラー効果,是观测者与波源存在相对运动时所发生的观测频率与波源频率不同的现象,日常生活中火车进站、警车鸣笛均属此类。根据观测者与波源的相对运动关系可分为以下三类。

\subsubsection{波源运动,观测者静止}

\begin{figure}[ht!]
    \centering
    \begin{tikzpicture}
        \clip (-3,-1.5) rectangle (3,1.5);
        \foreach \x/\r in {0/3, 0.2/2.5, 0.4/2, 0.6/1.5, 0.8/1} {
            \fill[opacity={\x+0.2}] (\x,0) circle (2pt);
            \draw[opacity={\x+0.2}] (\x,0) circle (\r);
        }
        \draw[thick, -latex] (0.4,0.5) -- (1,0.5);
    \end{tikzpicture}
    \caption{多普勒效应}
\end{figure}
由图可见,处于波源行进方向上的观测者会遇到更加密集的波面,反之则会遇到更加稀疏的波面。此所谓密集与稀疏实际指的是波面的间距,即波长。因此可以说波源的移动压缩/拉伸了波长。此时以波源为基准观察波速即为$V\pm v$。于是,根据波的基本公式可以得到以下结论。
\begin{equation*}
    V\pm v=f\lambda^\prime\implies
    \lambda^\prime=\frac{V\pm v}{f}
\end{equation*}
在观测者看来波仍然在以原速传播,所以观测者所感知到的频率即为
\begin{equation*}
    f^\prime=\frac{V}{\lambda^\prime}=\frac{V}{V\pm v}f
\end{equation*}

\subsubsection{波源静止,观测者运动}

此时,波源不移动,波长不发生改变。相当于观测者与各个波面在做追及+相遇,倘若两者相向而行,则单位时间内观测者会遇到更多的波面。反之则会遇到更少的波面。根据波的基本公式可以得到以下结论。
\begin{equation*}
    V\pm u=f^\prime\lambda\implies
    f^\prime=\frac{V\pm u}{\lambda}
    =\frac{V\pm u}{V}f
\end{equation*}

\subsubsection{波源运动,观测者运动}

基于上述两种情况的分析,我们可以把双方都运动的问题化简为运动的观测者感知波源压缩/拉伸后的波的问题。
\begin{itembox}[l]{多普勒效应}
    \begin{equation*}
        f^\prime=\frac{V\pm u}{V\pm v}f
    \end{equation*}
    \begin{itemize}
        \item V:波速
        \item u:观测者的速度
        \item v:波源的速度
    \end{itemize}
\end{itembox}
使用上述公式时应注意如下两点:
\begin{itemize}
    \item 观测者的速度和波源的速度的位置
    \item 分数中两个符号的判断\footnote{一般是结合推导过程记忆,但更推荐根据生活经验判断}
\end{itemize}
此外在实际题目中时而也会遇到一些其他变形。比如平面上的多普勒效应\footnote{基于运动的矢量性进行分解,只考虑发生多普勒效应的方向}、含有风的情况\footnote{风会带动介质运动,所以相当于改变了波速}等等。

\subsection{共振现象}

\paragraph{共振·共鸣}一般物体都具有一个固定的振动周期,当外界发生与之相同频率的变化时,该物体会受到其影响而发生距离振动的现象。当振动是以声音的形式展现时特称共鸣。

\subsubsection{弦振动}

拨动两端固定的弦,这个振动在弦上会以横波的形式向两端传播,到达端点后发生固定端反射,从而最终在弦上形成驻波。设弦的张力为$S$、线密度为$\rho$,则弦上波速可求。
\begin{figure}[ht!]
    \centering
    \begin{minipage}{0.48\textwidth}
        \begin{itembox}[l]{弦上波速}
            \begin{equation*}
                v=\sqrt{\frac{S}{\rho}}
            \end{equation*}
        \end{itembox}
        \begin{itembox}[l]{弦振动频率}
            \begin{itemize}
                \item $\lambda=\frac{2}{n}l$
                \item $f=\frac{nv}{2l}\implies nf_1=f_n$
            \end{itemize}
        \end{itembox}
    \end{minipage}
    \begin{minipage}{0.48\textwidth}
        \centering
        \begin{tikzpicture}
            \foreach \t/\yshift in {
                8/0, 4/-50, 2.67/-100, 2/-150
            } {
                \begin{scope}[yscale=0.8, yshift=\yshift]
                    \draw[color=gray] (0,0) -- (4,0);
                    \fill (0,0) circle (2pt);
                    \fill (4,0) circle (2pt);
                    \draw[domain=0:4] plot[samples=50, smooth] (\x, {0.5*sin((2*pi/\t)*(\x r)});
                    \draw[dashed, domain=0:4] plot[samples=50, smooth] (\x, {-0.5*sin((2*pi/\t)*(\x r)});
                \end{scope}
            }
        \end{tikzpicture}
        \caption{弦振动}
    \end{minipage}
\end{figure}
此外可以稳定形成驻波的振动均为弦的固有振动,其中我们将只有一个波峰的振动形式称为基本振动,称频率为基本振动频率n倍的振动形式为n倍振动。

\subsubsection{气柱振动}

与弦振动类似,通过吹气等方式使管内空气发生振动,同样也能形成驻波。当管口封闭时该端点形成固定端,反之形成自由端。我们将两端均开口的管称为开管,将一端开口、一端闭口的管称为闭管。分析其振动规律可得气柱振动的波长、频率等信息。此外对于闭管振动会出现开口端腹点(加强点)外移的现象\footnote{开口端补正:固定频率,改变管长,可由两次共振时的管长求得次偏移值。$\Delta x=\frac{l_2-3l_1}{2}$},但考试中鲜有提及,故此处略过。
\begin{figure}[ht!]
    \centering
    \begin{minipage}[t]{0.48\textwidth}
        \centering
        \begin{tikzpicture}
            \foreach \n/\t/\yshift in {
                1/8/0, 2/4/-50, 3/2.67/-100, 4/2/-150
            } {
                \begin{scope}[yscale=0.6, yshift=\yshift]
                    \node at (-1,0) {\n 倍振动};
                    \fill[fill=gray, opacity=0.2] (0,-0.6) rectangle (4,0.6);
                    \draw[thick] (0,0.6) -- (4,0.6);
                    \draw[thick] (0,-0.6) -- (4,-0.6);
                    \draw[domain=0:4] plot[samples=50, smooth] (\x, {0.5*cos((2*pi/\t)*(\x r)});
                    \draw[dashed, domain=0:4] plot[samples=50, smooth] (\x, {-0.5*cos((2*pi/\t)*(\x r)});
                \end{scope}
            }
        \end{tikzpicture}
        \caption{开管振动}
    \end{minipage}
    \begin{minipage}[t]{0.48\textwidth}
        \centering
        \begin{tikzpicture}
            \foreach \n/\t/\yshift in {
                1/16/0, 3/5.33/-50, 5/3.2/-100, 7/2.29/-150
            } {
                \begin{scope}[yscale=0.6, yshift=\yshift]
                    \node at (-1,0) {\n 倍振动};
                    \fill[fill=gray, opacity=0.2] (0,-0.6) rectangle (4,0.6);
                    \draw[thick] (4,-0.6) -- (0,-0.6) -- (0,0.6) -- (4,0.6);
                    \draw[domain=0:4] plot[samples=50, smooth] (\x, {0.5*sin((2*pi/\t)*(\x r)});
                    \draw[dashed, domain=0:4] plot[samples=50, smooth] (\x, {-0.5*sin((2*pi/\t)*(\x r)});
                \end{scope}
            }
        \end{tikzpicture}
        \caption{闭管振动}
    \end{minipage}
\end{figure}
\begin{itembox}[l]{气柱振动}
    \begin{itemize}
        \item 开管振动
        \begin{itemize}
            \item $\lambda=\frac{2}{n}l$
            \item $f=\frac{nv}{2l}\implies nf_1=f_n$
        \end{itemize}
        \item 闭管振动
        \begin{itemize}
            \item $\lambda=\frac{4}{2n-1}l$
            \item $f=\frac{(2n-1)v}{4l}\implies (2n-1)f_1=f_n$
        \end{itemize}
    \end{itemize}
\end{itembox}

% chapter 3 section 3

\section{光波}

光作为波的基本信息与第一节的内容相同,故在此略过,仅区分两个用语\footnote{为避免混淆,统一使用日语}。
\begin{itemize}
    \item 分散:虹、スペクトル(折射率不同导致)
    \item 散乱:青空、チンダル現象
\end{itemize}
其中分散现象由于折射率不同导致,体现光的波动性。而散乱现象主要体现光的粒子性,通俗地讲就是一些“跑偏”的光。

\subsection{光的折射}
\label{subsec:光的折射}

\subsubsection{平行层}

由于$\sin\theta_in_i=const$,所以无论中间途径多少层不同的介质,最终的出射光线都会与入射光线平行。
\begin{figure}[ht!]
    \centering
    \begin{tikzpicture}
        \filldraw[color=black, fill=gray, fill opacity=0.3] (0,0) rectangle (4,0.5);
        \filldraw[color=black, fill=gray, fill opacity=0.8] (0,0.5) rectangle (4,1);
        \draw[thick, midarrow] (0.5,2) -- (1,1);
        \draw[thick, midarrow] (1,1) -- (2.3,0.5);
        \draw[thick, midarrow] (2.3,0.5) -- (3,0);
        \draw[thick, midarrow] (3,0) -- (3.5,-1);
    \end{tikzpicture}
    \caption{平行层折射}
\end{figure}

\subsubsection{水中视差}

从空气中观察水中物体,由于光的折射,实物总会比看起来的更深。其中实际深度$h$和视觉深度$h^\prime$满足$\frac{h}{n}=h^\prime$的关系。
\begin{figure}[ht!]
    \centering
    \begin{tikzpicture}[scale=1.5]
        \fill[fill=gray, opacity=0.3] (0,0) rectangle (3,-2);
        \draw (0,0) -- (3,0);
        
        \coordinate (A) at (1,0); \node[above] at (A) {$A$};
        \coordinate (B) at (2,0); \node[below right] at (B) {$B$};
        \coordinate (P) at (1,-1.8); \node[left] at (P) {$P$};
        \coordinate (PP) at (1,-1.2); \node[left] at (PP) {$P^\prime$};
        
        \fill (P) circle (2pt);

        \draw[thick, -latex] (P) -- (B) -- (3,1.2) node[right] {eye};
        \draw[thick, dashed] (PP) -- (B);
        \draw[thick, dashed] (A) -- (P);
        \draw[thick, dashed] (2,1) -- (2,-1);
        \drawangle[$\alpha$]{(B)}{(PP)}{(A)};
        \drawangle[$\beta$]{(B)}{(P)}{(A)};
        \drawangle[$\alpha$]{(3,1.2)}{(B)}{(2,1)};
        \drawangle[$\beta$]{(P)}{(B)}{(2,-1)};
    \end{tikzpicture}
    \caption{水中视差}
\end{figure}
\begin{equation*}
    AB=h^\prime\tan\alpha=h\tan\beta
    \xrightarrow[\tan\approx\sin]{\sin\alpha=\sin\beta n}
    h^\prime=\frac{h}{n}
\end{equation*}

\subsubsection{透镜}

当给定透镜的部分定量信息后,通过透镜公式便可以其他相关定量信息。
\begin{itembox}[l]{透镜公式}
    \begin{equation*}
        \frac{1}{a}+\frac{1}{b}=\frac{1}{f}
    \end{equation*}
    \begin{itemize}
        \item 参数:
        \begin{center}
            \renewcommand\arraystretch{1.2}
            \begin{tabular}{c|cc}
                \hline
                参数&正数&负数\\\hline
                焦距f&凸透镜&凹透镜\\
                物距a&实物&虚物\\
                相距b&实相&虚像\\\hline
            \end{tabular}
        \end{center}
        \item 倍率:$m=\left\lvert\frac{b}{a}\right\rvert$
    \end{itemize}
\end{itembox}

\subsection{光的干涉}
\label{subsec:光的干涉}

\subsubsection{双缝干涉}

\paragraph{实验内容}双缝实验是英国科学家托马斯杨于1807年为了验证光的波动性所做的实验。从光源$S_0$发出的单色光,通过双缝$S_1$和$S_2$投射到光屏上,其中相比于双缝的间距$d$,光屏的距离$l$足够远。实验发现光屏上会呈现出明暗相间的条纹。
\begin{figure}[ht!]
    \centering
    \begin{tikzpicture}[scale=1.5]
        \draw[thick] (-0.5,0.3) -- (-0.5,-0.3);
        \draw[thick] (0,1) -- (0,0.55);
        \draw[thick] (0,0.45) -- (0,-0.45);
        \draw[thick] (0,-0.55) -- (0,-1);
        \draw (0,0) -- (3.5,0);
        \draw (3,-1) -- (3,1) -- (3.5,1);
        \draw[<->] (3.4,0) -- node[fill=white] {$x$} (3.4,1);

        \coordinate (S0) at (-0.5,0); \node[left] at (S0) {$S_0$};
        \coordinate (S1) at (0,0.5); \node[above left] at (S1) {$S_1$};
        \coordinate (S2) at (0,-0.5); \node[below left] at (S2) {$S_2$};
        \coordinate (H) at (0.4,-0.3); \node[below] at (H) {$H$};
        \coordinate (P) at (3,1); \node[above] at (P) {$P$};

        \draw[dashed] (S1) -- (-1.5,0.5);
        \draw[dashed] (S2) -- (-1.5,-0.5);
        \draw[<->] (-1.4,0.5) -- node[fill=white] {$d$} (-1.4,-0.5);
        \draw[<->] (0,-0.8) -- node[fill=white] {$l$} (3,-0.8);
        \draw[midarrow] (S0) -- (S1) -- (P);
        \draw[midarrow] (S0) -- (S2) -- (P);
        \draw[dashed] (0,0) -- (P);
        \draw[dashed] (0,0.5) -- (H);
        \drawangle[]{(S2)}{(S1)}{(H)};
        \drawangle{(3,0)}{(0,0)}{(P)};
    \end{tikzpicture}
    \caption{双缝干涉}
\end{figure}

\paragraph{干涉条件}根据图形关系可知,两束光线到达$P$点距离的差值为$|S_1P-S_2P|=\frac{dx}{l}$,再结合平面波干涉的结论可得双缝干涉条件。
\begin{itembox}[l]{双缝干涉条件}
    \begin{equation*}
        \frac{dx}{l}=
        \begin{cases}
            2m\cdot\frac{\lambda}{2}&\implies\textrm{明线}\\
            (2m+1)\cdot\frac{\lambda}{2}&\implies\textrm{暗线}
        \end{cases}
    \end{equation*}
\end{itembox}

\paragraph{条纹间距}基于相邻两次干涉位置可求双缝干涉的条纹间距。
\begin{equation*}
    \Delta x=x_m-x_{m-1}=\frac{l\lambda}{d}
\end{equation*}
因此,若是知道$l$,$d$,$\Delta x$的信息即可反向求出波长。

\paragraph{颜色变化}另外,如果光源发出的是全色光则会发现$x=0$的位置出现白色条纹,其两侧由紫色向红色渐变展开。因为对于所有颜色的光$x=0$的位置永远是第零次加强的位置,所以全部汇聚于此,呈现白色。再根据条纹间距可知,波长较短的紫色会离其第零次干涉位置更近,所以白色旁边先会出现紫色。

\paragraph{回折格子}即布满等距缝隙的薄片,其实验原理与双缝干涉一致,只是同时相互干涉的光线更多。因此,不同于双缝干涉,回折格子的明线会更加明显。在此只给出结论。
\begin{itembox}[l]{回折格子干涉条件}
    \begin{equation*}
        d\sin\theta=m\lambda
    \end{equation*}
\end{itembox}

\subsubsection{薄膜干涉}

\paragraph{反射相变}光波在发生反射时根据接触面光学性质的不同会发生位相的变化。在此将折射率大的介质称为\underline{光密},相反称折射率小的介质称为\underline{光疏}。
\begin{figure}[ht!]
    \centering
    \renewcommand\arraystretch{1.2}
    \begin{tabular}{c|cc}
        \hline
        反射&反射端&相变\\\hline
        光密$\to$光疏&自由端&0相变\\
        光疏$\to$光密&固定端&$\pi$相变\\\hline
    \end{tabular}
    \caption{反射相变}
\end{figure}

\paragraph{光程}日文为光学距離,指在统一相位变化的前提下描述光波走行距离的量,并且简称光程相等的两条路径\underline{等光程}。我们在所有干涉实验中着眼的便是这个物理量。
\begin{figure}[ht!]
    \centering
    \begin{tikzpicture}
        \foreach \n [count=\i] in {1.5, 2} {
            \begin{scope}[yscale=0.5, yshift={-100*(\i-1)}]
                \draw (-2.5,0) -- (5,0);
                \filldraw[color=black, fill=gray, fill opacity={0.3*\i}] (0,-1) rectangle ({4/\n},1);
                \draw[domain=-2:0] plot[samples=50] (\x, {sin((pi*\x) r)});
                \draw[domain=0:{4/\n}] plot[samples=50] (\x, {sin((\n*pi*\x) r)});
                \draw[domain={4/\n}:4.5] plot[samples=50] (\x, {sin((pi*(\x-4/\n)) r)});
                \draw[<->] (0,1.2) -- node[above] {$l_\i$} ({4/\n},1.2);
            \end{scope}
        }
    \end{tikzpicture}
    \caption{光程}
\end{figure}
\begin{itembox}[l]{光程公式}
    \begin{equation*}
        L=nl
    \end{equation*}
\end{itembox}

\paragraph{实验内容}空间中有一层薄膜,光波斜向而来,观察在膜的表面与底层发生反射的两束光线。实验发现,在膜的表面会出现明暗相间的干涉条纹。
\begin{figure}[ht!]
    \centering
    \begin{tikzpicture}[scale=0.8]
        \fill[fill=gray, opacity=0.3] (0,0) rectangle (6,-2);
        \draw (0,0) -- (6,0);
        \draw (0,-2) -- (6,-2);

        \draw[thick, midarrow] (1,1) -- (2,0);
        \draw[thick, midarrow] (2,2) -- (4,0);
        \draw[thick, -latex] (4,0) -- (5,1);
        \draw[thick] (2,0) -- (3,-2) -- (4,0);

        \node[below left] at (2,0) {$A$};
        \node[below right] at (4,0) {$B$};
        \node[below left] at (3,-2) {$C$};
        \node[right] at (4,-4) {$C^\prime$};
        \node[above right] at (3,1) {$D$};
        \node[right] at (2.4,-0.8) {$H$};

        \draw[dashed] (2,-1) -- (2,1);
        \draw[dashed] (2,0) -- (3,1);
        \draw[dashed] (2.4,-0.8) -- (4,0) -- (4,-4) -- (3,-2);
        \draw[dashed, <->] (0.1,0) -- node[right] {$d$} (0.1,-2);

        \drawangle[$\theta$]{(2,1)}{(2,0)}{(1,1)};
        \drawangle[$\phi$]{(2,-1)}{(2,0)}{(3,-2)};
    \end{tikzpicture}
    \caption{薄膜干涉}
\end{figure}

\paragraph{干涉条件}根据图形关系可知,两束光线在B点处的光程差为$2nd\cos\phi$。因此,结合反射相变可得如下干涉条件。
\begin{itembox}[l]{薄膜干涉条件}
    \begin{equation*}
        2nd\cos\phi=
        \begin{cases}
            2m\cdot\frac{\lambda}{2}&\implies
            \textrm{0相变明线/}\pi\textrm{相变暗线}\\
            (2m+1)\cdot\frac{\lambda}{2}&\implies
            \textrm{0相变暗线/}\pi\textrm{相变明线}
        \end{cases}
    \end{equation*}
\end{itembox}

\subsubsection{楔形膜干涉}

\paragraph{实验内容}空间中有两块以极小夹角叠放的玻璃砖,光线从其上方垂直而来,观察在上玻璃砖的底面与下玻璃砖的顶面发生反射的两束光线。实验发现,在上玻璃砖的表面会出现明暗相间的干涉条纹。
\begin{figure}[ht!]
    \centering
    \begin{tikzpicture}
        \draw (0,0) rectangle (3,-0.5);
        \draw[rotate=15] (0,0) rectangle (3,0.5);

        \draw[thick, -latex, rounded corners=2pt] ($(15:2)+(0,1)$) -- (15:2) -- (15:1.9) -- ($(15:1.9)+(0,1)$);
        \draw[thick, -latex, rounded corners=2pt] ($(15:2.1)+(0,1)$) -- (0:2) -- (0:2.1) -- ($(15:2.2)+(0,1)$);

        \drawangle{(0:3)}{(0,0)}{(15:3)};
        \draw[<->] (0,-0.1) -- node[below] {$x$} (2.1,-0.1);
        \draw[<->] (14:2.3) -- node[right] {$d$} (0:2.2);
    \end{tikzpicture}
    \caption{楔形膜干涉}
\end{figure}

\paragraph{干涉条件}根据图形关系不难发现两束光线的光程差为$2d$,一般夹角$\theta$为给定值,所以光程差也可以用$2x\tan\theta$表示。再结合反射相变可得如下干涉条件。
\begin{itembox}[l]{楔形膜干涉条件}
    \begin{equation*}
        2d=2x\tan\theta=
        \begin{cases}
            2m\cdot\frac{\lambda}{2}&\implies
            \textrm{0相变明线/}\pi\textrm{相变暗线}\\
            (2m+1)\cdot\frac{\lambda}{2}&\implies
            \textrm{0相变暗线/}\pi\textrm{相变明线}
        \end{cases}
    \end{equation*}
\end{itembox}

\paragraph{条纹间距}与双缝干涉条纹间距求解思路一致,楔形膜干涉中同种干涉条纹间距为
\begin{equation*}
    \Delta x=x_m-x_{m-1}=\frac{\lambda}{2\tan\theta}
\end{equation*}

\subsubsection{牛顿环干涉}

\paragraph{实验内容}在空间中有一块半径足够大的平凸透镜凸面朝下放在一块玻璃砖上,光线从其上方垂直而来,观察在平凸透镜的底面与玻璃砖的顶面发生反射的两束光线。实验发现,在平凸透镜表面会出现明暗相间的圆环状干涉条纹。
\begin{figure}[ht!]
    \centering
    \begin{tikzpicture}
        \filldraw[color=black, fill=gray, fill opacity=0.3] (-120:4) arc (-120:-60:4) --cycle;
        \filldraw[color=black, fill=gray, fill opacity=0.3] (-2,-4) rectangle (2,-4.5);

        \fill (0,0) circle (1.5pt);
        \draw[dashed] (0,0) -- (0,-4.7);
        \draw[dashed] (1.37,-3.76) -- (1.37,-4.7);
        \draw[<->] (0,-4.6) -- node[below] {$r$} (1.37,-4.6);
        \draw[dashed] (0,0) -- node[left] {$R$} (-70:4);
        \node at (2,-3.7) {$d$};

        \draw[thick, -latex, rounded corners=2pt] (1.37,-2) -- (-70:4) -- (1.27,-3.76) -- (1.27,-2);
        \draw[thick, -latex, rounded corners=2pt] (1.47,-2) -- (1.47,-4) -- (1.57,-4) -- (1.57,-2);
    \end{tikzpicture}
    \caption{牛顿环干涉}
\end{figure}

\paragraph{干涉条件}根据图形关系可得两束光线的光程差为$2d$,使用凸透镜半径$R$和干涉点到圆心距离$r$后可表示为$\frac{r^2}{R}$,结合反射相变可得如下干涉条件。
\begin{itembox}[l]{牛顿环干涉条件}
    \begin{equation*}
        2d=\frac{r^2}{R}=
        \begin{cases}
            2m\cdot\frac{\lambda}{2}&\implies
            \textrm{0相变明线/}\pi\textrm{相变暗线}\\
            (2m+1)\cdot\frac{\lambda}{2}&\implies
            \textrm{0相变暗线/}\pi\textrm{相变明线}
        \end{cases}
    \end{equation*}
\end{itembox}


    % chapter 4

\chapter{电磁}

% chapter 4 section 1

\section{电场}

\subsection{电场}

\subsubsection{电荷}

物体的带电量称为电荷,日文中也可用電気量一词。其单位为库伦(C),且带电体间时常有满足库仑定律\footnote{诱电率$\varepsilon=\frac{1}{4k\pi}$}的作用力存在。
\begin{itembox}[l]{库仑定律}
    \begin{equation*}
        F=k\frac{Qq}{r^2}
    \end{equation*}
    \begin{itemize}
        \item $k=9\times10^9N\cdot m^2\cdot C^{-2}$
        \item 同性相斥,异性相吸
    \end{itemize}
\end{itembox}

\subsubsection{电场}

为了更方便得解释诸如重力、万有引力、电磁力等非接触力的作用方式,人们引入了“场” 的概念,并认为是场传递了各种各样的物理信息。因此在空间中传递上述库仑力的场即为\underline{电场},用单位电荷的受力定义。

\begin{figure}[p!]
    \centering
    \begin{minipage}[t]{0.48\textwidth}
        \centering
        \begin{tikzpicture}[scale=0.5]
            \node[draw, circle] at (0,0) {$+$};
            \foreach \x in {-3,...,3} \foreach \y in {-3,...,3} {
                \pgfmathparse{and(equal(\x,0),equal(\y,0))}
                \ifnum\pgfmathresult=1
                \else
                \draw[-stealth, opacity={1/sqrt((\x)^2+(\y)^2)}] (\x,\y) -- ++ (
                    {0.5*\x/sqrt((\x)^2+(\y)^2)},
                    {0.5*\y/sqrt((\x)^2+(\y)^2)}
                );
                \fi
            }
        \end{tikzpicture}
        \caption{单正点电荷电场图示}
    \end{minipage}
    \begin{minipage}[t]{0.48\textwidth}
        \centering
        \begin{tikzpicture}[scale=0.5]
            \node[draw, circle] at (0,0) {$-$};
            \foreach \x in {-3,...,3} \foreach \y in {-3,...,3} {
                \pgfmathparse{and(equal(\x,0),equal(\y,0))}
                \ifnum\pgfmathresult=1
                \else
                \draw[-stealth, opacity={1/sqrt((\x)^2+(\y)^2)}] ($(\x,\y)+(
                    {0.5*\x/sqrt((\x)^2+(\y)^2)},
                    {0.5*\y/sqrt((\x)^2+(\y)^2)}
                )$) -- (\x,\y);
                \fi
            }
        \end{tikzpicture}
        \caption{单负点电荷电场图示}
    \end{minipage}
\end{figure}

\begin{figure}[p!]
    \centering
    \begin{tikzpicture}[scale=0.5]
        \node[draw, circle] at (-4,0) {$+$};
        \node[draw, circle] at (4,0) {$-$};
        \foreach \x in {-8,...,8} \foreach \y in {-4,...,4} {
            \pgfmathparse{or(
                and(equal(\x,-4),equal(\y,0)),
                and(equal(\x,4),equal(\y,0))
            )}
            \ifnum\pgfmathresult=1
            \else
            \draw[-stealth] (\x,\y) -- ++ (
                {0.3*((\x+4)/sqrt((\x+4)^2+(\y)^2)-(\x-4)/sqrt((\x-4)^2+(\y)^2))},
                {0.3*(\y/sqrt((\x+4)^2+(\y)^2)-\y/sqrt((\x-4)^2+(\y)^2))}
            );
            \fi
        }
    \end{tikzpicture}
    \caption{正-负点电荷电场图示}
\end{figure}

\begin{figure}[p!]
    \centering
    \begin{tikzpicture}[scale=0.5]
        \node[draw, circle] at (-4,0) {$+$};
        \node[draw, circle] at (4,0) {$+$};
        \foreach \x in {-8,...,8} \foreach \y in {-4,...,4} {
            \pgfmathparse{or(
                and(equal(\x,-4),equal(\y,0)),
                and(equal(\x,4),equal(\y,0))
            )}
            \ifnum\pgfmathresult=1
            \else
            \draw[-stealth] (\x,\y) -- ++ (
                {0.3*((\x+4)/sqrt((\x+4)^2+(\y)^2)+(\x-4)/sqrt((\x-4)^2+(\y)^2))},
                {0.3*(\y/sqrt((\x+4)^2+(\y)^2)+\y/sqrt((\x-4)^2+(\y)^2))}
            );
            \fi
        }
    \end{tikzpicture}
    \caption{正-正点电荷电场图示}
\end{figure}

\begin{itembox}[l]{电场}
    \begin{equation*}
        \vec{E}=\frac{\vec{F}}{q}
        \iff
        \vec{F}=q\vec{E}
    \end{equation*}
    \begin{itemize}
        \item 单位:N/C
        \item 点电荷周围的电场:$E=k\frac{Q}{r^2}$
    \end{itemize}
\end{itembox}
根据定义可知电场也是一个矢量,所以在实际运用过程中也满足矢量的合成与分解。

\subsection{电势}

\subsubsection{电势能与电势}

观察库仑力形式可发现其与万有引力十分相像,因此不难推测库仑力也是保存力。因而也有其对应的势能,即\underline{电势能},日文为静電気力による位置エネルギー。

此外,为了描述物理现象与计算使用方便,人们也引入了电场中单位电荷的电势能:电势(日文为電位)这个物理量,单位为伏特(V)。因此,电势能与电势之间满足如下关系。
\begin{itembox}[l]{电势能与电势}
    \begin{equation*}
        U=qV
    \end{equation*}
\end{itembox}

\subsubsection{典型电场中的电势}

\paragraph{匀强电场的电势}日文为一様電場,即大小、方向均衡定的电场。
\begin{figure}[ht!]
    \centering
    \begin{tikzpicture}
        \foreach \y in {0,1,2} {\draw[opacity=0.3, -latex] (4,\y) -- (0,\y);}
        \fill (0.5,1) circle (2pt) node[below] {$O$};
        \fill (1.5,1) circle (2pt) node[below] {$B$};
        \fill (3.5,1) circle (2pt) node[below] {$A$};
        \draw[<->] (1.5,1.3) -- node[fill=white] {$d$} (3.5,1.3);
    \end{tikzpicture}
    \caption{匀强电场电势}
\end{figure}
那么,匀强电场中倘若某两点间距为$d$,期间将会存在
\begin{equation*}
    V=\frac{Eq\times d}{q}=Ed
\end{equation*}
大小的电势,并称其为该两点间的\underline{电势差}。

\paragraph{点电荷电场的电势}

规定离场源电荷无限远处为电势基准,则可得点电荷周围电势公式。
\begin{itembox}[l]{点电荷电势}
    \begin{equation*}
        U=k\frac{kQ}{r}
    \end{equation*}
\end{itembox}

\subsubsection{等电势面}

类比于地图上描述地势分布的等高线,等电势线/面也常常用于描述电势的分布,且具有如下性质。
\begin{itemize}
    \item 与电场线垂直
    \item 等电位面密$\iff$电场线密$\iff$电场强
\end{itemize}
\begin{figure}[ht!]
    \centering
    \begin{tikzpicture}[scale=0.6]
        \foreach \r in {1,...,3} {\draw[opacity={1/\r}] (0,0) circle (\r);}
        \foreach \a in {1,...,12} {\draw[thick, -latex] (0,0) -- ({\a*30}:3.5);}
        \node[circle, fill=white] at (0,0) {$+$};
    \end{tikzpicture}
    \caption{等电位面}
\end{figure}

\subsubsection{总结}

鉴于本节内容较多且不易梳理清晰,特此简做总结。
\begin{figure}[ht!]
    \centering
    \renewcommand\arraystretch{1.5}
    \begin{tabular}{c|c|c|c|c}
        \hline
        &电场E&库仑力F&电势能U&电势V\\\hline
        重力场&$g$&$mg$&$mgh$&$gh$\\\hline
        点电荷&$\frac{kQ}{r^2}$&$\frac{kQq}{r^2}$&$\frac{kQq}{r}$&$\frac{kQ}{r}$\\\hline
    \end{tabular}
    \caption{电场电势总结}
\end{figure}

\subsection{电容器}

\textbackslash\textbackslash TODO

% chapter 4 section 2

\section{电流}

\subsection{欧姆定律}

本节内容较为基础,在此只逐条列举重要知识点。

\subsubsection{电流}

\begin{itemize}
    \item 描述:由电荷的定向移动形成
    \item 方向:正电荷的流动方向,或者负电荷流动的反方向
    \item 大小:单位时间内流过导体截面的电量
    \item 定义式1(宏观):
    \begin{equation*}
        I=\frac{\Delta Q}{\Delta t}
    \end{equation*}
    \item 定义式2(微观)\footnote{n为单位体积的电子密度,v为电子移动速度,S为导体截面积}:
    \begin{equation*}
        I=neSv
    \end{equation*}
    \item 单位:安培(A)
\end{itemize}

\subsubsection{欧姆定律}

\begin{equation*}
    V=IR
\end{equation*}
\begin{itemize}
    \item 描述:流过导体的电流与加在其两端电压的大小成正比
    \item 电阻R:日文为電気抵抗
    \begin{figure}[ht!]
        \centering
        \begin{circuitikz}[european]
            \draw (0,0) to[R=$3k\Omega$,o-o] (2,0);
        \end{circuitikz}
    \end{figure}
    \begin{itemize}
        \item 单位:欧姆($\Omega$)
        \item 定义式:$R=\rho\frac{l}{S}$
    \end{itemize}
\end{itemize}

\subsubsection{焦耳热}

\begin{itemize}
    \item 本质:电场力对电子的做功
    \item 公式:
    \begin{equation*}
        Q=VIt=I^2Rt=\frac{V^2}{R}t
    \end{equation*}
    \item (电)功率:
    \begin{equation*}
        W=VI=I^2R=\frac{V^2}{R}
    \end{equation*}
\end{itemize}

\subsection{直流电路}

\paragraph{基尔霍夫定律}适用于任何闭合电路,是解决复杂电路问题的重要手段。
\begin{itemize}
    \item 电流表述:回路中任意一个节点都满足$I_\textrm{流入}=I_\textrm{流出}$
    \item 电压表述:任意一个闭合回路都满足$V_\textrm{下降}=V_\textrm{上升}$
\end{itemize}
\begin{figure}[ht!]
    \centering
    \begin{circuitikz}[european]
        \draw (0,0)
        to[battery1=$V_1$, invert] (2,0)
        to[R=$R_1$] (4,0)
        to[short, i=$I_1$] (6,0)
        to[short] (6,4)
        to[short, i=$I_3$] (3,4)
        to[R=$R_3$] (0,4)
        to[short] (0,0);
        \draw (0,2)
        to[battery1=$V_2$, invert] (2,2)
        to[R=$R_2$] (4,2)
        to[short, i=$I_2$] (6,2);

        \draw[color=gray, opacity=0.5, thick, -latex] (0.1,-0.1) -- (6.2,-0.1) -- (6.2,4.1) -- (-0.2,4.1) -- (-0.2,-0.1);
        \draw[color=gray, opacity=0.5, thick, -latex] (0.4,2.1) -- (5.8,2.1) -- (5.8,3.9) -- (0.2,3.9) -- (0.2,2.1);
    \end{circuitikz}
    \caption{基尔霍夫定律}
\end{figure}
对于图示的两条闭合回路,则可以列如下等式。
\begin{itemize}
    \item $+V_1-R_1I_1-R_3I_3=0$
    \item $+V_2-R_2I_2-R_3I_3=0$
    \item $I_1+I_2=I_3$
\end{itemize}

\paragraph{电池内阻}在一般问题中题目常常将电池视作理想的供能元件,但事实上电池内部也存在电阻,简称为电池内阻。同时,当我们使用电压表测电压时,即便接在电池两端测量,得到的结果也是外路的数值,即无法直接测定电池的实际起电力。
\begin{figure}[ht!]
    \centering
    \begin{circuitikz}[european]
        \draw (0,0)
        to[battery1=$E$, invert]  (2,0)
        to[R=$r$] (4,0)
        to[short] (4,4)
        to[R=$R$] (0,4)
        to[short] (0,0);
        \draw (0,2)
        to[voltmeter, *-*] (4,2);
        \fill[fill=gray, opacity=0.3] (-0.2,-0.5) rectangle (4.2,1.2);
    \end{circuitikz}
    \caption{电池内阻}
\end{figure}
对上述简化电路列基尔霍夫方程可得
\begin{gather*}
    E=rI+V\\
    V(I)=-rI+E
\end{gather*}
因此,只要确定了该电路的$V-I$直线后即可通过纵截距间接求得电池实际的起电力。

\paragraph{惠斯通电桥}日文为ホイートストンブリッジ,是一个测量未知电阻阻值的结构。其中已知$R_1$与$R_2$,$R_3$是一个阻值可调的可变电阻,当图中电流表示数为0时,则$R_4$值可求。
\begin{figure}[ht!]
    \centering
    \begin{circuitikz}[european]
        \draw (0,0)
        to[short] (0,3)
        to[R=$R_1$] (3,3)
        to[vR=$R_3$] (6,3)
        to[short] (6,0)
        to[battery1, invert]  (0,0);
        \draw (0,1.5)
        to[R=$R_2$] (3,1.5)
        to[R=$R_4$] (6,1.5);
        \draw (3,1.5) to[ammeter, *-*] (3,3);
    \end{circuitikz}
    \caption{惠斯通电桥}
\end{figure}

当电流表示数为0,即其两端没有电势差时,$R_1$和$R_2$消耗的电压相同、$R_3$和$R_4$消耗的电压相同,因此可列方程
\begin{equation*}
    \frac{R_1}{R_2}=\frac{R_3}{R_4}
\end{equation*}

\paragraph{非线性电阻}一般用电器的电阻并非恒定,而是会随着诸如温度等物理条件变化,我们称其为非线性电阻。
\begin{figure}[ht!]
    \centering
    \begin{minipage}{0.48\textwidth}
        \centering
        \begin{circuitikz}[european]
            \draw (0,0)
            to[short, i=$I$] (0,1.5)
            to[lamp=$V$] (2,1.5)
            to[R=$1\Omega$] (4,1.5)
            to[short] (4,0)
            to[battery1=$4V$] (0,0);
        \end{circuitikz}
    \end{minipage}
    \begin{minipage}{0.48\textwidth}
        \centering
        \begin{tikzpicture}[scale=0.6]
            \draw[color=gray!30, step=1] (0,0) grid (4,4);
            \draw[-latex] (0,0) -- (4.5,0) node[right] {$V$};
            \draw[-latex] (0,0) -- (0,4.5) node[above] {$I$};
            \draw[thick] (0,0) .. controls (0.7,3.3) .. (4,4);
            \draw[thick] (0,4) -- (4,0);
        \end{tikzpicture}
    \end{minipage}
    \caption{非线性电阻}
\end{figure}
关于整个回路列基尔霍夫方程可得:$V+I=4$,也就是说回路中的灯泡同时满足这个等式和它自己的变化规律。数形结合,两条线的交点即为唯一符合条件的电压值与电流值。

% chapter 4 section 3

\section{磁场}

\subsection{磁场}

\subsubsection{电与磁}

磁的体系与电十分类似,几乎所有磁的概念\footnote{磁场中不存在独立的单极磁荷}都可以在电中找到对应的内容。
\begin{figure}[ht!]
    \centering
    \renewcommand\arraystretch{1.2}
    \begin{tabular}{c|c|c}
        \hline
        &电&磁\\\hline
        基本单位&电荷C&磁荷Wb\\\hline
        场力&电场力(库仑力)&磁场力\\\hline
        场常数&电容率$\varepsilon$(誘電率)&磁导率$\mu$(透磁率)\\\hline
    \end{tabular}
    \caption{电与磁对比}
\end{figure}

因此,磁场的直觉定义即为
\begin{equation*}
    \vec{H}=\frac{\vec{F}}{m}\quad(N/Wb)
\end{equation*}
同样也存在描述磁场走势、大小等信息的磁场线,并具有以下性质。
\begin{itemize}
    \item N极出,S极入
    \item 磁场线密$\iff$磁场强
\end{itemize}
\begin{figure}[ht!]
    \centering
    \begin{tikzpicture}[scale=0.45]
        \foreach \x in {-8,...,8} \foreach \y in {-4,...,4} {
            \pgfmathparse{or(
                and(equal(\x,-4),equal(\y,0)),
                and(equal(\x,4),equal(\y,0))
            )}
            \ifnum\pgfmathresult=1
            \else
            \draw[-stealth] (\x,\y) -- ++ (
                {0.3*((\x+4)/sqrt((\x+4)^2+(\y)^2)-(\x-4)/sqrt((\x-4)^2+(\y)^2))},
                {0.3*(\y/sqrt((\x+4)^2+(\y)^2)-\y/sqrt((\x-4)^2+(\y)^2))}
            );
            \fi
        }
        \filldraw[color=black, fill=white] (-4,-0.5) -- node[right] {N} (-4,0.5) -- (4,0.5) -- node[left] {S} (4,-0.5) -- cycle;
    \end{tikzpicture}
    \caption{磁场图示}
\end{figure}

\subsubsection{电生磁}

19世纪初叶,毕奥-萨伐尔定律\footnote{
    Biot-Savart Law
    \begin{equation*}
        \mathbf{B}=\frac{\mu_0}{4\pi}\int\mathbf{J}\times\frac{\mathbf{r}-\mathbf{l}}{|\mathbf{r}-\mathbf{l}|^3}d^3l
    \end{equation*}
}、安培定律\footnote{
    Ampère's circuital law
    \begin{equation*}
        \oint_{\partial S}\mathbf{B}\cdot d\mathbf{l}=\mu_0\int_S\mathbf{J}\cdot d\mathbf{S}
    \end{equation*}
}等揭示了电流的磁作用。下图是三种十分具有代表性的模型,其磁场方向可以由右手螺旋定则判断。
\begin{figure}[ht!]
    \centering
    \begin{minipage}[t]{0.48\textwidth}
        \centering
        \begin{tikzpicture}
            \foreach \r in {0.5,1,1.5} {
                \draw[-latex] (\r,0) arc (0:180:{\r} and {\r*0.3});
                \draw[-latex] ({-\r},0) arc (-180:0:{\r} and {\r*0.3});
            }
            \draw[color=white, line width=5pt] (0,0) -- (0,1);
            \draw[thick, -latex] (0,-1) -- (0,1) node[right] {$I$};
            \node at (2,0) {$H$};
        \end{tikzpicture}
        \caption{直线电流磁场}
    \end{minipage}
    \begin{minipage}[t]{0.48\textwidth}
        \centering
        \begin{tikzpicture}
            \draw[thick, -latex] (0,-1) arc (-90:90:0.3 and 1);
            \draw[color=white, line width=5pt] (0,0) -- (1,0);
            \draw[-latex] (-1,0) -- (1,0) node[above] {$H$};
            \draw[thick, -latex] (0,1) arc (90:270:0.3 and 1) node[below] {$I$};
        \end{tikzpicture}
        \caption{环形电流磁场}
    \end{minipage}
    \begin{minipage}{\textwidth}
        \centering
        \begin{tikzpicture}
            \clip (-0.5,-1) rectangle (4,1);
            \draw[thick, midarrow] (0,-1) -- node[left] {$I$} (0,0);
            \draw[thick] (0,0) arc (180:135:0.2 and 0.4);
            \draw[color=white, line width=5pt] (-0.5,0) -- (0.2,0);
            \draw[-latex] (-0.5,0) -- (3.5,0) node[above] {$H$};
            \foreach \n in {0,...,5} {
                \draw[color=white, line width=5pt] ($(0.2+\n*0.4,0)+(135:0.2 and 0.4)$) arc (135:-135:0.2 and 0.4);
                \draw[thick, midarrow] ($(0.2+\n*0.4,0)+(135:0.2 and 0.4)$) arc (135:-135:0.2 and 0.4);
            }
            \draw[color=white, line width=5pt] (2.8,0.2) -- (2.8,-1);
            \draw[thick] ($(2.6,0)+(135:0.2 and 0.4)$) arc (135:0:0.2 and 0.4);
            \draw[thick,midarrow] (2.8,0) -- (2.8,-1);
        \end{tikzpicture}
        \caption{线圈周围磁场}
    \end{minipage}    
\end{figure}
\begin{itembox}[l]{电流产生的磁场}
    \begin{itemize}
        \item 直线电流
        \begin{equation*}
            H=\frac{I}{2\pi r}
        \end{equation*}
        \item 环形电流(中心处)
        \begin{equation*}
            H=\frac{I}{2r}
        \end{equation*}
        \item 线圈
        \begin{equation*}
            H=nl\quad(n:\textrm{线圈匝数})
        \end{equation*}
    \end{itemize}
\end{itembox}

\subsection{安培力}

\subsubsection{平行电流间的力}

\begin{figure}[ht!]
    \centering
    \begin{tikzpicture}
        \draw[-latex] (1,1.5) arc (0:180:1 and 0.3);
        \draw[-latex] (2,1.5) arc (0:180:2 and 0.6);
        \draw[color=white, line width=5pt] (0,0) -- (0,3);
        \draw[thick, -latex] (0,0) -- (0,3) node[above] {$I_1$};
        \draw[-latex] (-1,1.5) arc (180:360:1 and 0.3);
        \draw[-latex] (-2,1.5) arc (180:360:2 and 0.6);
        \draw[thick, -latex] (2,0) -- (2,3) node[above] {$I_2$};
    \end{tikzpicture}
    \caption{平行电流间的力}
\end{figure}
倘若空间中存在两根无限长直导线,分别流过$I_1$与$I_2$大小的电流,那么单位长度的导线上会受到
\begin{equation*}
    f=\frac{\mu I_1I_2}{2\pi r}
\end{equation*}
大小的力。并且当电流同向时表现为吸引力,逆向时表现为排斥力。以$I_1$对$I_2$的视角审视上式则可理解为
\begin{equation*}
    f=\frac{\mu I_1}{2\pi r}I_2=\mu H_1\cdot I_2
\end{equation*}
即$I_2$受到的是来自于$I_1$所产生的磁场导致的力。因此,不难理解通电导线会受到来自于磁场的作用力。

\subsubsection{安培力}

\begin{figure}[ht!]
    \centering
    \begin{tikzpicture}[scale=0.6]
        \clip (-5,-4) rectangle (5,4);
        \begin{scope}[rotate=30]
            \filldraw[color=black, fill=gray, fill opacity=0.3] (-8,-2.5) rectangle (-3,2.5);
            \filldraw[color=black, fill=gray, fill opacity=0.5] (3,-2.5) rectangle (8,2.5);
            \foreach \y in {-2,...,2} {
                \draw[color=gray, -latex] (-3,\y) -- (3,\y);
            }
            \draw[thick, -latex] (-1.5,1.5) -- (1.5,-1.5);
            \drawangle{(1,-1)}{(0,0)}{(1,0)};
        \end{scope}
    \end{tikzpicture}
    \caption{安培力}
\end{figure}
上述实验表明将通电导线(电流)置于磁场中,通电导线会受到来自于磁场的力,即\underline{安培力}\footnote{安培力:$\vec{F}=I\vec{l}\times\vec{B}$},大小由$F=\mu HIl$给出。其中$\mu H$的部分也具有实际的物理意义:\underline{磁通量密度}\footnote{可以理解为由环境增强后的磁场。比如向通电线圈内放入铁芯会增强原磁场。},日文为磁束密度,用字母$B$表示。
\begin{itembox}[l]{安培力}
    \begin{equation*}
        F=IBl\sin\theta
    \end{equation*}
    \begin{itemize}
        \item $\theta$:电流方向与力的方向的夹角
        \item 左手定则
        \begin{itemize}
            \item 磁场穿过手心
            \item 四指与电流同向
            \item 拇指方向为安培力方向
        \end{itemize}
        \item 使用场景:“左力右电”
        \begin{itemize}
            \item 左手定则:电磁力的判断
            \item 右手(螺旋)定则:电磁现象的判断
        \end{itemize}
    \end{itemize}
\end{itembox}

\subsection{洛伦兹力}

\subsubsection{洛伦兹力}

由本章第二节的知识可知,电流是由带电载流子(キャリア)的移动而形成的。因此,重新审视安培力,我们可以认定带点的载流子,比如电子等也会受到磁场的作用。这个力便是洛伦兹力\footnote{洛伦兹力:$\vec{F}=q\vec{v}\times\vec{B}$},日文为ローレンツ力。但使用时需注意\underline{电流方向为正电荷的移动方向}。
\begin{itembox}[l]{洛伦兹力}
    \begin{equation*}
        F=qvB\sin\theta
    \end{equation*}
\end{itembox}

\subsubsection{带电粒子在磁场中的运动}

根据洛伦兹力的性质可知,在粒子运动过程中力与运动方向始终垂直,因此洛伦兹力不做功。结合动能定理可知,粒子运动速度不发生改变,所以作用在其身上的洛伦兹力也将会是恒力。基于数学推导可知,当粒子水平入射进入磁场后会做等速圆周运动。
\begin{figure}[ht!]
    \centering
    \begin{tikzpicture}
        \foreach \x in {1,3,5} \foreach \z in {1,3,5} {
            \draw[->, color=gray!80] (\x,-1,\z) -- (\x,0,\z);
        }
        \begin{scope}[xzplane=0]
            \fill[fill=gray!30] (0,0) rectangle (6,6);
            \draw[thick, -latex] (4,3) arc (0:-320:1.5);
        \end{scope}
        \foreach \x in {1,3,5} \foreach \z in {1,3,5} {
            \draw[->, color=gray!80] (\x,0,\z) -- (\x,1.5,\z);
        }
        \fill (4,0,3) circle (2pt) node[below] {$e^-$};
        \draw[thick, -latex] (4,0,3) -- (4,0,2) node[right] {$\vec{v}$};
        \draw[thick, -latex] (4,0,3) -- (2.5,0,3) node[below] {$\vec{F}$};
    \end{tikzpicture}
    \caption{粒子在磁场中的平面运动}
\end{figure}

此时洛伦兹力充当向心力,即
\begin{equation*}
    qvB=m\frac{v^2}{r}
\end{equation*}
则运动半径、周期可求。
\begin{itembox}[l]{粒子在磁场中运动结论}
    \begin{itemize}
        \item 半径:$r=\frac{mv}{qB}$
        \item 周期:$T=\frac{2\pi m}{qB}$
    \end{itemize}
\end{itembox}

当粒子斜向入射进入磁场时,除了垂直于磁场平面上的圆周运动以外,还会在平行于磁场的方向上做匀速直线运动,因此其轨迹将会为圆柱螺线。
\begin{figure}[ht!]
    \centering
    \begin{tikzpicture}
        \foreach \x in {-1,0,1} \foreach \z in {-1,0,1} {
            \draw[->, color=gray!80] (\x,-1,\z) -- (\x,0,\z);
        }
        \begin{scope}[xzplane=0]
            \fill[fill=gray!30] (-1.5,-1.5) rectangle (1.5,1.5);
            \draw[dashed] (0,0) circle (1);
        \end{scope}
        \foreach \x in {-1,0,1} \foreach \z in {-1,0,1} {
            \draw[->, color=gray!80] (\x,0,\z) -- (\x,4,\z);
        }
        \draw[thick, -latex] plot[variable=\t,domain=0:5.5, samples=100,smooth] ({cos((pi*\t) r)},{0.5*\t},{sin((pi*\t) r)});
        \draw[dashed] (1,0,0) -- (1,0,1);
        \draw[thick, -latex] (1,0,0) -- ++ (0,0.5,1);
        \draw[<->] (1.2,0,0) -- node[right] {$v_{z}T$} (1.2,1,0);
    \end{tikzpicture}
    \caption{粒子在磁场中的空间运动}
\end{figure}

此外,体现电磁场共同作用的\underline{霍尔效应}也十分具有代表性。
\begin{figure}[ht!]
    \centering
    \begin{tikzpicture}
        \begin{scope}[yzplane=0] % left
            \filldraw[color=black, dashed, fill=gray, fill opacity=0.3]  (0,0) rectangle (-0.5,3);
            \foreach \z in {0.5,1.5,2.5} {
                \node at (-0.25,\z) {$\ominus$};
            }
        \end{scope}
        \begin{scope}[yzplane=2] % right
            \filldraw[color=black, fill=gray, fill opacity=0.5] (0,0) rectangle (-0.5,3);
            \foreach \z in {0.5,1.5,2.5} {
                \node at (-0.25,\z) {$\oplus$};
            }
        \end{scope}
        \begin{scope}[xzplane=0] % up
            \draw (0,0) rectangle (2,3);
        \end{scope}
        \begin{scope}[xyplane=0] % back
            \draw[dashed] (0,0) rectangle (2,-0.5);
        \end{scope}
        \begin{scope}[xyplane=3] % front
            \draw (0,0) rectangle (2,-0.5);
        \end{scope}
        \draw[thick] (1,-0.25,-1) -- (1,-0.25,0);
        \draw[thick, dashed] (1,-0.25,0) -- (1,-0.25,3);
        \draw[thick, -latex] (1,-0.25,3) -- (1,-0.25,5) node[left] {$I$};
        \draw[thick, -latex] (1,0,1.5) -- (1,1,1.5) node[above] {$B$};
        \node at (3.5,-0.25,5) {$qvB=Eq\implies qB=\frac{V}{vd}$};
    \end{tikzpicture}
    \caption{霍尔效应}
\end{figure}

% chapter 4 section 4

\section{电磁感应}

\subsection{电磁感应}

\subsubsection{楞次定律}

\begin{figure}[ht!]
    \centering
    \begin{tikzpicture}[scale=0.5]
        \clip (-3.5,-3) rectangle (3,3);
        \draw (-4,-1) rectangle (-2.5,1);
        \node at (-3,0) {N};
        \foreach \y in {-1,0,1} \draw[-latex] (0,\y) -- (2,\y);
        \filldraw[thick, draw=black, fill=white] (-80:1 and 2) arc (-80:260:1 and 2);
        \draw[thick] (-80:1 and 2) -- ++ (0,-1);
        \draw[thick] (260:1 and 2) -- ++ (0,-0.8);
        \foreach \y in {-1,0,1} \draw (-2,\y) -- (0,\y);
    \end{tikzpicture}
    \caption{楞次定律}
\end{figure}
实验表明磁场的变化会激发出相应的电的效果,即电磁感应现象,日文为電磁誘導。其中称激发出来的电流为感应电流(誘導電流),称与之对应的起电力为感应电势(誘導起電力)。描述其感应电流方向的定律为楞次定律,日文为レンツの法則。
\begin{itembox}[l]{楞次定律}
    \centering
    变化的磁场会产生妨碍该变化的感应电流/电势(增反减同)
\end{itembox}

\subsubsection{法拉第电磁感应定律}

日文为ファラデーの電磁誘導の法則,其定量地描述了磁生电的效果。为此我们需要率先明确磁通量的概念。

磁通量,日文为磁束,是描述给定面上磁场大小的量,一般用字母$\Phi$表示。应注意只有与给定面垂直的部分才会被计为磁通量。
\begin{equation*}
    \Phi=B\cdot S
\end{equation*}

至此,磁生电的现象则可由磁通量随时间的变化描述,即法拉第电磁感应定律。
\begin{equation*}
    V=-\frac{\Delta\Phi}{\Delta t}
\end{equation*}

\subsubsection{运动导体的感应电势}

\paragraph{一般模型}实际题目中常常很难直接使用法拉第电磁感应定律,而是更多使用其在特定模型下的形式。题目中常见的模型如下,空间中布满匀强磁场,平行金属导轨上有一根可移动的金属棒,让金属棒以一定速度运动,探究此时的感应电势。
\begin{figure}[ht!]
    \centering
    \begin{tikzpicture}
        \foreach \x in {0.5,1.5,2.5,3.5} \foreach \z in {0.5,1.5,2.5}
            \draw[-latex,color=gray,opacity=0.5] (\x,-1,\z) -- (\x,1,\z);
        \draw (4,0,0) -- (0,0,0) -- (0,0,3) -- (4,0,3);
        \draw[ultra thick] (2,0,-0.2) -- (2,0,3.2);
        \draw[-latex] (2.5,0,1.5) -- (3.5,0,1.5) node[right] {$v$};
    \end{tikzpicture}
    \caption{电磁感应水平模型}
\end{figure}
根据法拉第电磁感应定律可得感应电势大小为
\begin{equation*}
    |V|=\frac{\Delta\Phi}{\Delta t}=\frac{BS}{t}=\frac{B\Delta xl}{\Delta t}=Bvl
\end{equation*}
其方向可由楞次定律或者右手定则判断。
\begin{itemize}
    \item 磁场穿过手心
    \item 拇指与运动方向同向
    \item 四指为电流方向
\end{itemize}

\paragraph{斜面模型}时而题目中会出现斜向的平行导轨。
\begin{figure}[ht!]
    \centering
    \begin{tikzpicture}
        \foreach \x in {0.5,1.5,2.5,3.5} \foreach \z in {0.5,1.5,2.5}
            \draw[-latex,color=gray,opacity=0.5] (\x,-1,\z) -- (\x,1,\z);
        \draw (4,1,0) -- (1,0,0) -- (0,0,0) -- (0,0,3) -- (1,0,3) -- (4,1,3);
        \draw[dashed] (1,0,3) -- (4,0,3);
        \drawangle{(4,0,3)}{(1,0,3)}{(4,1,3)};
        \draw[ultra thick] (2,1/3,-0.2) -- (2,1/3,3.2);
        \draw[-latex] (2.5,0.5,1.5) -- (3.5,5/6,1.5) node[right] {$v$};
    \end{tikzpicture}
    \caption{电磁感应斜面模型}
\end{figure}
此时实际与金属棒运动方向垂直的磁通量密度为$B\cos\theta$,根据磁通量的定义可得
\begin{equation*}
    V=Bvl\cos\theta
\end{equation*}

\paragraph{圆环模型}此外题目中偶尔也会出现稍复杂一些的圆环模型。
\begin{figure}[ht!]
    \centering
    \begin{tikzpicture}
        \foreach \x in {-1,0,1} \foreach \y in {-0.5,0.5}
            \draw[-latex,color=gray,opacity=0.5] (\x,{\y-0.3}) -- (\x,{\y+0.3});
        \draw (0,0) circle (2 and 1);
        \draw[ultra thick, color=gray!30] (0,0) --(2.1,0);
        \draw[ultra thick] (0,0) -- (45:2.1 and 1.05);
        \draw[-latex] (2.2,0) arc (0:45:2.2 and 1.1) node[above] {$\omega$};
        \fill[color=gray,opacity=0.3] (0,0) -- (2,0) arc (0:45:2 and 1) --cycle;
    \end{tikzpicture}
    \caption{电磁感应圆环模型}
\end{figure}
根据扇形公式可知,在$\Delta t$的时间内金属棒会扫过$\Delta S=\frac12\omega\Delta tl^2$的面积,结合法拉第电磁感应定律可得
\begin{equation*}
    V=\frac12\omega Bl^2
\end{equation*}

\subsubsection{自感与互感}

\textbackslash\textbackslash TODO

\subsection{交流电}

\textbackslash\textbackslash TODO


    % chapter 5

\chapter{原子}

% chapter 5 section 1

\section{波粒二象性}
\label{sec:5.1}

\subsection{光的粒子性}

自有关光的研究展开以来,其为波还是粒子的争论就未曾停歇。早在17世纪,就分为了两派。虽双方都没有决定性的证据,但以牛顿为首的粒子派在其领头人的影响力下占据了主导地位。随后在19世纪,随着电磁理论的发展,法拉第、麦克斯韦、赫兹等人先后从理论以及实验的角度证明了光为一种电磁波。尽管此时波动理论看似大获全胜,然而却始终无法解释“光电效应”这一试验的结果。

\subsubsection{光电效应}

\begin{figure}[ht!]
    \centering
    \begin{tikzpicture}
        \draw (0,0) rectangle node {metal} (3,0.5);
        \draw[-latex] (0,1.5) node[above] {light} -- (1,0.5);
        \draw[-latex] (2,0.5) -- (3,1.5) node[above] {$e^-$};
    \end{tikzpicture}
    \caption{光电效应图示}
\end{figure}
日文为光電効果,指金属在受到光的照射后释放出电子的现象,并特称这种电子为\underline{光电子},称产生的电流为\underline{光电流}。一般使用光电管作为实验装置。
\begin{figure}[ht!]
    \centering
    \begin{tikzpicture}
        \draw (0,0) -- (0,-3) -- (0.5,-3);
        \draw (1.5,0) -- (2,0) -- (2,-3) -- (1.5,-3);
        \node at (1,-3) {V};
        \draw[very thick,-latex] (0,-2.5) -- node[left] {I} (0,-1);
        \filldraw[draw=black,fill=white] (0,0) circle (0.2) node {P};
        \draw (80:1.5) arc (80:-80:1.5);
        \draw[ultra thick] (45:1.5) node[above right] {C} arc (45:-45:1.5);

        \draw[dashed, -latex] (150:1.5) -- (30:1.5) -- (30:0.3);
        \draw[dashed, -latex] (-150:1.5) -- (-30:1.5) -- (-30:0.3);
    \end{tikzpicture}
    \caption{光电管}
\end{figure}
这个现象为海因里希·赫兹(德国)于1887年首次发现,菲利普·莱纳德(德国)用实验验证了其重要规律,最后由爱因斯坦(美国)提出了正确的理论机制。
\begin{itembox}[l]{光电效应结论}
    \begin{itemize}
        \item 光电效应发生与否只取决于照射光的频率。
        \item 金属释放出的电子的动能仅依存于照射光的频率,与其强度无关。
        \item 增大照射光的光强会使得金属释放出的电子增多。
    \end{itemize}
\end{itembox}
也就是说当照射光的频率小于某个数值时无论光强如何大光电效应都不会发生;相反只要频率大于该值,无论光强如何小光电效应都会发生。一般称这个阈值$\nu_0$为\underline{极限频率},日文为限界振動数。然而以上的结论却与既有的理论不符。
\begin{itemize}
    \item 常理认为金属内的自由电子接收到来自于的光波的能量,从而离开金属,而且光波的能量同时依存于其频率和振幅。因此,即便频率很小但只要振幅足够,电子理应也能够得到足够的能量离开金属。这点却与结论一不符。
    \item 同理光强越强电子也应该得到更多的能量,这与结论二不符。
    \item 更令人意外的是虽然光强没有增加,每个电子的能量但却增加了电子的数量,这使得结论三变得不符合直觉。
\end{itemize}

\subsubsection{光量子假说}

为了合理解释光电效应的实验结果,爱因斯坦破天荒地提出了光量子假说,即光虽然是一种电磁波具有波动的性质(当时不争的事实),但同时也会表现出粒子的性质。
\begin{itembox}[l]{光量子假说}
    \begin{itemize}
        \item 光也是粒子,即光子。光强由光子数决定。
        \item 光子所携带的能量与其频率成比例,即$E=h\nu$($h$为普朗克常数)。
        \item 光子所携带的动量与其波长成反比,即$p=\frac{h}{\lambda}$
    \end{itemize}
\end{itembox}
基于以上的两个假设便可以解释一切光电效应的结论。
\begin{itemize}
    \item 光是粒子,一个光子给一个电子传递能量,光子越多(光强越大)便可以给予更多的电子能量,因此释放出的电子也会越多。
    \item 光子的能量取决于其频率,其自身能量越大能够给予电子的也就越多,然而光强只改变了光子的数量,因此电子的动能只和光子的频率有关。
\end{itemize}
至于结论一,我们先需要明确一下自由电子离开金属的具体细节。尽管电子可以在金属中自由移动,但其仍然会收到来自于金属阳离子的库仑力,并对其有一定的束缚作用。因此为了离开金属,就必须先克服掉这个束缚。将克服束缚所需的能量称为\underline{逸出功},日文为仕事関数\footnote{中文也有功函数的说法。}。
\begin{itemize}
    \item 若是光子所携带的能量足够电子克服金属的束缚,那么电子便可以离开金属。相反,当一个光子无法提供足够的能量,无论整体光强再大也无法使金属释放任何电子。
    \item 因此被释放出的电子的动能即为克服束缚后光子能量的剩余部分。
\end{itemize}

\subsubsection{光电效应理论解释}

将上一节的语言描述整理为数式即为
\begin{equation*}
    h\nu=W+K_{max}
\end{equation*}
其中$h\nu$为光子的能量,$W$为逸出功,$K_{max}$为电子的最大动能。
\begin{figure}[ht!]
    \centering
    \begin{tikzpicture}
        \draw[-latex] (-0.2,0) -- (2.5,0) node[right] {$\nu$};
        \draw[-latex] (0,-1.5) -- (0,1.5) node[above] {$K_{max}$};
        \draw[dashed] (0,-1) node[left] {$-W$} -- (1,0);
        \draw (1,0) node[below right] {$\nu_0$} -- (2,1);
    \end{tikzpicture}
    \caption{光电效应函数图}
\end{figure}
从图中可以发现纵截距的绝对值为逸出功,横截距为极限频率,斜率为普朗克常数。

此外,上式中的$K_{max}$可以通过调整实验装置的方式求得。当阳极P的电压高于阴极C的电压时,电子会逆着电场方向流向阳极,产生光电流。逐渐调低阳极电压直至低于阴极电压,此时电子为了来到阳极,需要不断消耗自身动能克服电场力的阻碍。最终若是某一阳极电压$V_0$下不再存在光电流,即证明动能最大的电子也无法到达,即可通过$K_{max}=eV_0$的方式求得电子的最大动能。一般称这个电压为\underline{遏制电压},日文为阻止電圧。

\subsection{X射线}

X射线一般可认为是波长短于紫外线的电磁波,由于发现之初并没能认清其本质,故命名为X射线。其发生机制为用高能电子轰击金属,撞击过程中电子减速,损失的动能以光子的形式释放,从而形成X射线。将收集到的X射线按照波长和其强度的关系绘制可得如下关系图。其中突出的部分称为固有X射线,剩下的部分称为连续X射线。
\begin{figure}[ht!]
    \centering
    \begin{tikzpicture}[xscale=1]
        \draw[-latex] (-0.2,0) -- (4,0) node[right] {$\lambda$};
        \draw[-latex] (0,-0.2) -- (0,2) node[above] {intensity};

        \draw[domain=0.61:3] plot[smooth] (\x, {1/((\x-0.5)^3*exp(1/(\x-0.5))});
        \draw[color=white, line width=2pt] (1.2,0.1) -- (1.2,2);
        \draw[color=white, line width=2pt] (2,0.1) -- (2,2);

        \draw (1.16,0.75) -- (1.18,1.8); \draw (1.24,0.65) -- (1.22,1.8);
        \draw (1.96,0.15) -- (1.98,1.8); \draw (2.04,0.15) -- (2.02,1.8);
    \end{tikzpicture}
    \caption{X射线谱}
\end{figure}

\subsubsection{连续X射线}

当电子行进到金属的原子核附近时,会收到来自于原子核的引力而发生路径的偏转,在这个过程中电子丢失动能并转化为光子释放出去,从而形成了连续X射线。
\begin{itemize}
    \item 电子具有的初动能有上限,因此不会存在能量高于其初动能的X射线。
    \item 电子可以损失的能量连续,因此整体函数也是连续的。
\end{itemize}

\subsubsection{固有X射线}

当高能电子撞击到原子的核外电子使其发生跃迁时,会产生出能级跃迁所对应的能量的光子,这些时刻在图中的凸起处得以体现\footnote{本人编辑本文时未能找到解释固有射线强度高于连续射线强度的文章,在此根据个人理解简作说明。原子核所占空间远小于核外电子,因此核外电子好似内部原子核的屏障一般。如此一来高能电子会有更大的概率与核外电子相撞而产生固有射线,相反只有巧妙避开了核外电子的幸运儿才会在和原子核的作用下产生连续射线。所以按照光量子假说固有射线所对应的光子更多,即其强度就会更强。以上内容为个人解释,尚未有材料佐证,仅供参考以辅助理解。}。由于跃迁的能量取决于金属本身的原子结构,所以增大电子的加速电压虽然会使得图像整体左移,却不会改变凸起处的位置。

此外,X射线也是电磁波,和光一样也会表现出粒子性。其波动性可由布拉格反射验证,粒子性可由康普顿效应验证。

\subsection{物质波}

在上两节中,物理学家破天荒地为波引入了粒子性,从而合理解释了一些现象。在此基础上德布罗意(法国)认为既然波可以有粒子性,那么粒子同样也可以具有波动性,因而提出了物质波的概念。
\begin{equation*}
    \lambda=\frac{h}{mv}
\end{equation*}
而后的实验也证明了德布罗意的猜想,证实了物质的波粒二象性。比如我们可通过定义式求出由电压$V$加速后电子的物质波长。
\begin{equation*}
    eV=\frac12mv^2\to
    \lambda=\frac{h}{\sqrt{2meV}}
\end{equation*}

% chapter 5 section 2

\section{原子模型}
\label{sec:原子模型}

\subsection{电子的发现}

在空气稀薄的玻璃管两端施加高电压后,可在管内发现放电而产生的发光现象。进一步抽去管内气体后,发光现象消失,但可在玻璃管的正极侧发现亮斑。
\begin{figure}[ht!]
    \centering
    \begin{tikzpicture}[scale=0.6]
        \draw (0,0) rectangle (2,1);
        \draw (2,0.5) node[left] {$+$} -- (5,0.5) -- (5,2.5) -- (4,2.5);
        \draw (4,2.75) -- (4,2.25);
        \draw (0,0.5) node[right] {$-$} -- (-3,0.5) -- (-3,2.5) -- (-2,2.5);
        \draw (-2,2.75) -- (-2,2.25);
        \filldraw[rounded corners=8pt, fill=gray, color=black, fill opacity=0.2] (-2.5,2) rectangle (4.5,3);
    \end{tikzpicture}
    \caption{真空放电}
\end{figure}
这种现象就像是有什么物质从负极射出后击打到了正极侧一样,所以当时人们将其命名为阴极射线。并发现阴极射线具有如下的性质。
\begin{itemize}
    \item 在无外界影响的环境下其运动轨迹为直线。
    \item 在空间中施加电场或是磁场后其运动轨迹会发生偏移。
    \item 击打到物体上之后会有一定的压力作用。
    \item 上述性质与极板金属和玻璃管内气体无关。
\end{itemize}
因此人们断定阴极射线是实物粒子且带有负电,后命名为电子。

\subsection{有核模型}

\subsubsection{初期猜想}

1897年JJ汤姆孙(英国)发现了电子,并提出了原子的枣糕模型,即电子均匀地分布在带有正电的球中。尽管这种无核模型在某种程度上可以解释原子的稳定性的问题,但却无法说明原子质量的问题。1908年汤姆孙的学生卢瑟福(新西兰)通过其著名的$\alpha$粒子轰击金箔的实验\footnote{如果原子无核,那么质量远大于电子的$\alpha$粒子在穿过原子时理应只发生小幅度的偏移,虽然大部分实验现象皆是如此,但仍然存在极少的$\alpha$粒子发生了巨大的偏移,即原子内存在$\alpha$粒子也无法撼动的极重物质。}发现了原子核的存在,从而确立了有核模型的正确性。

然而,卢瑟福(新西兰)的模型仍非完美。倘若电子在围绕着原子核不停地转动,那么势必会形成环形电流,进而形成磁场。磁场的变化再次激发出电场,如此往复则会释放出电磁波,最终使得电子的能量在不断地减少。也就是说电子并非处于一个安定的状态,原子也非安定。对此尼尔斯波尔(丹麦)于1913年给出了解决方案。

\subsubsection{波尔模型}

\paragraph{波尔假说} 波尔认为原子核周围存在\underline{能级},核外电子只会处于各个能级上,并且是稳定的。同时,这些能级从物质波的角度可以理解为电子在特定轨道上形成了驻波,从而稳定存在。
\begin{figure}[ht!]
    \centering
    \begin{tikzpicture}[scale=1]
        \draw (0,0) circle (1);
        \fill (0,0) circle (0.1);
        \foreach \a in {0,...,360}
            \fill ($({cos(\a)},{sin(\a)}) + (\a:{0.3*sin(8*\a)})$) circle (0.5pt);
    \end{tikzpicture}
    \caption{波尔模型}
\end{figure}
因此只有轨道周长是电子波长的整数倍时才能够实现,并将这个条件称为\underline{量子条件}
\begin{equation*}
    2\pi r=n\lambda=n\cdot\frac{h}{mv}
\end{equation*}

\paragraph{氢原子轨道半径} 有核模型下电子围绕原子核做等速圆周运动,以库仑力作为向心力,因此
\begin{equation*}
    m\frac{v^2}{r}=k\frac{e^2}{r^2}
\end{equation*}
再结合量子条件可得
\begin{equation*}
    r_n=\frac{h^2}{4\pi^2kme^2}n^2
\end{equation*}
由此可见氢原子的轨道半径的确是随着整数$n$而离散变化的,而这些离散的轨道正是化学中所提及的电子壳层(KLM等)。

\paragraph{氢原子能级} 基于电子运动的动能以及库仑力作用下的电势能可求出每一能级下电子的能量
\begin{equation*}
    E=\frac12mv^2+k\frac{-e^2}{r}=-\frac{ke^2}{2r}
\end{equation*}
再代入轨道半径可整理得到
\begin{equation*}
    E_n=-\frac{2\pi^2k^2me^4}{h^2}\times\frac{1}{n^2}
\end{equation*}
可见氢原子的每个能级所对应的能量也是随着整数$n$而离散变化的。并且一般情况下电子会选择处于能量更小的能级,也就是所谓的\underline{基态}(基底状態),同时称非基底状态为\underline{激发态}(励起状態)。由能级公式可知,不同能级间的能量差为定值
\begin{equation*}
    E_m-E_n=\frac{2\pi^2k^2me^4}{h^2}\left(\frac{1}{n^2}-\frac{1}{m^2}\right)
\end{equation*}
因此,当核外电子从激发态回归到基态时只会释放出能量等同于两个能级能量差的光子,即
\begin{equation*}
    E_m-E_n=h\nu
\end{equation*}
相反,只有能量等于两个能级能量差的光子才会使处于基态的电子转移到对应的激发态。一般称这个过程为\underline{跃迁}。

虽然得益于波尔的猜想合理解释了关于氢原子的诸多疑点,比如其吸收光谱不连续等问题,但原子层面的问题却并未因此尘埃落定。事实上,该猜想只能够完美说明氢原子的情况,对于更加普遍也更加复杂的其他原子就有些力不从心了。

\subsection{放射性衰变}

\begin{figure}[ht!]
    \centering
    \begin{minipage}[c]{0.48\textwidth}
        \centering
        \begin{tikzpicture}[scale=0.6]
            \foreach \y in {0,1,2}
                \draw[-latex,color=gray,opacity=0.5] (0,\y) -- (3,\y);
            \node[right] at (3,1) {E};
            \draw[-latex] (1.5,-0.5) ..controls(1.7,1).. (2.5,2.5) node[right] {$\alpha$ ray};
            \draw[-latex] (1.5,-0.5) ..controls(1,0.75).. (0.5,1) node[left] {$\beta$ ray};
            \draw[-latex] (1.5,-0.5) -- (1.5,2.5) node[above] 
            {$\gamma$ ray};
        \end{tikzpicture}
    \end{minipage}
    \begin{minipage}[c]{0.48\textwidth}
        \centering
        \begin{tikzpicture}[scale=0.6]
            \foreach \x in {0.5,1.5,2.5,3.5} \foreach \z in {0.5,1.5,2.5}
                \draw[-latex,color=gray,opacity=0.5] (\x,0.5,\z) -- (\x,-0.5,\z);
            \node at (0,-0.5,2.5) {B};
            \draw[-latex] (2,0,2.5) ..controls(1,0,0.5).. (0.5,0,0) node[left] {$\alpha$ ray};
            \draw[-latex] (2,0,2.5) ..controls(2.5,0,1.5).. (3,0,1.5) node[right] {$\beta$ ray};
            \draw[-latex] (2,0,2.5) -- (2,0,0) node[right] {$\gamma$ ray};
        \end{tikzpicture}
    \end{minipage}
    \caption{放射线}
\end{figure}
处于不稳定状态的原子核通过放射粒子或是能量后变得稳定的过程为衰变(radioactive decay),大体有三种典型衰变模式。
\begin{figure}[ht!]
    \centering
    \renewcommand\arraystretch{1.2}
    \begin{tabular}{c|c|c|c}
        \hline
        &本质&穿透力&电离效果\\\hline
        $\alpha$衰变&\ce{^4_2He}原子核&低&高\\\hline
        $\beta$衰变&高速\ce{^0_{-1}e}&中&中\\\hline
        $\gamma$衰变&高能电磁波&高&低\\\hline
    \end{tabular}
    \caption{衰变一览}
\end{figure}
计算时可先使用前后质量只差算出$\alpha$衰变的次数,而后使用$\beta$衰变补全电荷的数量。
\begin{itemize}
    \item \ce{^226_88Ra -> ^222_86Rn + ^4_2He}
    \item \ce{^210_83Bi -> ^210_84Po + ^0_-1e-}
\end{itemize}
此外,一般用\underline{半衰期}来描述衰变的发生速度,即每过一个半衰期原子核的数量便会减少一半。
\begin{figure}[ht!]
    \centering
    \begin{tikzpicture}[scale=2.5]
        \draw[-latex] (-0.2,0) -- (1.5,0) node[right] {t};
        \draw[-latex] (0,-0.2) -- (0,1.2) node[above] {N};
        \draw[domain=0:1.4] plot[smooth] (\x,{0.2^\x});
        \draw[dashed] (0,1/2) node[left] {1/2} -| (0.43,0) node[below] {T};
        \draw[dashed] (0,1/4) node[left] {1/4} -| (0.86,0) node[below] {2T};
        \draw[dashed] (0,1/8) node[left] {1/8} -| (1.29,0) node[below] {3T};
    \end{tikzpicture}
    \caption{半衰期}
\end{figure}

\subsection{核能}

守恒一直是物理学家们的执着,和对称性一样被认为是自然的美。然而,随着粒子研究的深入,人们惊异地发现在原子核的层面上出现了质量不守恒的现象。即构成原子核和核子的总质量小于原子核所表现出的质量,并称该现象为\underline{质量亏损}。

在此爱因斯坦(美国)提出质量和能量的等价性,由此将质量守恒和能量守恒统一在了一起。
\begin{equation*}
    E=mc^2
\end{equation*}
此外,从物质的稳定性的角度也可以解释质量亏损的现象:物质总会自发地向能量低(稳定)的状态发展,因此不稳定的核子结合为稳定的原子核后能量(质量)会降低。

% chapter 5 section 3

\section{核反应}

\textbackslash\textbackslash TODO


    % appendix + lof
    \appendix
    % appendix

\chapter{追加内容}

\section{三角复习}

\subsection{弧度制}

\begin{itemize}
    \item 定义:
    \begin{equation*}
        \textrm{弧度}=\frac{\textrm{弧长}}{\textrm{半径}}
    \end{equation*}
    \item 特殊值:
    \begin{center}
        \renewcommand\arraystretch{1.2}
        \begin{tabular}{c|c|c|c}
            \hline
            角度         & 弧度        & 角度         & 弧度 \\\hline
            $30^\circ$  & $\frac\pi6$ & $120^\circ$  & $\frac{2\pi}3$\\
            $45^\circ$  & $\frac\pi4$ & $135^\circ$  & $\frac{3\pi}4$\\
            $60^\circ$  & $\frac\pi3$ & $150^\circ$  & $\frac{5\pi}6$\\
            $90^\circ$  & $\frac\pi2$ & $180^\circ$  & $\pi$\\
            \hline
        \end{tabular}
    \end{center}
\end{itemize}

\subsection{三角函数}

\begin{itemize}
    \item 正弦:$\frac{对边}{斜边}$
    \item 余弦:$\frac{邻边}{斜边}$
    \item 正切:$\frac{对边}{邻边}$
    \item 特殊值:
    \begin{center}
        \renewcommand\arraystretch{1.2}
        \begin{tabular}{c|ccc}
            \hline
            角度 & 正弦 & 余弦 & 正切 \\\hline
            $0^\circ$   & $0$              & $1$              & $0$              \\
            $30^\circ$  & $\frac12$        & $\frac{\sqrt3}2$ & $\frac1{\sqrt3}$ \\
            $45^\circ$  & $\frac{\sqrt2}2$ & $\frac{\sqrt2}2$ & $1$              \\
            $60^\circ$  & $\frac{\sqrt3}2$ & $\frac12$        & $\sqrt3$         \\
            $90^\circ$  & $1$              & $0$              & $\infty$         \\
            $180^\circ$ & $0$              & $-1$             & $\infty$         \\
            $\frac\pi2-\theta$ & $\cos\theta$ & $\sin\theta$ & $\frac1{\tan\theta}$\\
            \hline
        \end{tabular}
    \end{center}
\end{itemize}

\section{微分公式}

\subsection{基本初等函数的导数}

\begin{itemize}
    \item 常函数:$y^\prime=0$
    \item 幂函数:$y^\prime=a\cdot x^{a-1}$
    \item 指数函数:$y^\prime=a^x\cdot\ln a$
    \item 对数函数:$\frac1{x\ln a}$
    \item 正弦函数:$(\sin x)^\prime=\cos x$
    \item 余弦函数:$(\cos x)^\prime=-\sin x$
    \item 正切函数:$(\tan x)^\prime=\frac1{\cos^2x}$
\end{itemize}

\subsection{组合函数的导数}

\begin{itemize}
    \item $(f(x)\pm g(x))^\prime=f^\prime(x)\pm g^\prime(x)$
    \item $(f(x)\cdot g(x))^\prime=f^\prime(x)\cdot g(x)+f(x)\cdot g^\prime(x)$
    \item $(\frac{f(x)}{g(x)})^\prime=\frac{f^\prime(x)\cdot g(x)-f(x)\cdot g^\prime(x)}{g^2(x)}$
    \item $f(g(x))^\prime=\frac{df(x)}{dg(x)}\cdot\frac{dg(x)}{dx}=f^\prime(g(x))g^\prime(x)$
\end{itemize}

\section{积分公式}

\begin{itemize}
    \item 常函数:$\int kdx=kx+C$
    \item 幂函数($a\neq-1$):$\int x^adx=\frac{x^{a+1}}{a+1}+C$
    \item 幂函数($a=-1$):$\int x^{-1}dx=\ln|x|+C$
    \item 指数函数:$\int a^xdx=\frac{a^x}{\ln a}+C$
    \item 正弦函数:$\int\sin xdx=-\cos x+C$
    \item 余弦函数:$\int\cos xdx=\sin x+C$
\end{itemize}

\section{力学延伸}

\subsection{平抛运动抛物线验证}

\begin{equation*}
    \begin{cases}
        x=v_0t\\
        y=\frac12gt^2
    \end{cases}\implies
    y=\frac{g}{2{v_0}^2}x^2
\end{equation*}

\subsection{动能推导}

\begin{equation*}
    E_k=\int\vec{F}\cdot d\vec{s}
        =\int m\vec{a}\cdot d\vec{s}
        =\int m\vec{v}\cdot d\vec{v}
        =\frac12mv^2
\end{equation*}

\subsection{机械能守恒推导}

\begin{align*}
    \int_{t_1}^{t_2}mv\frac{dv}{dt}dt&=\int_{t_1}^{t_2}F\frac{dx}{dt}dt\\
    m\int_{v_1}^{v_2}vdv&=\int_{x_1}^{x_2}Fdx\\
    m\int_{v_1}^{v_2}vdv&=\int_{x_0}^{x_2}Fdx-\int_{x_0}^{x_1}Fdx\\
    \Delta E_k&=-\Delta U\\
    \Delta E_k+\Delta U&=0
\end{align*}

\subsection{重力势能推导}

\begin{equation*}
    E_p=W=\int_{-h}^0mg\cdot dx=mgh
\end{equation*}

\subsection{弹力势能推导}

\begin{equation*}
    E_p=W=\int_{x}^0-k\Delta x\cdot d\Delta x=\frac12kx^2
\end{equation*}

\subsection{动量与冲量关系推导}

\begin{equation*}
    \int_{t_1}^{t_2}Fdt
    =\int_{t_1}^{t_2}m\frac{dv}{dt}dt
    =\int_{v_1}^{v_2}mdv
    =mv_2-mv_1
\end{equation*}

\subsection{圆周运动向心加速度推导}

\begin{gather*}
    v=\left(\frac{dx}{dt},\frac{dy}{dt}\right)
    =(-r\omega\sin\omega t,r\omega\cos\omega t)\\
    a=\left(\frac{dv_x}{dt},\frac{dv_y}{dt}\right)
    =(-r\omega^2\cos\omega t,-r\omega^2\sin\omega t)
    =-\omega^2\vec{r}
\end{gather*}

\subsection{单摆周期推导}

\begin{equation*}
    F\approx -mg\sin\theta=-mg\cdot\frac{x}{l}=-\frac{mg}{l}x
\end{equation*}

\subsection{万有引力势能推导}

\begin{equation*}
    E_p=W=\int_r^\infty-G\frac{Mm}{x^2}dx=-G\frac{Mm}{r}
\end{equation*}

\section{热学延伸}

\subsection{理想气体内能公式推导}

如图,气体分子在边长为l的立方体容器内运动、三个方向上的速度均为$v$,假设其与容器壁做弹性碰撞,计算气体压强与分子运动速度的关系。
\begin{figure}[ht!]
    \centering
    \begin{tikzpicture}
        \filldraw[color=black, fill=gray!30, rounded corners=3pt] (0, 0) rectangle (2, 2);
        \node[below] at (1, 0) {$l$};
        \node[left] at (0, 1) {$l$};
        \draw[thick, -latex] (1, 1) -- node[above] {$v_x$} ++ (0.5, 0) ;
        \fill (1, 1) circle (2pt);
    \end{tikzpicture}
    \caption{气体分子运动与压强}
\end{figure}
由于气体分子在x方向上每折返一次就会与容器壁发生一次碰撞,所以$\Delta t$时间内会与容器壁碰撞$\frac{v\Delta t}{2l}$次。因此这段时间内容器壁受到来自气体分子的冲量为
\begin{equation*}
    2mv\cdot\frac{v_x\Delta t}{2l}=\frac{m{v_x}^2}{l}\Delta t
\end{equation*}
即相当于单个气体分子给予容器壁$F=\frac{m{v_x}^2}{l}$大小的力,相对应的就会有
\begin{equation*}
    P=\frac{F}{l^2}=\frac{m{v_x}^2}{l^3}=\frac{m{v_x}^2}{V}
\end{equation*}
大小的压强。此外,该气体分子也会在y和z方向上做同样的运动,所以其速度的平均值为$\overline{v^2}=\overline{{v_x}^2}+\overline{{v_y}^2}+\overline{{v_z}^2}$。同时,容器内倘若有N个这样的气体分子,那么其压强的总和为
\begin{equation*}
    P=N\cdot\frac{m{v_x}^2}{V}=\frac{Nm\overline{v^2}}{3V}
\end{equation*}
将上述压强值代入理想气体状态方程就可以得到1摩尔该气体分子平均速度的平方。
\begin{gather*}
    PV=nRT\\
    \frac{Nm\overline{v^2}}{3V}V=\frac{N}{N_A}RT\\
    \overline{v^2}=\frac{3RT}{M}
\end{gather*}
那么n摩尔该气体分子的总动能可求。
\begin{equation*}
    n\cdot\frac12M\overline{v^2}=\frac32nRT
\end{equation*}

\subsection{等温变化做功}

\begin{equation*}
    Q_\textrm{吸}=W_\textrm{した}=\int_{V_1}^{V_2}PdV=nRT\cdot\log\frac{V_2}{V_1}
\end{equation*}

\subsection{断热变化PV等式推导}

基于断热变化的热力学第一定律可得
\begin{gather*}
    \Delta U=-W_\textrm{した}\\
    nC_v\Delta T=-PdV
\end{gather*}
简做整理后对两侧分别关于T和V积分
\begin{gather*}
    \int\frac{dT}{T}=-\frac{R}{C_v}\int\frac{dV}{V}\\
    \log T=-\frac{R}{C_v}\log V+c^\prime\\
    T=V^{-\frac{R}{C_v}}\cdot c\\
    TV^{\frac{R}{C_v}}=c
\end{gather*}
在此定义$\gamma=\frac{C_p}{C_v}$,结合$C_p-C_v=R$即有
\begin{equation*}
    TV^{\gamma-1}=c
\end{equation*}
再结合理想气体状态方程就可以得到$PV^\gamma=const$的结论。

\section{波动延伸}

\subsection{波的解析式推导法二}

假设$t=0$时刻的波形为
\begin{equation*}
    y_0=A\sin\left(\frac{2\pi}{\lambda}x\right)
\end{equation*}
那么$t$时间后改图像会向右平移$vt$单位长度,即
\begin{equation*}
    y=A\sin\left(\frac{2\pi}{\lambda}\left(x-vt\right)\right)
    =A\sin\left(2\pi\left(\frac{x}{\lambda}-\frac{t}{T}\right)\right)
\end{equation*}

\subsection{差频公式推导}

设两束声波分别为$y_1=\sin((\omega+\theta)t)$和$y_2=\sin((\omega-\theta)t)$,其中$\omega$远大于$\theta$。那么,根据波的叠加原理可知合成波为
\begin{align*}
    y=&\sin((\omega+\theta)t)+\sin((\omega-\theta)t)\\
    =&(\sin\omega t\cos\theta t+\cos\omega t\sin\theta t)+(\sin\omega t\cos\theta t-\cos\omega t\sin\theta t)\\
    =&(2\cos\theta t)\sin\omega t
\end{align*}
此外,声音的强度与振幅的平方相关,所以合成波的波峰与波谷都会被人耳当成差频接收。因此差频的频率为合成波括号中控制振幅部分频率的两倍。
\begin{align*}
    f=&2\times\frac{\theta}{2\pi}\\
    =&\frac{\omega_1-\omega_2}{2\pi}\\
    =&f_1-f_2
\end{align*}

\subsection{光程公式推导}

使波长为$\lambda$的波在折射率为$n$的介质中传播$l$长的距离,那么在这段空间中存在$\frac{l}{\lambda^\prime}=\frac{nl}{\lambda}$个波。倘若把这些波换算到真空中($n=1$)中,则需要
\begin{equation*}
    L=\frac{l}{\lambda^\prime}\times\lambda=nl
\end{equation*}
的距离。称$L$为光程,满足$L=nl$。

\subsection{双缝干涉光程差推导}

\begin{equation*}
    |S_1P-S_2P|=S_2H\approx d\sin\theta\approx d\tan\theta=\frac{dx}{l}
\end{equation*}

\subsection{薄膜干涉光程差推导}

根据三角关系以及折射定律可知
\begin{equation*}
    \begin{cases}
        AH=AB\sin\phi\\
        BD=AB\sin\theta\\
        \sin\theta=n\sin\phi
    \end{cases}
\end{equation*}
即$n\cdot AH=BD$,$AH$与$BD$等光程。也就是说两束光到B点前的光程差为$HC+BC$。在此,将$BC$向下对称翻转得到$BC^\prime$,因而
\begin{equation*}
    HC+BC=HC+BC^\prime=HC^\prime=2d\cos\phi
\end{equation*}
由于这个距离是折射率为$n$的介质中的,所以其真空中等效的光程为$2nd\cos\phi$。

\subsection{牛顿环干涉光程差推导}

结合勾股定理关于$R$、$r$、$d$列方程即可。
\begin{align*}
    R^2=&r^2+(R-d)^2\\
    r^2=&(2R-d)d\\
    \downarrow&\quad R\gg d\\
    r^2=&2Rd
\end{align*}

\section{电磁延伸}

\subsection{点电荷电势推导}

\begin{gather*}
    U=\int_r^\infty k\frac{Qq}{x^2}dx=k\frac{Qq}{r}\\
    V=\frac{U}{q}=k\frac{Q}{r}
\end{gather*}

\subsection{电容器储能公式推导}

\begin{equation*}
    U=\int_0^Q\frac{q}{C}dq=\frac{Q^2}{2C}
\end{equation*}

\subsection{电容器储能损耗推导}

假设电路中除电容器以外的外阻为$R$,则回路方程为
\begin{equation*}
    E=RI+\frac{Q}{C}\implies
    \frac{dQ}{dt}=I=\frac{E-\frac{Q}{C}}{R}
\end{equation*}
对等式两边积分可得电荷量随时间变化的函数
\begin{gather*}
    \int\frac{dQ}{E-\frac{Q}{C}}=\int\frac{dt}{R}\\
    -C\log\left\lvert E-\frac{Q}{C}\right\rvert=\frac{t}{R}+\gamma\\
    E-\frac{Q}{C}=\Gamma\exp\left(\frac{t}{RC}\right)\quad\left(\Gamma=\exp\left(-\frac{\gamma}{C}\right)\right)\\
\end{gather*}
根据初始条件$(t,Q)=(0,0)$可得$\Gamma=E$,因此
\begin{equation*}
    Q(t)=CE\left(1-\exp\left(-\frac{t}{RC}\right)\right)
\end{equation*}
最后使用$W=I^2Rt$积分求解即可
\begin{gather*}
    I=\frac{dQ}{dt}=\frac{E}{R}\exp\left(-\frac{t}{RC}\right)\\
    W=\int_0^\infty I^2Rdt=\frac12CE^2
\end{gather*}

\subsection{直线电流磁场推导}

根据Biot-Savart定律可得
\begin{align*}
    \mathbf{B}=&\frac{\mu_0}{4\pi}\int_{-\infty}^{\infty}\mathbf{J}\times\frac{\mathbf{r}-\mathbf{l}}{|\mathbf{r}-\mathbf{l}|^3}d^3l\\
    =&\frac{\mu_0I}{4\pi}\int_{-\infty}^{\infty}d\mathbf{l}\times\frac{\mathbf{r}-\mathbf{l}}{|\mathbf{r}-\mathbf{l}|^3}
\end{align*}
设$(0,0,r_z)$为矢量基准点,则可简记$r=(r_x,r_y,0)$
\begin{align*}
    \mathbf{B}=&\frac{\mu_0I}{4\pi}\int_{-\infty}^{\infty}dl\cdot e_z\times\frac{r_xe_x+r_ye_y-le_z}{|r^2+l^2|^{3/2}}\\
    =&\frac{\mu_0I}{4\pi}\int_{-\infty}^{\infty}dl\frac{r_xe_y-r_ye_x}{|r^2+l^2|^{3/2}}
\end{align*}
关于$l$求解反常积分可得
\begin{align*}
    \mathbf{B}=&\frac{\mu_0I}{4\pi}\left[\frac{l(r_xe_y-r_ye_x)}{r^2\sqrt{r^2+l^2}}\right]_{-\infty}^{\infty}\\
    =&\frac{\mu_0I}{2\pi r^2}(r_xe_y-r_ye_x)
\end{align*}
将结果改写为位置向量的形式
\begin{equation*}
    \mathbf{B}=\frac{\mu_0I}{2\pi r^2}(-r_y,r_x,0)
\end{equation*}
即大小为$B=\frac{\mu_0I}{2\pi r}$,方向与$\mathbf{r}$垂直的磁场

\subsection{环形电流磁场推导}

设位于xy平面上的环形电流半径为$r$,计算z轴上点$\mathbf{z}=(0,0,z)$处的磁场。则根据Biot-Savart定律$\mathbf{l}=(l\cos\phi,l\sin\phi,0)$处的微小电流产生的磁场即为
\begin{align*}
    d\mathbf{B}=&\frac{\mu_0I}{4\pi}\frac{d\mathbf{l}\times(\mathbf{z}-\mathbf{l})}{|\mathbf{z}-\mathbf{l}|^3}\\
    =&\frac{\mu_0I}{4\pi|\mathbf{z}-\mathbf{l}|^3}(-\sin\phi,\cos\phi,0)dl\times(-l\cos\phi,-l\sin\phi,r)\\
    =&\frac{\mu_0I}{4\pi|\mathbf{z}-\mathbf{l}|^3}(r\cos\phi,r\sin\phi,l)dl
\end{align*}
分别关于xyz三个方向积分可得
\begin{align*}
    \begin{cases}
        B_x=\frac{\mu_0}{4\pi}\oint\frac{z\cos\phi}{|\mathbf{z}-\mathbf{l}|^3}dl
        =\frac{\mu_0}{4\pi}\int_0^{2\pi}\frac{z\cos\phi}{|\mathbf{z}-\mathbf{l}|^3}ld\phi=0\\
        B_y=\frac{\mu_0}{4\pi}\oint\frac{z\sin\phi}{|\mathbf{z}-\mathbf{l}|^3}dl=0\\
        B_z=\frac{\mu_0}{4\pi}\oint\frac{l}{|\mathbf{z}-\mathbf{l}|^3}dl
        =\frac{\mu_0}{4\pi}\oint\frac{ldl}{(z^2+l^2)^{3/2}}=\frac{\mu_0Il^2}{2(z^2+l^2)^{3/2}}\\
    \end{cases}
\end{align*}
可见除$B_z$以外的磁场均为0,并且当$z=0$时$B_z=\frac{\mu_0I}{2l}$的确成立

\chapter{历年考点}

\section{年份题号对照表}

\begin{changemargin}{-2.5cm}{-2.5cm}
    \begin{center}
        \small
        \renewcommand\arraystretch{1.2}
        \begin{tabular}{c*{19}{|c}}
            \hline
            章节&\multicolumn{6}{|c}{力学}&\multicolumn{3}{|c}{热学}&\multicolumn{3}{|c}{波动}&\multicolumn{6}{|c}{电磁}&\multicolumn{1}{|c}{原子}\\\hline
            题号&1&2&3&4&5&6&7&8&9&10&11&12&13&14&15&16&17&18&19\\\hline
            21A&1.2&2.2&1.2&2.2&3.2&3.1&
                1&2.2&2.3&
                1.1&2.3&3.1&
                1.1&1.3&2.1&3.1&3.2&4.1&
                3\\\hline
            20B&1.3&1.2&2.2&2&3.1&3.3&
                1&2.1&2.3&
                1.1&2.3&3.1&
                1.1&1.2&1.3&2.3&3.3&4.1&
                2\\\hline
            19A&1.3&1.2&1.2&2&3.2&3.3&
                1&2.1&2.2&
                1.1&2.3&3.1&
                1.1&1.1&2.1&1.3&3.1&3.3&
                -\\\hline
            18B&1.2&1.1&2.1&2.2&3.1&3.2&
                1&2.1&2.2&
                1.1&2.2&3.2&
                1&1.3&2.3&3.2&3.3&4.1&
                2\\\hline
            18A&1.2&1.2&2.2&2.2&2.1&3.2&
                1&2.1&-&
                1.1&2.2&3.1&
                1.1&1.3&2.3&3.1&3.2&4.1&
                2\\\hline
            17B&1.2&1.2&2.1&2&2.1&3.2&
                1&2.1&2.3&
                1.1&2.2&3.2&
                1.1&1.2&1.3&2.1&3.1&3.2&
                3\\\hline
            17A&1.2&1.2&2.2&3.1&2&3.3&
                1&2.1&2.2&
                1.1&2.3&3.1&
                1.1&1.3&2.3&3.1&3.2&4.1&
                3\\\hline
            16B&1.2&1.1&1.2&2&2.2&3.1&
                1&2.2&2.3&
                1.1&2.2&3.1&
                1.1&1.2&1.3&2.3&2.3&3.2&
                3\\\hline
            16A&1.2&1.2&2&2.2&3.2&3.1&
                1&2.1&2.3&
                1.3&2.2&3.2&
                1.1&1.2&2.1&1.3&3.1&3.3&
                2\\\hline
            15B&1.3&1.1&1.2&2.1&2.2&3.1&
                1&2.2&2.2&
                1.3&1.2&3.2&
                1.2&1.3&2.1&3.1&3.2&4.1&
                1\\\hline
            15A&1.3&1.2&2.2&2.1&2.2&3.1&
                1&2.2&2.3&
                1.1&1.2&3.1&
                1.1&1.3&2.3&3.1&3.2&4.1&
                1\\\hline
        \end{tabular}
    \end{center}
\end{changemargin}
以上为2015年至今所有真题的知识点一览表,每个单元格内是对应的小节号。

\section{章节频次对照表}

\begin{center}
    \renewcommand\arraystretch{1.2}
    \begin{minipage}{0.48\textwidth}
        \centering
        \begin{tabular}{c|c|c}
            \hline
            小节&内容&次数\\\hline
            \multicolumn{3}{c}{力学}\\\hline
            1.1.1&速度加速度&3\\\hline
            1.1.2&运动与力&18\\\hline
            1.1.3&刚体与力&4\\\hline
            1.2.1&能量&12\\\hline
            1.2.2&动量&18\\\hline
            1.3.1&圆周运动&8\\\hline
            1.3.2&简谐振动&6\\\hline
            1.3.3&天体运动&3\\\hline
            \multicolumn{3}{c}{热学}\\\hline
            2.1&热与能量&11\\\hline
            2.2.1&气体法则&7\\\hline
            2.2.2&热力学第一定律&8\\\hline
            2.2.3&气体状态变化&6\\\hline
            \multicolumn{3}{c}{波动}\\\hline
            3.1.1&波的传播&9\\\hline
            3.1.2&波的干涉&2\\\hline
            3.1.3&衍射·反射·折射&2\\\hline
            3.2.1&声波&0\\\hline
        \end{tabular}
    \end{minipage}
    \begin{minipage}{0.48\textwidth}
        \centering
        \begin{tabular}{c|c|c}
            \hline
            小节&内容&次数\\\hline
            3.2.2&多普勒效应&5\\\hline
            3.2.3&共振现象&4\\\hline
            3.3.1&光的折射&7\\\hline
            3.3.2&光的干涉&4\\\hline
            \multicolumn{3}{c}{电磁}\\\hline
            4.1.1&电场&11\\\hline
            4.1.2&电势&6\\\hline
            4.1.3&电容器&11\\\hline
            4.2.1&欧姆定律&5\\\hline
            4.2.3&直流电路&7\\\hline
            4.3.1&磁场&8\\\hline
            4.3.2&安培力&7\\\hline
            4.3.3&洛伦兹力&4\\\hline
            4.4.1&电磁感应&7\\\hline
            4.4.2&交流电&0\\\hline
            \multicolumn{3}{c}{原子}\\\hline
            5.1&原子模型&2\\\hline
            5.2&光与电磁波&4\\\hline
            5.3&核反应&4\\\hline
        \end{tabular}
    \end{minipage}
\end{center}

\chapter{配图一览}

\makeatletter
\begin{multicols}{2}
    \@starttoc{lof}
\end{multicols}
\makeatother

    % postscript
    \backmatter
    % back matter

\chapter{后记}

\section*{推荐书目}
\begin{itemize}
    \item 物理教室(ISBN:978-4777213757)
    \item チャート式シリーズ新物理物理基礎・物理(ISBN:978-4410118425)
    \item 物理[物理基礎・物理]入門問題精講(ISBN:978-4010347126)
    \item 物理[物理基礎・物理]基礎問題精講(ISBN:978-4010347096)
    \item 物理[物理基礎・物理]標準問題精講(ISBN:978-4010347287)
    \item 実戦物理重要問題集 物理基礎・物理(ISBN:978-4410143120)
    \item 物理のエッセンス 力学・波動(ISBN:978-4777213559)
    \item 物理のエッセンス 熱・電磁気・原子(ISBN:978-4777213566)
    \item 漆原の物理(物理基礎・物理)明快解法講座(ISBN:978-4010346136)
    \item 良問の風物理頻出・標準入試問題集(ISBN:978-4777213658)
    \item \ldots
\end{itemize}

\section*{初稿感想}
自项目启动至今,零零散散花了大约半个多月的时间完成了主要部分的初稿,整体初见规模。个人觉得作为首次\LaTeX 项目整体还是十分成功的,也实现了当初的目标:不仅熟悉了基本的文档框架,而且也练习了Tikz的绘图。除此之外也同时学习了许多相关联的内容,比如:beamer的使用、开源知识共享协议等等。如今面对\LaTeX 编辑相关的问题虽不至于无懈可击,但已经能够做到来者不拒了。最后,由于近期即将面临考研相关的复习准备工作,本文档也足够基本使用,余下的内容就暂且搁置,留到事后慢慢更新了。

\begin{flushright}
    PENG AO\\
    2022-05-17 in Tokyo
\end{flushright}

\end{document}
