% chapter 4 section 2

\section{电流}

\subsection{欧姆定律}

本节内容较为基础,在此只逐条列举重要知识点。

\subsubsection{电流}

\begin{itemize}
    \item 描述:由电荷的定向移动形成
    \item 方向:正电荷的流动方向,或者负电荷流动的反方向
    \item 大小:单位时间内流过导体截面的电量
    \item 定义式1(宏观):
    \begin{equation*}
        I=\frac{\Delta Q}{\Delta t}
    \end{equation*}
    \item 定义式2(微观)\footnote{n为单位体积的电子密度,v为电子移动速度,S为导体截面积}:
    \begin{equation*}
        I=neSv
    \end{equation*}
    \item 单位:安培(A)
\end{itemize}

\subsubsection{欧姆定律}

\begin{equation*}
    V=IR
\end{equation*}
\begin{itemize}
    \item 描述:流过导体的电流与加在其两端电压的大小成正比
    \item 电阻R:日文为電気抵抗
    \begin{itemize}
        \item 单位:欧姆($\Omega$)
        \item 定义式:$R=\rho\frac{l}{S}$
    \end{itemize}
\end{itemize}

\subsubsection{焦耳热}

\begin{itemize}
    \item 本质:电场力对电子的做功
    \item 公式:
    \begin{equation*}
        Q=VIt=I^2Rt=\frac{V^2}{R}t
    \end{equation*}
    \item (电)功率:
    \begin{equation*}
        W=VI=I^2R=\frac{V^2}{R}
    \end{equation*}
\end{itemize}

\subsection{直流电路}

\textbackslash\textbackslash TODO
