% back matter

\chapter{后记}

\section*{初稿感想}
自项目启动至今,零零散散花了大约半个多月的时间完成了主要部分的初稿,整体初见规模。个人觉得作为首次\LaTeX 项目整体还是十分成功的,也实现了当初的目标:不仅熟悉了基本的文档框架,而且也练习了Tikz的绘图。除此之外也同时学习了许多相关联的内容,比如:beamer的使用、开源知识共享协议等等。如今面对\LaTeX 编辑相关的问题虽不至于无懈可击,但已经能够做到来者不拒了。最后,由于近期即将面临考研相关的复习准备工作,本文档也足够基本使用,余下的内容就暂且搁置,留到事后慢慢更新了。

\begin{flushright}
    PENG AO\\
    \formatdate{17}{5}{2022} in Tokyo
\end{flushright}

\section*{二稿感想}

如今二稿完成,已然深秋,本人也完成了考研,顺利地进入了东京大学情报理工学系研究科,继续学习计算机科学。三万余字的成稿也令人颇有成就感。自初稿以来有所增删,彻底完成了第四章与第五章的内容,也将留学生统一考试考纲修订后的真题略解电子化了。算是做到了有始有终。实际使用过程中,这份文档的确带来了所预期的便利,同时也意外地对同学们的学习起到了激励的作用,我也因此倍感本项目的价值所在。然而留学生统一考试终究是我的过往,尽管曾登峰造极,但也绝不应制作本文档而贪乐其中、忘乎所以。正所谓“往者不可谏,来者犹可追”,本文档的大规模编辑至此便告一段落,日后仅根据需求做小幅调整,避免个人疏漏误导使用的读者。

\begin{flushright}
    PENG AO\\
    \formatdate{30}{10}{2022} in Tokyo
\end{flushright}
